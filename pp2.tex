% top stuff
\documentclass{amsart}
%
%
% packages
\usepackage{pgfplots} % for plots - loads tikz, wc itself loads xcolor, graphicx
\usepackage{tikz-cd} % for graphs - loads tikz too
\usepackage[figuresright]{rotating} % to rotate table w LHS at bottom
\usepackage{hyperref} % for links - load last-ish
\usepackage{amsrefs} % for full cites - use author-year option? - load last
%
%
% new commands to avoid text formatting inline
\newcommand{\code}{\texttt} % for code
\newcommand{\lit}{\textit} % for book titles etc
\newcommand{\ship}{\textit} % for ship names etc
\newcommand{\ment}{\textit} % to mention (not use) terms
\newcommand{\forn}{\textit} % for foreign words
%
%
% new commands to avoid maths formatting inline
\DeclareMathOperator{\ab}{Ab} % place variable
\DeclareMathOperator{\cd}{Cd} % ^
\DeclareMathOperator{\id}{Id} % place name
\DeclareMathOperator{\mt}{Mt} % ^
\DeclareMathOperator{\bc}{Bc} % ^
\DeclareMathOperator{\et}{Et} % ^
\DeclareMathOperator{\nt}{Nt} % ^
\DeclareMathOperator{\pre}{pre} % time period name
\DeclareMathOperator{\pst}{pst} % ^
\DeclareMathOperator{\cpd}{cpd} % ^
%
%
% amsthm styles where (i) mainclm > minrclm = gloss > note = bodytext > footnote, or (ii) clm >= gloss > etc
% plain style
\newtheorem{mainclm}{Main Claim} % for main claims
\newtheorem{minrclm}{Minor Claim} % for minor claims
\newtheorem{clm}{Claim} % for claims
% definition style
\theoremstyle{definition}
\newtheorem{gloss}{Gloss} % for glosses
% remark style
\theoremstyle{remark}
\newtheorem*{note}{Note} % for notes
%
%
% amsthm style for proof
\newenvironment{steps}{\begin{proof}[Steps]}{\end{proof}} % for steps
\renewcommand{\qedsymbol}{\textit{End}.} % for all proof env
%
%
% tikz ie pgsplots, tikzcd commands
\pgfplotsset{compat=1.18}
\usepgfplotslibrary{groupplots} % to group plots
\pgfmathdeclarefunction{gauss}{2}{\pgfmathparse{1/(#2*sqrt(2*pi))*exp(-((x-#1)^2)/(2*#2^2))}} % to use in \addplot
%\tikzcdset{} % global graphs style
%
%
% xcolor commands for darkmode - just for draft
\pagecolor{darkgray}
\color{white}
%
%
% hyperref commands
\hypersetup{pdfauthor={Angel E Navidad},pdftitle={On the Origins of Belize},pdfsubject={},pdfkeywords={}} % pdf metadata
%
%
\begin{document}
%
%
% starting stuff
\title{On the Origins of Belize}
\author{A Navidad}
\address{Harvard College, Cambridge, MA, US}
\curraddr{Benque Viejo, Cayo, Belize}
\email{navidad@college.harvard.edu}
\date{12 Feb 2025}
\thanks{} % incl non-grant support per ams
\begin{abstract}
\end{abstract}
\keywords{}
\maketitle
%
%
%
%
%
% sec intro
\section{Introduction}
\label{s:intro}
	The origins of Belize have been an open problem in historical literature since at least the 18th century, and remain so today, with over two dozen theories proposed so far, and little consensus to boot.%
	\footnote{Belize is the only pirate settlement in the Caribbean, and only logwood settlement, to have survived past the 19th century. As such, it is the only country in the Americas for which this problem is open. \emph{All} pirate or logwood settlements seem to have unclear origins though, including at least the Bay Islands, Campeachy, Catoche, the Mosquito Shore, and Tortuga. The preceding uniqueness claim is from \cite{res19}*{p~6}, but has previously been made in literature for the same or smaller comparison classes. The generality claim has prior been made for at least some pirate or logwood settlements, but seems to hold for all of them. Theories for Belize listed in Table~\ref{tab:literature}. Note this table is not a systematic literature review, and further note we claim only that given readings are possible or acceptable. All data for this paper deposited at \url{https://github.com/aenavidad/pp-0/}. In particular, Table~\ref{tab:literature} data in \href{https://github.com/aenavidad/pp-0/tree/main/literature}{\nolinkurl{/literature}}.}
	% tab literature
	% tab literature
\begin{table} % sidewaystable is no go in amsart apparantly
\caption{Stories* in 18th to 21st cent literature.}
\label{tab:literature}
% missg long (s596) + s1644 rejection
% tab code begins
\begin{sideways}
\begin{tabular}{cp{.4\textheight}lp{.3\textheight}cc}
No	&Date	&In	&Claim	&Yes	&No\\
(1)	&mid 16th cent	&Old	&Eng seek haven	&2	&0\\ % campbell - locn:St George's - yes (s2316, s2320) - no ()
(2)	&ca Sp Hon flotilla est	&Old	&Wallace settles	&	&\\ % asturias - first in live wiki
(3)	&ca 1610	&Old&Wallace settles	&&\\ % egli
(4)	&pre/ca Fuensalida missions	&N	&Brit log	&&\\ % roys
(5)	&in 1617	&--	&Wallace present&&\\ % asturias
(6)	&pre penult Bacl sacking	&Old	&Eng log	&&\\ % reads 1732
(7)	&ca Prov est	&S	&Provers seek refuge or farm	&&\\ % winzerling
(8)	&mid 17th cent	&--	&Wallace settles	&--	&1\\ % fenix 1849 - yes () - no (s1644)
(9)	&pre treaty	&?	&Brit present	&&\\ % mcculloch
(10)	&mid third 17th cent	&--	&Wallace present	&&\\ % ancona
(11)	&in 1638	&--	&Brit wreck	&&\\ % HA 1829
(12)	&in 1638	&--	&Wallace wrecks	&--	&1\\ % HA 1827 - yes () - no (s1644)
(13)	&in 1640	&--	&Wallace wrecks	&&\\ % ungewitter
(14)	&in Cromwell govt	&--	&Eng seek haven	&--	&1\\ % HA 1826 - yes () - no (s1644)
(15)	&post Bacl/Trux sacking or post Camp logging	&--	&Brit seek haven or log	&&\\ % b-thomas
(16)	&last half 17th cent or pre early 1680s	&--	&Eng present	&&\\ % finamore
(17)	&post Jam invasion or post Camp logging	&--	&Jamers log	&&\\ % robertson
(18)	&ca 1663	&--	&Wallace logs	&&\\ % g diaz
(19)	&in 1662	 or pre 1670&--	&Wallace logs	&--	&1\\ % c quijano - yes () - no (s2316)
(20)	&post Catoche/Camp logging	&--	&Jamers log	&&\\ % ancona
(21)	&post treaty	&--	&Wallace present	&&\\ % n ortega
(22)	&last third 17th cent	&--	&Eng present	&&\\ % ancona
(23)	&early 18th cent	&--	&Eng seek haven	&&\\ % brockhaus
(24)	&post Camp logging	&--	&Wallace seeks haven or logs	&&\\ % ancona - last in live wiki
(25)	&post Camp logging	&Old	&Eng log	&1	&0 % restall - yes (s1644) - no ()
\end{tabular}
\end{sideways}
% tab code ends
\end{table}
%
 % \label{tab:literature} - may swap in 'From' (\cite values) for 'Source' col - may remove self-citation from 'Cit' tallies
	
	This paper seeks to either complicate or clarify matters by presenting existing and possible understandings of the problem in Section~\ref{s:problem}, representations of the historical record in Section~\ref{s:models}, and theories in Section~\ref{s:theories}.%
	\footnote{For appropriately qualified (not strict) \ment{existing} or \ment{possible} understandings or representations or theories, eg per our readings of literature, historically relevant, non-trivial, seemingly viable, so on. In this paper, we naively imagine or sketch historical reasoning as first taking a problem, then carving or refining it into well-specified questions, then representing or modelling the historical record, and finally using this model to answer the posed questions, where such answers hold \emph{for} the questions as so specified and \emph{for} the model as so built. We claim only that this is a possible or acceptable naive sketch, not that it is uniquely so, nor that this is \emph{actually} how historians have previously tackled the problem of the origins of Belize. We think the naive sketch might help to get a grip of the problem, make sense of current discordance in literature, or see and fill in gaps therein. This paper uses the name \ment{Belize} anachronistically, and does \emph{not} regard the related but distinct problem of the \emph{name's} origins. The problem at hand seems to have been first noted by \cite{alm26}*{p~5} (implicitly), first discussed by \cite{nun77}*{pp~6--9, 12--14} (obliquely), and most recently discussed by \citelist{\cite{bul16} \cite{cam09} \cite{res19}}.} % may add note re weak vs strong empircal claims made for problem/models/theories - may note historical names used - may note interpretation is carving plus representation (cf Notes)
%
%
%
%
% sec problem
\section{Problem}
\label{s:problem}
	In this section, we characterise the problem in Section~\ref{ss:carving}, as it is currently \emph{and} as it may possibly be understood, and note some salient features in Section~\ref{ss:features}.%
	\footnote{The carving given here is not fully realised, but seems straightforward and useful enough as is. No uniqueness (`this is the \emph{only} carving available') claim made. A weak empirical claim \emph{is} made though, ie that the given carving \emph{is} how at least some historians understand the problem, subject to our possible or acceptable readings of literature. No strong empirical (`this is \emph{actually} how most historians have previously carved') claims made.}
	%
	%
	% subsec problem carving
	\subsection{Carving}
	\label{ss:carving}
		Let \ment{Honduras} be the earliest predecessor state or polity of modern Belize, and say it first emerged or came to be at some time \(t_{\og}\).%
		\footnote{\ment{Bay} also appropriate here. For simplicity, say modern Belize had only two predecessor states or polities, British Honduras and Honduras, where the former succeeded the latter in 1862.}
		Then Belize did \emph{not} exist at any time before \(t_{\og}\), and \emph{did} exist at least at some times after \(t_{\og}\).\footnote{Allowing for state or polity discontinuity.} In which case, we say the problem of the origins of Belize is a class of questions pertinent to this happening or process.%
		\footnote{Appropriately qualified, eg class is non-empty and finite, questions are direct or non-embedded and immediately or straightforwardly pertinent, so on. We imagine the pertinent questions are at least `What was Honduras?,' `Where was Honduras?,' `When was \(t_{\og}\)?,' and `How did Honduras come to be at \(t_{\og}\)?'}
		And a theory for this problem is a class of answers to those questions.%
		\footnote{Appropriately qualified, eg class has an answer for each required question, answers are immediate or direct, so on.}
	%
	%
	% subsec problem features
	\subsection{Features}
	\label{ss:features}
		We note this carving requires at least concepts of \ment{state} or \ment{polity} and their emergence and existence and succession, but does not specify them well.%
		\footnote{It seems to require binary concepts (some thing is or is not a state or polity, and does or does not exist, and cannot \emph{sort of} nor \emph{partially} be one nor exist). For simplicity, require that all concepts be binary, but let \ment{state} be a strict subconcept of \ment{polity} where the latter includes things which are kind of but not properly states eg state-like entities or quasi-states (leaving the state-non-state boundary in \ment{polity} vague or fuzzy). We owe this allowance to \cite{cam11}*{pp~95--96}, who credits \cite{luk50}*{p~50}.}
		To that extent, it is barely realised, and so remains open to a number of distinct senses or specifications of these concepts.%
		\footnote{We mainly have in mind here existing or possibly useful senses of \ment{polity} and their emergence, so disregard existence and succession. For polities, these seem to be state vs non-state, small vs large (ie simple vs complex ie insignificant vs significant, including spatiotemporally, demographically, economically, institutionally, so on), continuous vs discontinuous (including spatially, temporally, voluntarily vs involuntarily). For emergence, these seem to be intentional vs accidental, in compressed vs over prolonged time, via single vs various acts, of corporate vs collective aspects of polity. We note that some of these senses have already been disallowed by this carving eg some specifications of largeness or complexity or significance of polities.} % may gloss these senses - call this a 'minimal/blunt' carving, and say anything more specified (eg admitting only states, etc) is 'maximal/sharp' carving - may add contin/discont of collective vs corporate aspects
	
		Further, let a \ment{vague} problem be one where, given a \emph{complete} historical record, at least some of the particular senses of the required concepts give us discordant theories.%
		\footnote{Appropriately qualified, eg record is relevant, discordance is significant, so on. The idea being that the historical record itself might not straightforwardly close the problem for us. Examples of non-vague (closed) problems are those of the origins of Virginia and St Kitt's \& Nevis. Regardless of what we mean by \ment{polity}, \ment{polity emergence}, so on, theories for those problems invariably agree on the relevant happenings and date them to 1607 and 1623, respectively. In contrast, theories for a vague problem would \emph{not} so invariably agree (unless trivially).}
		Then we say the problem at hand is vague.%
		\footnote{Further, say the problem's vagueness is \ment{artificial} if the non-extant historical record would dissolve the vagueness, and say it is \ment{inherent} if it would not. Then we say the problem might even be \emph{inherently} vague. We note the problem seems to be vague and possibly inherently so for all pirate or logwood settlements. The preceding vagueness claim has previously been made in literature for at least some of these polities. As in the case of Belize though, some literature is written in a way which suggests there is no such vagueness (eg saying Campeachy was settled \emph{in} 1662, despite this strictly being the upper bound of settlement). This is more often than not done without support, and so is not counted as a substantive denial of vagueness, rather merely as a convenience.}
		Further, let a \ment{minimal} theory be one which does \emph{not} presuppose or fix particular senses of the required concepts.%
		\footnote{Appropriately qualified, eg has no trivial answers, so on.} % may fix minimal def eg minimal wrt carving or wrt model
		Then we say we currently have very few minimal theories for the problem at hand.%
		\footnote{These might be desireable, given the current glut of maximal theories in literature, ie of theories which \emph{do} presuppose or fix particular senses of the required concepts. Arguably though, the historian's job \emph{is} to provide such maximal theories and say `these are the senses we ought to use for this problem.' In which case, we might at least want \emph{transparent} theories, eg ones which flag the senses in use, give us reasons to accept their proffered senses, to reject competing ones, or at least which note competing senses exist. Then we say we do not seem to have very many transparent theories in literature either.} % may note bul16/etc comments re state of non-specialist works (cf Notes)
%
%
%
%
% sec models
\section{Models}
\label{s:models}
	In this section, we represent or model the relevant, extant historical record, as it is currently \emph{and} as it may possibly be modelled, spatially in Section~\ref{ss:spatial}, and temporally in Section~\ref{ss:temporal}.%
	\footnote{No uniqueness claim made. A weak empirical claim \emph{is} made, ie that the given models \emph{are} how at least some current theories in literature represent the historical record (contingent on our possible or acceptable readings of literature). We mainly have in mind theories which seem viable, but do not mean to discard any from the outset. Wallacian theories pose a difficulty here, given their popularity (cf Table~\ref{tab:literature}) and at least partial unfoundedness. \citelist{\cite{cam09}*{pp~72--110} \cite{bul16}*{pp~137--151} \cite{res19}*{pp~23--28}} have thankfully shown Wallace is most likely apocryphal. We do \emph{not} disregard such theories on this basis, however (though we \emph{do} end up disregarding most of them later on, for a distinct but related reason), as we imagine the following might be the case for at least some of them. Say some historian carefully reviewed the relevant archives, and concluded from this that there was heightened piratical activity in the Bay of Honduras in the mid-third of the 17th century, followed by increasingly significant logging activity in the last third of the same. And say they then came across some work claiming Wallace landed and stayed on the Old River in 1638. Then they might just give us a Wallacian theory, but the Wallace portion therein would \emph{not} be necessary, such that if the aforementioned work later proved unfounded, we might simply excise it from the theory, eg by ignoring biographical details, and reading \ment{Wallace} symbolically ie as \ment{one of the first Baymen}. We think this ought to be done, eg for Meridian theories, including \citelist{\cite{fen49}*{p~3} \cite{car71}*{pp~55, 210} \cite{car78}*{pp~260--261} \cite{anc78}*{pp~370--376} \cite{pen69}*{pp~217--219}}, given the non-zero chance of unique, non-extant primary sources in Merida in and after the \ment{Meridian War} (the frequent Anglo-Spanish hostilities from 1690s to 1790s). This is to say, we do \emph{not} think Wallace's being apocryphal (by itself) significantly pares down the number of viable theories in literature, nor restricts the number of possible ones (eg as in \cite{bul16}*{pp~137--138, 145, 151--156}). Further, as charges of dishonesty in Wallacian theories by \citelist{\cite{bul16}*{pp~138, 140--141} \cite{cam09}*{pp~87--88, 105--106} \cite{res19}*{pp~27--28}} regard \ment{Wallace} properly (not symbolically) read, they are likewise disregarded here (though we agree that almost all of them are ideologically biased). We note that the Meridian War does not seem to have been properly named before, though it may and seems like it ought to be (not discussed here).} % may note 'Mer War' ought to be properly named (but not discussed here) - may note proper names uncommon in lit, even when fitting or useful (ie name vs description, or proper vs improper name cf Notes)
	
	We start by listing happenings or phenomena (and relations among them) commonly represented in literature.%
	\footnote{To see what parts of the record are or are not deemed relevant.}
	We find that the most common are piracy and logging, where the former (somehow) led to the latter.%
	\footnote{Appropriately qualified, eg \ment{piracy} by resident or non-resident Dutch or English or Frenchmen, \ment{logging} of any dyewood or of only logwood, so on. Most (but not all) existing theories seem to represent \emph{both} phenomena \emph{and} the relation. Further, only phenomena or relations related to these three seem common in literature, eg social or legal pressures for or against piracy, so on. A notable class of existing theories which sometimes do not represent all three are \ment{normative} theories where agents simply land and immediately settle and found a state in some given year. Incidentally (or unsurprisingly), most Wallacian theories seem to be normative ones. We owe this class to \cite{cam11}*{pp~95--96}, who credits \cite{luk50}*{p~50}.}
	So models must represent at least these two phenomena and one relation.%
	\footnote{Spatially, it looks like we only need to consider piracy, as it seems to subsume logging, while the relation does not seem to be so extended. Temporally, we consider all three.}
	%
	%
	% subsec spatial models
	\subsection{Spatial}
	\label{ss:spatial}
		We seem to need only one spatial model, namely, \ref{eqn:map}.%
		\footnote{The Bay and Atlantic surroundings (up to some 800--900 nautical miles from centroid) mapped from northwest to southeast. For vertices as places, and edges as a certain relation among them. }
		% eqn map
		\begin{equation}
		\tag{Map}
		\label{eqn:map}
		\begin{tikzcd}[arrows=dash]
		\nt &\bc \rar \ar[rr,bend left] &\mt \rar &\id &\et
		\end{tikzcd}
		\end{equation}
	
		We start by listing places (and their relations) commonly represented in literature.%
		\footnote{For places, Bacalar ie north-central Belize and the Bay Islands in the Bay, plus Campeachy (including Terminos and Tris), Catoche (including Cozumel and possibly Ascension), Jamaica, south-central Mosquito Shore, Old Providence, and Tortuga (including north-west Hispaniola) outside of it. For relations, distance.}
		And note that these seem to be places in or about the Bay of Honduras which pirates are thought to have frequented at least in the 17th century.%
		\footnote{For simplicity, we say all these places are pirate haunts, and further say all logging stands were pirate haunts. The former is arguably so, whereas the latter is sort of expected if piracy led to logging. We may build a similar \ref{eqn:map} without saying so though.}
		In which case, we map \emph{all} such haunts in or about the Bay, and \emph{all} pertinent relations among them.%
		\footnote{And so map all logging stands too. Strictly though, the historical record \emph{possibly} overlooks minor haunts (infrequent, visited only by few, so on), and \emph{likely} does so for logging stands (minor, far-flung, so on). The reconstructed natural habitat of dyewoods might help a bit with the latter.}
	
		\subsubsection{Places}
		For haunts outside of the Bay, we note that the only mode of transport we need here is maritime, and the only maritime entrances to the Bay are via the north or east.%
		\footnote{For simplicity, counting north-east entrance via the Caymans as via east.}
		So, we might simply map these as \(\nt\) and \(\et\) in \ref{eqn:map}, respectively, and count these as proxies for all such haunts.
	
		For haunts inside the Bay, we have no such proxies, so we check this by hand, and find we may want to add to Bacalar and the Bay Islands only one further place, namely, the mouth of the Bay.%
		\footnote{We assume this hand check was thorough enough. For the Bay bound from Catoche to Gracias a Dios (rather than Ascension or Espiritu Santo to Camaron), the only other places we might possibly add are Valladolid (from Ascension to Catoche including islands) to the north or north Mosquito Shore (from Camaron to Gracias including cayes) to the east, but these showed up too infrequently. The Bay Islands seem to widely be recognised as a pirate haunt in literaure (and works on piracy), with Bacalar less so (and infrequently in works on piracy), and the mouth of the Bay almost never (even in works on piracy). If we identify haunts empirically (from piratical activity), then the mouth seems to have been more active than Bacalar, such that it ought to be deemed one if the latter is. If we instead identify haunts strategically, then the mouth seems a likelier haunt than Bacalar and even the Bay Islands. For strategic identification, say pirates frequented places from which they could readily spy on shipping ports, and say this was especially so for major rather than minor ports. Then all the Bay's major ports (Puerto Caballos, Santo Tomas, Golfo Dulce) were in its mouth, so none could be readily spied on from Bacalar nor the Bay Islands, whereas they all could from the mouth itself (including cayes). Piratical activity is plotted in Figure~\ref{fig:piracyinbay} (black for in mouth of Bay), as gleaned from literature (for all years) and the Guatemala and Mexico series in AGI (to 1669 inclusive). For this figure, note the review process very likely missed some literature and records, and further note incidents do not seem consistently carved out, and further include some defence incidents, and ones dubiously deemed piratical. Figure~\ref{fig:piracyinbay} data in \href{https://github.com/aenavidad/pp-0/tree/main/incidents}{\nolinkurl{/incidents}}, with a tally and partial calendar of defence records in \href{https://github.com/aenavidad/pp-0/tree/main/cartas}{\nolinkurl{/cartas}} (not plotted).} % tally/calendar of reviewed (def and non-def) records
		So, we map these as \(\bc\), \(\id\), and \(\mt\) in \ref{eqn:map}, respectively.%
		\footnote{So \ref{eqn:map} has all pirate haunts (and so all logging stands) in or about the Bay.}
		% fig pirate activity in bay
		% fig piracy 
\begin{filecontents}{ppfigpirdata.dat}
% incidents from /incidents in Bacalar only (bacl) vs mouth of Bay (mouth) vs Bay Isl only (isl) vs elsewhere/unknown in Bay (bay) - make sure bay value *excludes* nobz value to stack bars in plot
yr	bay	bacl	mouth	isl
1601	2	0	0	0
1602	0	0	0	0
1603	0	0	1	1
1604	0	0	2	0
1605	0	0	0	0
1606	0	0	4	0
1607	0	0	4	1
1608	0	0	0	0
1609	0	0	0	0
1610	0	0	1	0
1611	0	0	0	0
1612	0	0	0	0
1613	0	0	0	1
1614	0	0	0	0
1615	0	0	0	0
1616	0	0	0	0
1617	0	1	0	0
1618	0	0	1	0
1619	0	0	0	0
1620	0	0	0	0
1621	0	0	0	0
1622	0	0	0	0
1623	0	0	0	0
1624	0	0	0	0
1625	0	0	0	0
1626	0	0	0	0
1627	0	0	0	0
1628	0	0	0	0
1629	0	0	0	0
1630	1	0	0	1
1631	0	0	1	0
1632	0	0	0	1
1633	1	0	0	1
1634	1	0	0	0
1635	0	0	0	0
1636	1	0	0	2
1637	0	0	0	2
1638	2	1	1	2
1639	0	0	1	4
1640	0	0	3	3
1641	2	1	0	1
1642	1	2	2	2
1643	0	0	1	3
1644	0	0	2	2
1645	0	0	0	1
1646	0	0	0	1
1647	0	0	0	1
1648	0	1	0	1
1649	0	0	0	0
1650	0	0	1	2
1651	0	0	1	1
1652	0	2	0	0
1653	0	0	0	0
1654	0	1	1	0
1655	0	0	0	0
1656	1	0	0	0
1657	0	0	0	0
1658	0	0	0	0
1659	1	0	0	0
1660	1	0	0	1
1661	0	0	0	0
1662	0	0	0	0
1663	0	0	0	0
1664	0	0	0	0
1665	0	0	1	2
1666	0	0	1	0
1667	1	0	2	0
1668	0	0	0	0
1669	0	0	0	0
1670	1	0	0	0
1671	0	0	0	0
1672	0	0	0	1
1673	0	0	0	0
1674	0	0	0	0
1675	0	0	0	0
1676	0	0	0	1
1677	0	1	0	0
1678	0	0	1	1
1679	0	1	1	0
1680	0	2	0	0
1681	0	0	0	0
1682	0	2	0	0
1683	3	0	1	1
1684	0	0	1	0
1685	1	0	1	0
1686	1	0	1	1
1687	1	0	2	0
1688	1	0	2	1
1689	0	0	0	0
1690	0	0	1	0
1691	0	0	0	0
1692	0	0	0	0
1693	0	0	0	0
1694	0	1	0	0
1695	0	0	0	0
1696	0	0	0	0
1697	0	0	0	0
1698	0	0	0	0
1699	1	0	0	0
1700	0	0	0	0
\end{filecontents}
\begin{figure}
% pgsplots code begins
\begin{sideways}
\begin{tikzpicture}
	\begin{axis}[
		xmin=1601, xmax=1700,
		xtick={1610,1620,1630,1640,1650,1660,1670,1680,1690}, % to not display all yrs
		xticklabel style={/pgf/number format/1000 sep=}, % to not add comma
		enlarge x limits=0.02, % to not cut off y bars
		ymin=0, ymax=8,
		ybar stacked, % use y bars and stack
		bar width=0.6, % for thinner bars
		width=1\textheight,height=1\textwidth, % expand sideways fig
		]
	\addplot [
		black,fill=black,% colour outline and fill
		] table [x=yr,y=mouth] {ppfigpirdata.dat};
	\addplot [
		lightgray,fill=lightgray,% colour outline and fill
		] table [x=yr,y=bay] {ppfigpirdata.dat};
	\addplot [
		lightgray,fill=lightgray,% colour outline and fill
		] table [x=yr,y=bacl] {ppfigpirdata.dat};
	\addplot [
		lightgray,fill=lightgray,% colour outline and fill
		] table [x=yr,y=isl] {ppfigpirdata.dat};
\end{axis}
\end{tikzpicture}
\end{sideways}
% pgsplots code ends
\caption{Piracy in the Bay in the 17th century.}
\label{fig:piracyinbay}
\end{figure}
%
 % \label{fig:piracyinbay}
	
		\subsubsection{Relations}
		For relevant relations among haunts, we seem to only need cost (of maritime transport) from place to place.%
		\footnote{Possibly missing minor relations. Cost naturally subsumes distance, but additionally expense, labour, know-how, so on. No non-cost relations seem immediately relevant.}
		So say some place is \ment{close} to another if it is \emph{not} costly to get from one to the other and vice versa.%
		\footnote{Appropriately qualified, eg for a given cost measure, for a given affordable-costly threshold, so on. Though cost is naturally directed and weighted, we make closeness undirected and unweighted for simplicity, and further ignore self-closeness for the same.}
		Then we say \(\bc\), \(\mt\), and \(\id\) are all close to each other, while \(\nt\) and \(\et\) are close to none, and add to \ref{eqn:map}.%
		\footnote{So \ref{eqn:map} has all pertinent relations among mapped places.} % move Bay Triangle stuff here ?
	%
	%
	% subsec temporal models
	\subsection{Temporal}
	\label{ss:temporal}
		We seem to need a few temporal models, namely, \ref{eqn:logs}.%
		\footnote{For space restricted to in and about the Bay, per \ref{eqn:map}. For vertices as points in time, and arcs as series of happenings or events through time (allowing loops for \(t_i\leq t_j\) where \(t_i=t_j\)). Labelled arcs represent a certain series of events, while unlabelled ones represent any such series. \ref{eqn:logs} are a bit trickier than \ref{eqn:map}.}
		% eqn logs - may number logs
		\begin{equation}
		\tag{Logs}
		\label{eqn:logs}
		\begin{tikzcd}[row sep=tiny]
		t_0 \rar &t_{\og} \rar &t_1\\
		t_0 \rar &t_{\buc} \rar &t_{\ct} \rar &t_1\\
		t_0 \rar &t_{\ct} \rar &t_{\buc} \rar &t_1\\
		t_i \rar \ar[rr,bend right,"a_k"] &t_{\ct} \rar &t_j
		\end{tikzcd}
		\end{equation}
		
		We start by listing temporal points (and their relations) commonly represented in literature.%
		\footnote{For points (other than \(t_{\og}\)), the start and end of piracy (loose or tight bounds, \ment{piracy} variously understood, only or not only in the Bay, so on), start of logging (loose or tight bounds, \ment{logging} variously understood, only or not only in the Bay, so on), and start or end of various phenomena (or appearance or disappearance of various states of affairs) bearing on piracy or logging (mostly tight bounds, mostly not only in Bay, so on), including social, legal, commercial, or military pressures for or against piracy or logging. For relations (other than \ment{lead to}), \ment{encourage to} and \ment{discourage from}. For this hand check, we disregard theories which do not represent piracy as leading to logging. This class includes most normative theories, and so includes most Wallacian ones. Temporal models for these would seem to be pretty straightforward in any case, eg just a point in time. Further though, we agree with \cite{cam11}*{pp~95--96} (and most recent literature, it would seem) that normative theories seem to be the least viable ones available. The reason for this stance is sometimes given as historians' having found no record of such an act despite having looked for it since at least the 19th century. But the act may not have been recorded in the first place, or if it was, the native record would have almost \emph{surely} been lost or destroyed in the Meridian War, so this does not strictly seem like a particularly good reason for the stance. On the other hand, we note that the act, if it took place, would have \emph{possibly} entered oral tradition. And we \emph{do} have a few 17th and 18th century sources where mention of such an act would have been crucial or at least pertinent, even with reference to tradition, and where we nonetheless find no such notice. \emph{This} absence seems like a stronger reason for the stance, and is the one we give (though strictly this only establishes that the act \emph{possibly} did not take place, unless we say the act \emph{more than} possibly would have entered tradition).} % may add gloss of tight vs loose bounds - spell out how recent lit seems to disallow normative founding ?
		And note that these seem to be bounds of piracy or logging or related phenomena or states of affairs near or about \(t_{\og}\).%
		\footnote{Or near or about the bounds of piracy or logging.}
		In which case, we plot \emph{only} the bounds of piracy and logging, and \emph{only} the \ment{lead to} relation.%
		\footnote{Plus \(t_{\og}\), plus a further relation needed for \ment{lead to}. We do not plot related phenomena nor states of affairs (nor their \ment{encourage to} nor \ment{discourage from} relations) for simplicity, and because current evidence in literature does not really give us a clear picture of them nor their relations, such that their relevance is also not very clear. For instance, it is possible for commercial pressures to have had much more of a bearing on piracy or logging than legal or military ones, and vice versa, but we have not really come across solid evidence for nor against either possibility (despite the latter one seemingly being more often than not presumed in literature, judging by how often legal or military pressures are cited, as compared to commercial ones). Further, almost all of these would seem to more naturally fit a graded (as phenomena) rather than binary (as states of affairs) representation (despite the latter being more common in literature). That said, we might still end up with similar (but messier) \ref{eqn:logs} if we plotted them.}
		
		\subsubsection{Bounds}
		For outer bounds, we fix the lower bound as that of the earlier phenomenon (piracy), and upper bound as that of the later one (logging), and plot in \ref{eqn:logs} as  \(t_0\) and \(t_1\), respectively.%
		\footnote{For appropriate phenomenal bounds, eg loose lower and tight upper ones, say 1520s to 1720s for piracy and 1590s to 1770s for logging. If we imagine piratical and logging activity as bell curves, then we seem to need the left tail ends wholly contained in our models, but not so much the right tail ends.}
		Then for all existing and at least some possible \(t_{\og}\), we ought to and do have \(t_0 \leq t_{\og} \leq t_1\).%
		\footnote{Existing \(t_{\og}\) range from 1550s to 1710s. Possible \(t_{\og}\) range further to 1780s (if polity requires permanent ie uninterrupted settlement), 1800s or 1810s (if polity requires multiple settlements), so on. So this restricts possible carvings of (and so possible theories for) the problem.}
		
		For inner bounds, we need only the upper bound of piracy, and lower bound of logging. But as these are and may be variably understood, let \ment{buccaneering} and \ment{cutting} be subclasses of piracy and logging, respectively.%
		\footnote{For either strict or non-strict subclasses, eg piracy by only certain vs any agents, logging of only logwood vs any dyewoods, so on.}
		Then we fix the upper bound of buccaneering and lower bound of cutting \emph{in} the Bay, and plot in \ref{eqn:logs} as \(t_{\buc}\) and \(t_{\ct}\), respectively.%
		\footnote{For either loose or tight bounds. We restrict bounds to the Bay for simplicity, and feel this is licensed as all existing and most possible \(t_{\og}\) likewise mark an event or happening in the same. This disallows a spatially discontinuous polity though, so further restricts possible carvings and theories. Note a good number of existing theories consider \emph{both} loose and tight lower bounds for cutting (ie light vs heavy or sporadic vs frequent cutting).}
		
		\subsubsection{Paths} % no need to mention/consider literature/piracy/logging
		For the relation, say some phenomenon \ment{finely} leads to another if most of the latter's earlier incidents are incidents of the former too.%
		\footnote{Appropriately qualified, eg for temporally tight but spatially loose incident bounds, for some early-late threshold, so on. We mean to count as single incidents cases where buccaneers first set out to plunder in one place but then end up cutting in another (say because they found no ships to surprise in the first), so on. There are ways of getting at this concept which do not rely so much on careful carving of incidents, eg in terms of closeness and coincidence, so on.}
		And say they \ment{tightly} track each other in some given period of time if their incidence curves therein are similar and very temporally close.%
		\footnote{Appropriately qualified, eg for incidence plotted only across time, for naive similarity (curves roughly having the same shape), for some temporal closeness threshold (few months or years), so on. Further, say they \ment{loosely} track each other in some period if their curves therein are similar but not temporally close.}
		And lastly distinguish \ment{gainful} from \ment{lossy} buccaneering.%
		\footnote{For some appropriate profitable-unprofitable theshold, eg as determined by agents vs empirically, gross vs net, so on. We need this to split buccaneering into disjoint subphenomena. If we set a minimal empirical threshold (eg \ment{gainful} if buccaneers got at least a penny's worth of booty, else \ment{lossy}), then we note that the historical record almost exclusively offers a picture of gainful buccaneering only (strictly speaking), this being especially so for buccaneering after the appearance or rise of resident buccaneers in the western Caribbean, and for buccaneering far from or not tied to a formal colony (Jamaica, Providence), and so for buccaneering in the Bay too (for which we almost completely rely on Spanish sources). Loosely speaking, we would have a proxy picture of lossy buccaneering for some given period if it tightly tracked gainful buccaneerring therein (straightforwardly) and if they loosely tracked (given their temporal order and distance). For Spanish records, we note their tendency to label any foreigner \ment{pirate} (inflating gainful buccaneering incidence), and the likely or possible chance of their missing certain gainful buccaneering incidents (deflating incidence) eg later or minor ones (by victims' being desensitised), ones during heightened buccaneering activity (being overwhelmed), so on, where the inflation is easy to correct, but the deflation not so much.} % may note no picture of loss bucc esp after appearance of resident pirates in Caribb, or buccaneering away/not near/not from Jam, etc
		Then for buccaneering and cutting \emph{in} the Bay, lossy buccaneering either did or did not finely lead to cutting, and either did or did not tightly track gainful buccaneering near or about \(t_{\ct}\).%
		\footnote{We say \ment{earlier} incidents in \ment{finely leading to} cutting are near or about \(t_{\ct}\), and so restrict all paths to the Bay as \(t_{\ct}\) was so restricted. For buccaneering and cutting in or out of the Bay, both gainful buccaneering's finely leading to cutting and its not tightly nor loosely tracking lossy buccaneering (in some appropriate period) do not seem viable. A notable class of theories which do not represent lossy buccaneering in the Bay as finely leading to cutting in the Bay are those where only too few of the early cutting incidents were buccaneering ones too, or where most were but only outside the Bay. Note the unviability claims do not strictly hold, but seem likely if buccaneers were rational agents (such that their subjective measures of profit mirrored our empirical ones) and if lossy buccaneering was multimodally distributed (as gainful buccaneering in Figure~\ref{fig:piracyinbay}).} % may not need to note order of peaks for tight/loose tracking - may note gain bucc is multimodal distrib and we imagine loss bucc is so too - may note reasons for unviability claims
		So we fix four paths near or about \(t_{\ct}\), and plot in \ref{eqn:logs} as \(a_k\). % may add a_fl for finely lead (yes/no) and a_tt for tightly track (yes/no)
%
%
%
%
% sec theories
\section{Theories} % no need to mention/consider piracy/logging
\label{s:theories}
	In this section, we review theories (existing ones to Section~\ref{ss:restall}, and possible ones from Section~\ref{ss:possypolities}) to see which carvings of the problem and models of the record they hold for.%
	\footnote{Literature haphazardly chosen from Table~\ref{tab:literature}. Qualify all claims regarding literature as being \emph{for} a possible or at least slightly acceptable (not an intended nor accurate nor most acceptable) reading. Possible theories are presented in bit and pieces, also haphazardly chosen. These seem to be at least as viable as existing theories, but may not be exhaustive.} % may note no ordered viability claim - may note does not exhaust viable theories - may add unviable possible theories
	%
	%
	% subsec existing reads
	\subsection{\lit{Read's}}
	\label{ss:reads}
		Honduras was a stand- and settlement-state in Bacalar founded by 1648 or by 1652 when the English settled for cutting.%
		\footnote{In \cite{rds32}*{p~3}. The unsigned letter, dated Campeachy, 24 Nov 1731, reads in part:\begin{quote}As to the State of the Bay of Honduras, I shall give it you as briefly as possible. The ancient City of Bacalar, situate in that Part of the Province of Yucatan, which lies on the Bay of Honduras, was twice sack'd, and at last totally ruined by the English many Years ago; on which the Logwood-Cutters of that Nation, who had settled on the River of Valis, possessed themselves of the New River and that of the Hondo; which last is distant from the Ruins of Bacalar about five Leagues. Here they built a great many Houses and Hutts, and employ'd Multitudes of Negroes in cutting Logwood, which was transported to Jamaica and Europe by Numbers of Vessels trading from thence to the Bay.\end{quote} \ment{Bacalar} does not seem to name Bacalar-in-Pacha nor -in-Chunhuhub, so we read \ment{twice sack'd} and \ment{at last totally ruined} as referring to the Blauvelt sackings of Bacalar. Further, this theory may be abridged, but we read it as if it were not. This is one of the earliest theories in print (at least), and one of a few to claim later Bacalar sackings were not merely piratical (possibly). \cite{res19}*{pp~17, 20} notes a similar theory by Pitt in 1734 (that cutting stands on Old `had been possessed by the English for more than a hundred years,' in AGI Mexico 3099 ff~5--10), but deems it `a politically motivated rhetorical exaggeration.' We agree that this Bayman and Pitt were likely politically biased, but can see carvings for which their theories hold (whereas they clearly do not for Restall's).} % may note bul16 pp~151-152 say 1642 Mulato Bacl sack may have been more than piratical (to force Sp to withdraw) - may note Blauvelt sacking seem more brutal than earlier ones ?
		% re theory = none more
		For this theory to hold, we need a carving a bit more realised than that in Section~\ref{ss:carving}, most obviously, one disallowing at least non-state polities and certain largeness or continuity requirements of state.%
		\footnote{This Bayman seems to say a state is founded if some territory is settled or possessed. They may further have meant to require that a state be demographically or economically large or significant in some sense (cf last sentence of extract), but we have a number of mid- and late 17th century sources which make no mention of such presence in mainland Bacalar, even when it would have been pertinent or when they do mention maritime presence. So we read this Bayman as \emph{not} requiring these, and say this portion of their letter refers to post-1710s Honduras (or else is political self-aggrandisement). Alternatively though, they may have meant to allow certain senses of discontinuity.}
		We likewise need models which pare down both \ref{eqn:map} and \ref{eqn:logs}, at least so as to restrict Honduras to Bacalar and ignore buccaneering.%
		\footnote{This latter looks like political obfuscation and so ought to be ignored. The former may not have been meant by this Bayman, but would be needed for what seems to be their intended carving, as we have a number of records which suggest buccaneers settled or possessed the Bay Islands prior to or circa mid 17th century.} % may add Et entrance seems more likely than Nt prior to mid 17th cent
	%
	%
	% subsec existing fenix
	\subsection{\lit{Fénix}}
	\label{ss:fenix}
		Honduras was a haunt- and settlement-state in Bacalar founded in the mid 17th century when Wallace settled for buccaneering.%
		\footnote{In \cite{fen49}*{p~3}. Honduras founded when south-east Yucatan was no longer inhabited by the Spanish ie after the Maya uprising in or Blauvelt raids of Bacalar, and founded before its territory was ceded to Wallace by the Mosquito ruler via treaty. Sierra O'Reilly uses \ment{Wallace} properly, but we read the name symbolically, and ignore biographical details. The Maya uprising here may be the Tipu Rebellion, and the Mosquito \emph{ruler} is properly the Mosquito \emph{King}. We note that the Rebellion is sometimes not properly named in literature, though it seems like it ought to be (not discussed here). Further, its (proper and improper) names are sometimes used to refer to a multi-year event involving acts by Maya residents (torch-and-flight, so on) \emph{and} by Spanish ones (\forn{entradas}, \forn{visitas}, so on), but the latter are strictly responses \emph{to} the former (ie the Maya acts themselves suffice to constitute a rebellion), so this does not seem like it ought to be so (not discussed here).} % may note 'Tipu Reb' sometimes names multi-year rebellion or incl Sp responses to rebellion - for Tip Reb, note Sp acts are non-rebellious, so only Maya acts are rebellious
		% re theory = none more
		This theory would hold for carvings and models similar to \lit{Read's}, except that now cutting is ignored.%
		\footnote{As with \lit{Read's} though, this last again seems political, and so is ignored. Sierra O'Reilly may have required that a state be demographically large (Wallace's 80 men) and continuous (stationary, permanent settlement on Old), and that its founding be with respect to both its collective (settlement or occupation of Old) and corporate aspects (naming or cession of Old). After ignoring biographical details though, we may read them as not requiring largeness nor continuity, or requiring them only with respect to corporate aspects, or requiring them over all of Bacalar, so on. Further, Sierra O'Reilly may have meant to fix a temporal model (Spanish bearing on buccaneers' presence).}
		% re carving ↓
		% state emerges if place is occupied, or if a perpetual den or lair is established or built or named, or if den or lair's territory is legally ceded via treaty
		% must further allow involuntary temporal discontinuity at least, given 18th cent evacs
		%
		% re temporal ↓
		% bucc = non-Sp res pirates
		% pressures = mil by Sp vs bucc/cutt (to 1638 or 1648 or 1652)
	%
	%
	% subsec existing campbell
	\subsection{Campbell}
	\label{ss:campbell}
		Honduras was a haunt- and stand- and settlement-polity in Belize which emerged in the 1550s and gained statehood by the mid 17th century (likely) or by the 1670s (definitely) as the British went from visiting to frequenting to squatting to settling for buccaneering or cutting.%
		\footnote{In \cite{cam03}*{pp~171--179}. Honduras located first in cayes [p~174] then in riverine mainland too [p~178]. Honduras was only a corsair haunt at first, then mostly a buccaneering haunt, then a buccaneering haunt cum cutting stand, then mostly a cutting stand at last, with British presence from first to last, and settlement from second or third to last. The corsair haunt first emerged in Elizabethan period when corsairs began to appear [pp~171--172] ie when British corsairs began to use cayes as hiding places probably as early as mid 1550s [pp~176, 179]. Cutting began at some uncertain date, and intensified by mid 17th century when tide turned dramatically against buccaneering ie when buccaneers were forced into more sedentary activities [p~171]. Settlement began or expanded when cutting intensified [p~171], and was expansive firmly by 1670s when Sharp `reigned' as `Chief' at St George's ie before Delgado journey [pp~176--178]. Campbell uses \ment{corsairs} to name any pirates [pp~171--172, 176], and \ment{buccaneers} for the particular sort of pirate which began to take shape in the Caribbean from around 1620s [pp~171--173, 176, 199]. This theory is further worked out in \cite{cam11}*{to cap~4}, where Campbell notes they mean to say that the state emerged gradually, bit by bit, over an extended period of time \cite{cam11}*{pp~95--96}. In which case, we read Campbell as saying haunts and stands were quasi-states which piecemeal became states long after emerging (ie statehood gradually gained or crystallised), though they may not have meant this exactly. Lastly (and to be fussy), Campbell and some literature misname Belizean cayes (eg St George's \emph{Cay}), but this ought to be avoided regardless of English variant or style guide (we do not say the Florida \emph{Cayes} in Belizean English).} % may note loose naming generally (ie name vs name cf Notes) - may note is one of few minimal theories in lit
		% re theory fm cam03 ↓
		% by 17th cent buccaneers infested most cayes [p~174]
		% haunt expanded to most of Belize's cayes [p~174]
		% Brit settled after buccaneers succeeded corsairs [pp~171--173] ie after Brit buccaneers came to dominate the haunt [pp~174, 176]
		% settlement began at St George's [pp~171--172, 174, 176]
		% Brit presence extended to riverine mainland by 1670s ie before Delgado journey [p~178]
		% British occupation encouraged by 1695 or 1696 on failure of Ursua entradas [pp~178-179]
		%
		% re theory fm cam11 ↓
		% cutting occurs by Eliz period or by early 17th or before Jam invasion [pp~106--107]
		% cutting intensifies by early 1670s [p~108] or by 1680s [p~104] or by 1708 [p~91--92]
		This theory would hold for the Section~\ref{ss:carving} carving as is, and even for full \ref{eqn:map} and \ref{eqn:logs} with respect to \(t_{\og}\), except that both models need to be pruned with respect to the point when statehood was gained.%
		\footnote{Campbell seems to say a non-state polity emerges if some territory is used or loose or disorganised communities form in it, and gains statehood if said territory is settled or said communities grow non-loose or organised. Campbell meant to fix particular spatial and temporal models, and the point when statehood was gained (but not \(t_{\og}\)) would seem to vary if we deviate from these. Spatially, we need to isolate Belize from the Bay Islands, else we may get earlier definite statehood (possibly 1641 per \cite{cam11}*{p~151}). Temporally, we need certain specific legal pressures (in \citelist{\cite{cam03}*{pp~171--172} \cite{cam11}*{pp~102--104}}) to have so strongly discouraged buccaneering so as to essentially have rendered other pressures moot, else we may get earlier likely or definite statehood. And so on.}
		% re carving ↓
		% state = collective [cam03]
		% quasi-state emerges if used as haven [cam03 171, 176] or if settlement is attempted [cam11 vii] or if agents entrench themselves well [cam11 95]
		% proper-state emerges if sedentary activities occur [cam03 171], or if a settlement is established [cam03 174] or a principal settlement [cam03 175 /cam11 96] or a permanent one [cam03 176] or one with a leader [cam03 176]
		%
		% re models ↓ fm cam03
		% 'corsairs' = incl Eliz sea dogs [pp~171--172, 176] = ie Sp-deemed pirates [p~176]
		% 'buccaneers' = later succeeded corsairs [pp~171-172], came long after corsairs et learnt to cure meat ie boucan fm Caribs et were of vars Euro countries [p~173] - name 'buccaneers' began to take shape in region fm around 1620s [p~176] - first haunt emerged in Tortuga before 1640s [fn 1 p 199]
		% cutt = of logwood [pp~171--172]
		% pressures = dramatic tide vs bucc [p~171] - legal by GB vs buccan [p~172]
		%
		% re temporal ↓
		% cutt = of any dyewood by bucc [cam11 pp~104--105]
		% pressures = social/legal by Jam vs bucc for cutt by 1680s [cam11 102-104]
	%
	%
	% subsec existing bulmer-thomases
	\subsection{Bulmer-Thomases}
	\label{ss:thomases}
		Honduras was a haunt- and stand- and settlement-state in Belize founded in 1642--1669 (possibly) or by the 1670s (likely) or by 1677 (definitely) when the British settled for buccaneering or cutting.%
		\footnote{In \cite{bul16}*{pp~137--138, 145, 147, 151--156}. Honduras located only in cayes if founded pre-1670 [p~151, 154--155], else in coastal mainland too [pp~155--156]. Honduras founded after capture of Providence [p~145], and only after Mulato sacking of Bacalar [pp~151--152], and \emph{before} establishment of logwood cutting as main economic activity [pp~137--138], and before Delgado journey [pp~147, 151]. Cutting picked up in favour of buccaneering by most settlers after 1670 [pp~154, 156], and intensified likely in 1670s on arrival of Catoche cutters [p~156], and occured surely by 1680 [p~154].}
		This theory would hold for carvings similar to \lit{Read's}.%
		\footnote{The Bulmer-Thomases seem to say a state is founded if some territory is settled.} % note their model is clearer to see than their carving - note cannot use nonstrict succession
		% re carving ↓
		% state emerges if a more than transitory settlement is established [p~137], or if agents make some place their home [p~154], or if permanently settled [pp~147, 151]
		And we would need to cut \ref{eqn:map} and \ref{eqn:logs} in ways similar to Campbell.%
		\footnote{Spatially, again remove the Bay Islands, else we may get \(t_{\og}\) in 1638 per [pp~152--153, 155]. Temporally, again render all but a few specific military and legal pressures bearing on buccaneering or cutting moot, else we may get earlier possible and likely \(t_{\og}\). Similarly, for 1642--1669 in the Bay, we need gainful and lossy buccaneering to have tightly tracked, and a demographically tight \(t_{\ct}\) (lots of cutters), else we may get this \(t_{\og}\) as the likely one. And so on.} % do we get these new t_og for sec 2.1 carving or bul-thom intended carving ?
		% re temporal ↓
		% bucc = by non-Sp res of Caribb - cutt = of logwood by bucc
		% pressures = mil by Sp vs bucc/cutt (to 1642) + mil by Maya vs cutt (to when there were enough cutters ie 1670s) + legal by Jam vs bucc for cutt (in/post 1670s)
		% bounds = tight t_buc (lots of bucc) - tight t_ct (lots of cutters) - for t_buc close to t_ct in 1670s
		% finely led to = no
	%
	%
	% subsec existing restall
	\subsection{Restall}
	\label{ss:restall}
		Honduras was a stand- and settlement-state in Bacalar founded in 1716 when Campeachy men settled for cutting.%
		\footnote{In \cite{res19}*{pp~6--8, 16--17, 21, 24--25, 34}. Honduras located on and about Old [pp~12--13]. Honduras founded definitely in 1716 when regular cutting began [pp~16--17], and definitely not in 17th century [pp~7-8, 16, 21, 24--25, 34]. Cutting occurs possibly from 1662 in small occasional incursions [pp~16, 21], and is regularised definitely in 1716 [p~17]. This is one of the few theories in literature building a model from historical trend data (cartographic patterns, in this case) rather than discrete data points.} % may note etymology/toponomy/archaeological trends also explicit in some lit - note not immediately sure that cartog data relevant for non-Restallian carvings
		This theory would hold for a sharper carving than that in Section~\ref{ss:carving}, most immediately admitting only states and requiring largeness or continuity in some sense.%
		\footnote{Restall seems to say a state is founded if it is settled. They further seem to (and need to) require some largeness or continuity or both with respect to the collective aspect of state.}
		% state founded if permanently settled
		And it would hold for \ref{eqn:map} as is, though \ref{eqn:logs} would seemingly need to be pruned.%
		\footnote{Restall may have meant to split Honduras and the Bay Islands, but we would seemingly not need to for their carving. Temporally, if their carving required both largeness and continuity of state, then full \ref{eqn:logs} would seem to work. Else, we may need to prune them to not model a large discontinuous state or small continuous one before their \(t_{\og}\).} % may double check - may expand on required pruning
	%
	%
	% subsec possible polity senses
	\subsection{\ment{Polity} variants}
	\label{ss:possypolities}
		Most literature seems to regard only states, and some only large or continuous ones.%
		\footnote{The latter mostly with respect to the collective aspect of state.} % and sometimes judged relative to Map space rather than Bay only ?
		This does not straightforwardly seem like it ought to be so, as preceding quasi-states, and small and discontinuous states and quasi-states are possible.%
		\footnote{Nor have we come across explicit support for nor a defence of this carving.}
		Further, if the first Baymen were a maritime or mobile people, then there is some sense in which these carvings are unfitting.%
		\footnote{For buccaneers mostly. Further if they were a corporate people, or if tight \(t_{\ct}\) preceded tight \(t_{\buc}\).}
	%
	%
	% subsec possible emergence senses
	\subsection{\ment{Emergence} variants}
	\label{ss:possyemergences}
		Most literature seems to regard only intentional emergence (founding), and some only of the collective aspect (settlement). Again, this does not immediately seem necessary nor fitting.%
		\footnote{For buccaneers nor cutters, and further if they were a corporate people. We are likewise missing explicit support for or a defence of this carving.} % may note also Campbellian 'emergence' unfitting in a sense (not sharp enough as admits Eliz sea dogs who might have lacked proper attitude/behaviours for quasi-state, despite Campbell's claim they formed loose/disorganised communities)
	%
	%
	% subsec possible spatial models
	\subsection{\ref{eqn:map} variants}
	\label{ss:possymaps}
		Most literature (naturally) restricts Honduras to the territory of modern Belize, but it is nonetheless not obvious that this was nor ought to be so. Most notably, if the first Baymen regarded the inner Bay as one single or whole territory (possibly composed of three contiguous parts as in \ref{eqn:map}), then we might rather want to locate Honduras in the inner Bay.% yes if we defer to their carving (polity, emergence), but also yes even if we do not (some sense in which we have to take into acct their sense of space) ?
		\footnote{If we were to defer their carving (naturally), but even so if we did not (though not so straightforwardly), as regardless of carving, there is a sense in which our spatial models ought to mirror their understanding of physical space. The former, for instance, might be done as follows. Say they derived their sense of ownership from use, and say they first used the Bay Islands as a haunt at some time \(t_i\), and only later used the mouth of the Bay and Bacalar as haunts or stands at \(t_k\). Then if they already felt like they owned the Islands at some intermediate \(t_j\), later use of the mouth and Bacalar might count to them as \emph{expansion} of their original territory, such that they might claim first ownership of \emph{all} the inner Bay from \(t_i\) or \(t_j\), despite only gaining two-thirds of it from \(t_k\). (This seems analogous to saying `America was founded in 1776,' despite a vast majority of the country not existing at the time, and despite the War not being won and the Union not being constituted but for several years later.) Further, say any of these places fell into disuse from \(t_m\) for some extended time. If buccaneers came back at some later \(t_n\) to find no one using said place, then they might consider this simply a \emph{continuation} of their original use and ownership from \(t_i\) or \(t_j\), rather than a \emph{renewal} of use or ownership. (This seems analogous to fixing \(t_{\og}\) before 1783, as strictly, Baymen's presence in Honduras was uninterrupted \emph{only} since this date.) Finally, say it was the \emph{original} place (the Islands here) which fell into disuse from \(t_m\), this time \emph{permanently}. Then the Baymen might nonetheless claim ownership from \(t_i\) or \(t_j\), especially if they already viewed themselves as constituting a polity. (Analogously, if the Thirteen Colonies' successor states seceded from America tomorrow, it would still be the case that said country was founded in 1776.) We note that the political expediency of making or recognising such claims does not seem to wholly nor significantly bear on their truth or reasonableness, and that official recognition (eg by Providence) might have only further encouraged or fostered self-identification as a polity. We imagine the inner Bay may have functioned as an early \ment{Bay Triangle} eg if tight \(t_{\ct}\) preceded Jamaican invasion. The \ment{Bay Triangle} meaning Honduras, Jamaica, and the Mosquito Shore is named in \citelist{\cite{cam11}*{p 130} \cite{cam03}*{p 186}}, but was first identified by \cite{nay89}*{pp 46--53} for 18th century commerce, it seems. On deference to popular carvings or senses despite political bias, we do \emph{not} mean that this ought to be done, rather only that it may and already seems to be done. For instance, we say America was founded in 1776, despite the Declaration's being blatantly political, and say Belize gained self-governance in 1964 and independence in 1981, despite the choice of \emph{these} rather than earlier points of self-governance (eg from emergence, from unofficial majority) or independence (eg from Jamaica, from Spain, from the Mosquito Shore even) being arguably political. We would ideally have a principled way of deciding on a carving, but other than extreme strictness or laxity of carving, or else deference to popular senses, we do not see one (except that it ought to viable or fitting for all pirate or logwood settlements, it would seem).} % last Hond evac in 1779 - may add principled way of deciding on a carving, or more this part up
	%
	%
	% subsec possible temporal models
	\subsection{\ref{eqn:logs} variants}
	\label{ss:possylogs}
		Some literature dates the start of heavy cutting to right after the end of heavy buccaneering, with this order claimed for buccaneers in the western Caribbean. That this was so \emph{in} the Bay is not immediately clear.%
		\footnote{That for distinct tight bounds in the Bay, \(t_{\buc} \leq t_{\ct}\) rather than \(t_{\ct} \leq t_{\buc}\), possibly over extended time.}
		Further, significance of cutting and buccaneering seem to be judged relative to buccaneers in the western Caribbean, though this comparison class does not seem necessary.%
		\footnote{Honduras or the Bay seem possible too.} % and more fitting ?
		Additionally, this seems to often be claimed due to certain strong legal or military pressures against buccaneering, despite possibly strong or stronger demographic or commercial pressures for cutting.%
		\footnote{Including at least the population or density of resident buccaneers, or of coastal or insular Spanish or Maya or native residents, coastal or maritime availability of marketable logwood cut by the Spanish, or of Spanish merchandise generally, English or Dutch demand for logwood, and legal and illegal Spanish or English or Dutch maritime trade.} % add maritime smuggling, logwood legalisation, Jam letpasses ?
		Lastly, for happenings in the Bay, some literature may be committed to lossy buccaneering's not finely leading to cutting, or to its tightly tracking gainful buccaneering in some period of interest, even though alternative paths have not been ruled out. % difficult to say for sure as not so clear in lit ? would want this clear ?
%
%
%
%
% sec concl
\section{Conclusion}
\label{s:concl}
	In sum, there are quite a few ways of understanding the problem of the origins of Belize, and of representing the relevant historical record, giving us a myriad existing (and possible) theories, most of which seem more or less as viable as the rest.%
	\footnote{Given current evidence, and despite some apparent (possibly not substantive) claims to the contrary in literature. Further, this seems to hold generally for any pirate or logwood settlement, not uniquely for Belize (likewise qualified).}
	
	So, if this problem is vague, we would ideally wish for theories which provide explicit reasons for their carving (or against other viable carvings).%
	\footnote{Ideally also an explicitly helpful or principled (not implicit nor haphazard) way of fixing a carving.}
	Further, if its vagueness is inherent, much would seem to depend on this. Else, its vagueness is artificial, in which case we might further wish for theories with strong or granular evidence for their models (or against other viable models).
%
%
%
%
% end stuff
%
% s references
\begin{bibdiv}
\label{s:references}
	\begin{biblist}
	\bibselect{pprefs}
	\end{biblist}
\end{bibdiv}
%
%
%
\end{document}