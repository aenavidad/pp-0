% top stuff
\documentclass{amsart}
%
%
% packages
\usepackage{pgfplots} % for plots - loads tikz, wc itself loads xcolor, graphicx
\usepackage{tikz-cd} % for graphs - loads tikz too
\usepackage[figuresright]{rotating} % to rotate table w LHS at bottom
\usepackage{hyperref} % for links - load last-ish
\usepackage{amsrefs} % for full cites - use author-year option? - load last
%
%
% new commands to avoid text formatting inline
\newcommand{\code}{\texttt} % for code
\newcommand{\lit}{\textit} % for book titles etc
\newcommand{\ship}{\textit} % for ship names etc
\newcommand{\ment}{\textit} % to mention (not use) terms
\newcommand{\forn}{\textit} % for foreign words
%
%
% new commands to avoid maths formatting inline
\DeclareMathOperator{\ab}{Ab} % place variable
\DeclareMathOperator{\cd}{Cd} % ^
\DeclareMathOperator{\id}{Id} % place name
\DeclareMathOperator{\mt}{Mt} % ^
\DeclareMathOperator{\bc}{Bc} % ^
\DeclareMathOperator{\et}{Et} % ^
\DeclareMathOperator{\nt}{Nt} % ^
\DeclareMathOperator{\pre}{pre} % time period name
\DeclareMathOperator{\pst}{pst} % ^
\DeclareMathOperator{\cpd}{cpd} % ^
%
%
% amsthm styles where (i) mainclm > minrclm = gloss > note = bodytext > footnote, or (ii) clm >= gloss > etc
% plain style
\newtheorem{mainclm}{Main Claim} % for main claims
\newtheorem{minrclm}{Minor Claim} % for minor claims
\newtheorem{clm}{Claim} % for claims
% definition style
\theoremstyle{definition}
\newtheorem{gloss}{Gloss} % for glosses
% remark style
\theoremstyle{remark}
\newtheorem*{note}{Note} % for notes
%
%
% amsthm style for proof
\newenvironment{steps}{\begin{proof}[Steps]}{\end{proof}} % for steps
\renewcommand{\qedsymbol}{\textit{End}.} % for all proof env
%
%
% tikz ie pgsplots, tikzcd commands
\pgfplotsset{compat=1.18}
\usepgfplotslibrary{groupplots} % to group plots
\pgfmathdeclarefunction{gauss}{2}{\pgfmathparse{1/(#2*sqrt(2*pi))*exp(-((x-#1)^2)/(2*#2^2))}} % to use in \addplot
%\tikzcdset{} % global graphs style
%
%
% xcolor commands for darkmode - just for draft
\pagecolor{darkgray}
\color{white}
%
%
% hyperref commands
\hypersetup{pdfauthor={Angel E Navidad},pdftitle={On the Origins of Belize},pdfsubject={},pdfkeywords={}} % pdf metadata
%
%
\begin{document}
%
%
% starting stuff
\title{On the Origins of Belize}
\author{A Navidad}
\address{Harvard College, Cambridge, MA, US}
\curraddr{Benque Viejo, Cayo, Belize}
\email{navidad@college.harvard.edu}
\date{12 Feb 2025}
\thanks{} % incl non-grant support per ams
\begin{abstract}
\end{abstract}
\keywords{}
\maketitle
%
%
%
%
% sec intro
\section{Introduction}
\label{s:intro}
	The origins of Belize have been an open problem in historical literature since at least the 18th century, and remain so today, with over two dozen theories proposed so far, and little consensus to boot.%
	\footnote{Belize is the only pirate settlement in the Caribbean, and only logwood settlement, to have survived past the 19th century. As such, it is the only country in the Americas for which this problem is open. \emph{All} pirate or logwood settlements seem to have unclear origins though, including at least the Bay Islands, Campeachy, Catoche, the Mosquito Shore, and Tortuga. The preceding uniqueness claim has previously been made in literature for the same or smaller comparison classes. The generality claim has prior been made for at least some pirate or logwood settlements, but seems to hold for all of them. Theories for Belize listed in Table~\ref{tab:literature}. Note this table is not a systematic literature review, and further note we claim only that given readings are possible or acceptable. All data for this paper deposited at \url{https://github.com/aenavidad/pp-0/}. In particular, Table~\ref{tab:literature} data in \href{https://github.com/aenavidad/pp-0/tree/main/literature}{\nolinkurl{/literature}}.}
	% tab literature
	% tab literature
\begin{table} % sidewaystable is no go in amsart apparantly
\caption{Stories* in 18th to 21st cent literature.}
\label{tab:literature}
% missg long (s596) + s1644 rejection
% tab code begins
\begin{sideways}
\begin{tabular}{cp{.4\textheight}lp{.3\textheight}cc}
No	&Date	&In	&Claim	&Yes	&No\\
(1)	&mid 16th cent	&Old	&Eng seek haven	&2	&0\\ % campbell - locn:St George's - yes (s2316, s2320) - no ()
(2)	&ca Sp Hon flotilla est	&Old	&Wallace settles	&	&\\ % asturias - first in live wiki
(3)	&ca 1610	&Old&Wallace settles	&&\\ % egli
(4)	&pre/ca Fuensalida missions	&N	&Brit log	&&\\ % roys
(5)	&in 1617	&--	&Wallace present&&\\ % asturias
(6)	&pre penult Bacl sacking	&Old	&Eng log	&&\\ % reads 1732
(7)	&ca Prov est	&S	&Provers seek refuge or farm	&&\\ % winzerling
(8)	&mid 17th cent	&--	&Wallace settles	&--	&1\\ % fenix 1849 - yes () - no (s1644)
(9)	&pre treaty	&?	&Brit present	&&\\ % mcculloch
(10)	&mid third 17th cent	&--	&Wallace present	&&\\ % ancona
(11)	&in 1638	&--	&Brit wreck	&&\\ % HA 1829
(12)	&in 1638	&--	&Wallace wrecks	&--	&1\\ % HA 1827 - yes () - no (s1644)
(13)	&in 1640	&--	&Wallace wrecks	&&\\ % ungewitter
(14)	&in Cromwell govt	&--	&Eng seek haven	&--	&1\\ % HA 1826 - yes () - no (s1644)
(15)	&post Bacl/Trux sacking or post Camp logging	&--	&Brit seek haven or log	&&\\ % b-thomas
(16)	&last half 17th cent or pre early 1680s	&--	&Eng present	&&\\ % finamore
(17)	&post Jam invasion or post Camp logging	&--	&Jamers log	&&\\ % robertson
(18)	&ca 1663	&--	&Wallace logs	&&\\ % g diaz
(19)	&in 1662	 or pre 1670&--	&Wallace logs	&--	&1\\ % c quijano - yes () - no (s2316)
(20)	&post Catoche/Camp logging	&--	&Jamers log	&&\\ % ancona
(21)	&post treaty	&--	&Wallace present	&&\\ % n ortega
(22)	&last third 17th cent	&--	&Eng present	&&\\ % ancona
(23)	&early 18th cent	&--	&Eng seek haven	&&\\ % brockhaus
(24)	&post Camp logging	&--	&Wallace seeks haven or logs	&&\\ % ancona - last in live wiki
(25)	&post Camp logging	&Old	&Eng log	&1	&0 % restall - yes (s1644) - no ()
\end{tabular}
\end{sideways}
% tab code ends
\end{table}
%
 % \label{tab:literature} - may swap in 'From' (\cite values) for 'Source' col - may remove self-citation from 'Cit' tallies
	
	This paper seeks to either complicate or clarify matters by presenting existing and possible understandings of the problem in Section~\ref{s:problem}, representations of the historical record in Section~\ref{s:models}, and theories in Section~\ref{s:theories}.%
	\footnote{For appropriately qualified (not strict) \ment{existing} or \ment{possible} understandings or representations or theories, eg historically-relevant, non-trivial, seemingly viable, so on. In this paper, we naively imagine or sketch historical reasoning as first taking a problem, then carving or refining it into well-specified questions, then representing or modelling the historical record, and finally using this model to answer the posed questions, where such answers hold \emph{for} the questions as so specified and \emph{for} the model as so built. We claim only that this is a possible or acceptable naive sketch, not that it is uniquely so, nor that this is \emph{actually} how historians have previously tackled the problem of the origins of Belize. We think the naive sketch might help to get a grip of the problem, make sense of current discordance in literature, or see and fill in gaps therein. This paper uses the name \ment{Belize} anachronistically, and does \emph{not} regard the related but distinct problem of the \emph{name's} origins. The problem at hand seems to have been first noted by \cite{alm26}*{p~5} (implicitly), first discussed by \cite{nun77}*{pp~6--9, 12--14} (obliquely), and most recently discussed by \citelist{\cite{bul16} \cite{cam09} \cite{res19}}.} % may add note re weak vs strong empircal claims made for problem/models/theories
%
%
%
% sec problem
\section{Problem}
\label{s:problem}
	In this section, we characterise the problem in Section~\ref{ss:carving}, as it is currently \emph{and} as it may possibly be understood, and note some salient features in Section~\ref{ss:features}.%
	\footnote{The carving given here is not fully realised, but seems straightforward and useful enough as is. No uniqueness (`this is the \emph{only} carving available') claim made. A weak empirical claim \emph{is} made though, ie that the given carving \emph{is} how at least some historians understand the problem, modulo our possible or acceptable readings of literature. No strong empirical (`this is \emph{actually} how most historians have previously carved') claims made.}
	%
	%
	% subsec problem carving
	\subsection{Carving}
	\label{ss:carving}
		Let \ment{Honduras} be the earliest predecessor state of modern Belize, and say it first emerged or came to be at some time \(t_{\og}\).\footnote{\ment{Bay} also appropriate here.} Then Belize did \emph{not} exist at any time before \(t_{\og}\), and \emph{did} exist at least at some times after \(t_{\og}\).\footnote{Allowing for state discontinuity.} In which case, we say the problem of the origins of Belize is a class of questions pertinent to this happening or process.%
		\footnote{Appropriately qualified, eg class is non-empty and finite, questions are direct or non-embedded and immediately or straightforwardly pertinent, so on. We imagine the pertinent questions are at least `What is Honduras?,' `Where is Honduras?,' `When is \(t_{\og}\)?,' and `How did Honduras come to be at \(t_{\og}\)?' Call these the \ment{What}, \ment{Where}, \ment{When}, \ment{How} questions of the problem.}
		And a theory for this problem is a class of answers to those questions.%
		\footnote{Appropriately qualified, eg class has an answer for each required question, answers are immediate or direct, so on.}
	%
	%
	% subsec problem features
	\subsection{Features}
	\label{ss:features}
		We note this carving requires at least concepts of \ment{state} and their emergence and continuity and succession, but does not specify them. To that extent, it is barely realised, and so remains open to a number of distinct senses or specifications of these concepts.%
		\footnote{We mainly have in mind here existing or possibly useful senses. These are at least the spatiotemporal vs collective vs corporate state, small vs large state, intentional vs accidental emergence of state, continuous vs discontinuous state, and voluntary vs involuntary discontinuity of state.}
	
		Further, let a \ment{vague} problem be one where, given a \emph{complete} historical record, at least some of the particular senses of the required concepts give us discordant theories.%
		\footnote{Appropriately qualified, eg record is relevant, discordance is significant, so on. The idea being that the historical record itself might not straightforwardly close the problem for us. Examples of non-vague (closed) problems are those of the origins of Virginia and St Kitt's \& Nevis. Regardless of what we mean by \ment{state}, \ment{state emergence}, so on, theories for those problems invariably agree on the relevant happenings and date them to 1607 and 1623, respectively. In contrast, theories for a vague problem would \emph{not} so invariably agree (unless trivially).}
		Then we say the problem at hand is vague.%
		\footnote{Further, say the problem's vagueness is \ment{artificial} if the non-extant historical record would dissolve the vagueness, and say it is \ment{inherent} if it would not. Then we say the problem might even be \emph{inherently} vague. We note the problem seems to be vague and possibly inherently so for all pirate or logwood settlements. The preceding vagueness claim has previously been made in literature for at least some of these polities.}
		Further, let a \ment{minimal} theory be one which does \emph{not} presuppose or fix particular senses of the required concepts.\footnote{Appropriately qualified, eg has no trivial answers, so on.} Then we say we currently have very few minimal theories for the problem at hand.%
		\footnote{These might be desireable, given the current glut of maximal theories in literature, ie of theories which \emph{do} presuppose or fix particular senses of the required concepts. Arguably though, the historian's job \emph{is} to provide such maximal theories and say `these are the senses we ought to use for this problem.' In which case, we might at least want \emph{transparent} theories, eg ones which flag the senses in use, give us reasons to accept their proffered senses, to reject competing ones, or at least which note competing senses exist. Then we say we do not seem to have very many transparent theories in literature either.}
%
%
%
% sec models
\section{Models}
\label{s:models}
	In this section, we represent or model the relevant, extant historical record, as it is currently \emph{and} as it may possibly be modelled, spatially in Section~\ref{ss:spatial}, and temporally in Section~\ref{ss:temporal}.%
	\footnote{No uniqueness claim made. A weak empirical claim \emph{is} made, ie that the given models \emph{are} how at least some current theories in literature represent the historical record (modulo our possible or acceptable readings of literature). We mainly have in mind theories which seem viable, but do not mean to discard any from the outset. Wallacian theories pose a difficulty here, given their popularity (cf Table~\ref{tab:literature}) and at least partial unfoundedness. \citelist{\cite{cam09}*{pp~72--110} \cite{bul16}*{pp~137--151} \cite{res19}*{pp~19--24}} have thankfully shown Wallace is most likely apocryphal. We do \emph{not} disregard such theories on this basis, however (though we \emph{do} end up disregarding most of them later on, for a distinct but related reason), as we imagine the following might be the case for at least some of them. Say some historian carefully reviewed the relevant archives, and concluded from this that there was heightened piratical activity in the Bay of Honduras in the mid-third of the 17th century, followed by increasingly significant logging activity in the last third of the same. And say they then came across some work claiming Wallace landed and stayed on the Old River in 1638. Then they might just give us a Wallacian theory, but the Wallace portion therein would \emph{not} be necessary, such that if the aforementioned work later proved unfounded, we might simply excise it from the theory, eg by ignoring biographical details, and reading \ment{Wallace} symbolically ie as \ment{one of the first Baymen}. We think this ought to be done, eg for Meridian theories, including \citelist{\cite{fen49}*{p~3} \cite{car71}*{pp~55, 210} \cite{car78}*{pp~260--261} \cite{anc78}*{pp~370--376} \cite{pen69}*{pp~217--219}}, modulo the non-zero chance of unique, non-extant primary sources in Merida in and after the 18th century. This is to say, we do \emph{not} think Wallace's being apocryphal (by itself) significantly pares down the number of viable theories in literature, nor restricts the number of possible ones (eg as in \cite{bul16}*{pp~137--138, 145, 151--156}). Further, as charges of dishonesty in Wallacian theories by \citelist{\cite{bul16}*{pp~138, 140--141} \cite{cam09}*{pp~87--88, 105--106} \cite{res19}*{pp~23--24}} regard \ment{Wallace} properly (not symbolically) read, they are likewise disregarded here (though we agree that almost all of them are ideologically biased).}
	
	We start by noting that the most common consensus available in literature is that piracy somehow lead to logging, such that nearly all existing theories invoke piracy or logging or both, which is to say they represent at least one or both these happenings or phenomena, deeming them relevant parts of the record.%
	\footnote{Appropriately qualified eg invoke in some significant way ie deem significantly relevant, for \ment{piracy} by resident or non-resident Dutch or English or Frenchmen, for \ment{logging} of any dyewood or of only logwood, so on. Most theories seem to invoke both. A notable class which sometimes invoke only one or neither are \ment{normative} theories where agents simply land and immediately settle and found a state in some given year. Incidentally (or unsurprisingly), most Wallacian theories seem to be normative ones. We owe this class to xx, who credits xx.}
	So our models must represent \emph{both} of these.%
	\footnote{Appropriately qualified, eg must represent completely or thoroughly or appropriately, up to similarity, so on.}
	%
	%
	% subsec spatial models
	\subsection{Spatial}
	\label{ss:spatial}
		We seem to need only one spatial model, namely, \ref{eqn:map}.%
		\footnote{The Bay and Atlantic surroundings (up to some 800--900 nautical miles from centroid) mapped from northwest to southeast. For vertices as places, and edges as a certain relation among them. }
		% eqn map
		\begin{equation}
		\tag{Map}
		\label{eqn:map}
		\begin{tikzcd}[arrows=dash]
		\nt &\bc \rar \ar[rr,bend left] &\mt \rar &\id &\et
		\end{tikzcd}
		\end{equation}
	
		We start by listing places referenced in literature.%
		\footnote{Namely, Bacalar ie north-central Belize and the Bay Islands in the Bay, plus Campeachy (including Terminos and Tris), Catoche (including Cozumel and possibly Ascension), Jamaica, the Mosquito Shore (south-central), Old Providence, and Tortuga (including north-west Hispaniola) outside of it. Possibly missing infrequent places.}
		And note that these seem to be places in or about the Bay of Honduras which pirates are thought to have frequented in the 17th century.%
		\footnote{Or that plus formal English colonies, or plus logging stands. Possibly also frequented in 16th or 18th centuries.}
		In which case, we map \emph{all} such haunts in or about the Bay, and \emph{all} pertinent relations among them.%
		\footnote{Pirate haunts seem to subsume logging stands, so mapping all the former ought to give us all the latter.}
	
		For haunts outside of the Bay, we note that the only mode of transport we need here is maritime, and the only maritime entrances to the Bay are via the north or east.\footnote{Counting north-east entrance via the Caymans as via east.} So, we might simply map these as \(\nt\) and \(\et\) in \ref{eqn:map}, respectively, and count these as proxies for all such haunts.
	
		For haunts inside the Bay, we have no such proxies, so we check this by hand, and find we may want to add to Bacalar and the Bay Islands only one further place, namely, the mouth of the Bay.%
		\footnote{We assume this hand check was thorough enough. For the Bay bound from Catoche to Gracias a Dios (rather than Ascension or Espiritu Santo to Camaron), the only other places we might possibly add are Valladolid (from Ascension to Catoche including islands) to the north or north Mosquito Shore (from Camaron to Gracias including cayes) to the east, but these showed up too infrequently. The Bay Islands seem to widely be recognised as a pirate haunt in literaure (and works on piracy), with Bacalar less so (and infrequently in works on piracy), and the mouth of the Bay almost never (even in works on piracy). If we identify haunts empirically (from piratical activity), then the mouth seems to have been more active than Bacalar, such that it ought to be deemed one if the latter is. If we instead identify haunts strategically, then the mouth seems a likelier haunt than Bacalar and even the Bay Islands. For strategic identification, say pirates frequented places from which they could readily spy on shipping ports, and say this was especially so for major rather than minor ports. Then all the Bay's major ports (Puerto Caballos, Santo Tomas, Golfo Dulce) were in its mouth, so none could be readily spied on from Bacalar nor the Bay Islands, whereas they all could from the mouth itself (including cayes). Piratical activity is plotted in Figure~\ref{fig:piracyinbay}, as gleaned from literature (for all years) and the Guatemala and Mexico series in the Archivo General de Indias (to 1669 inclusive). For this figure, note the review process very likely missed some literature and records, and further note incidents do not seem consistently carved out, and further include some defence incidents, and ones dubiously deemed piratical. Figure~\ref{fig:piracyinbay} data in \href{https://github.com/aenavidad/pp-0/tree/main/incidents}{\nolinkurl{/incidents}}, with a tally and partial calendar of defence related records in \href{https://github.com/aenavidad/pp-0/tree/main/cartas}{\nolinkurl{/cartas}} (not plotted).}
		So, we map these as \(\bc\), \(\id\), and \(\mt\) in \ref{eqn:map}, respectively.%
		\footnote{So \ref{eqn:map} has \emph{all} pirate haunts (and so all logging stands) in or about the Bay.}
		% fig pirate activity in bay
		% fig piracy 
\begin{filecontents}{ppfigpirdata.dat}
% incidents from /incidents in Bacalar only (bacl) vs mouth of Bay (mouth) vs Bay Isl only (isl) vs elsewhere/unknown in Bay (bay) - make sure bay value *excludes* nobz value to stack bars in plot
yr	bay	bacl	mouth	isl
1601	2	0	0	0
1602	0	0	0	0
1603	0	0	1	1
1604	0	0	2	0
1605	0	0	0	0
1606	0	0	4	0
1607	0	0	4	1
1608	0	0	0	0
1609	0	0	0	0
1610	0	0	1	0
1611	0	0	0	0
1612	0	0	0	0
1613	0	0	0	1
1614	0	0	0	0
1615	0	0	0	0
1616	0	0	0	0
1617	0	1	0	0
1618	0	0	1	0
1619	0	0	0	0
1620	0	0	0	0
1621	0	0	0	0
1622	0	0	0	0
1623	0	0	0	0
1624	0	0	0	0
1625	0	0	0	0
1626	0	0	0	0
1627	0	0	0	0
1628	0	0	0	0
1629	0	0	0	0
1630	1	0	0	1
1631	0	0	1	0
1632	0	0	0	1
1633	1	0	0	1
1634	1	0	0	0
1635	0	0	0	0
1636	1	0	0	2
1637	0	0	0	2
1638	2	1	1	2
1639	0	0	1	4
1640	0	0	3	3
1641	2	1	0	1
1642	1	2	2	2
1643	0	0	1	3
1644	0	0	2	2
1645	0	0	0	1
1646	0	0	0	1
1647	0	0	0	1
1648	0	1	0	1
1649	0	0	0	0
1650	0	0	1	2
1651	0	0	1	1
1652	0	2	0	0
1653	0	0	0	0
1654	0	1	1	0
1655	0	0	0	0
1656	1	0	0	0
1657	0	0	0	0
1658	0	0	0	0
1659	1	0	0	0
1660	1	0	0	1
1661	0	0	0	0
1662	0	0	0	0
1663	0	0	0	0
1664	0	0	0	0
1665	0	0	1	2
1666	0	0	1	0
1667	1	0	2	0
1668	0	0	0	0
1669	0	0	0	0
1670	1	0	0	0
1671	0	0	0	0
1672	0	0	0	1
1673	0	0	0	0
1674	0	0	0	0
1675	0	0	0	0
1676	0	0	0	1
1677	0	1	0	0
1678	0	0	1	1
1679	0	1	1	0
1680	0	2	0	0
1681	0	0	0	0
1682	0	2	0	0
1683	3	0	1	1
1684	0	0	1	0
1685	1	0	1	0
1686	1	0	1	1
1687	1	0	2	0
1688	1	0	2	1
1689	0	0	0	0
1690	0	0	1	0
1691	0	0	0	0
1692	0	0	0	0
1693	0	0	0	0
1694	0	1	0	0
1695	0	0	0	0
1696	0	0	0	0
1697	0	0	0	0
1698	0	0	0	0
1699	1	0	0	0
1700	0	0	0	0
\end{filecontents}
\begin{figure}
% pgsplots code begins
\begin{sideways}
\begin{tikzpicture}
	\begin{axis}[
		xmin=1601, xmax=1700,
		xtick={1610,1620,1630,1640,1650,1660,1670,1680,1690}, % to not display all yrs
		xticklabel style={/pgf/number format/1000 sep=}, % to not add comma
		enlarge x limits=0.02, % to not cut off y bars
		ymin=0, ymax=8,
		ybar stacked, % use y bars and stack
		bar width=0.6, % for thinner bars
		width=1\textheight,height=1\textwidth, % expand sideways fig
		]
	\addplot [
		black,fill=black,% colour outline and fill
		] table [x=yr,y=mouth] {ppfigpirdata.dat};
	\addplot [
		lightgray,fill=lightgray,% colour outline and fill
		] table [x=yr,y=bay] {ppfigpirdata.dat};
	\addplot [
		lightgray,fill=lightgray,% colour outline and fill
		] table [x=yr,y=bacl] {ppfigpirdata.dat};
	\addplot [
		lightgray,fill=lightgray,% colour outline and fill
		] table [x=yr,y=isl] {ppfigpirdata.dat};
\end{axis}
\end{tikzpicture}
\end{sideways}
% pgsplots code ends
\caption{Piracy in the Bay in the 17th century.}
\label{fig:piracyinbay}
\end{figure}
%
 % \label{fig:piracyinbay} - may split bars into Mt/Bc/Id/rest-of-Bay colours
	
		For relevant relations among haunts, we seem to only need cost (of maritime transport) from place to place.%
		\footnote{Possibly missing minor relations. Cost naturally subsumes distance, but additionally expense, labour, know-how, so on. No non-cost relations seem immediately relevant.}
		So say some place is \ment{close} to another if it is \emph{not} costly to get from one to the other and vice versa.%
		\footnote{Appropriately qualified, eg for a given cost measure, for a given affordable-costly threshold, so on. Though cost is naturally directed and weighted, we make closeness undirected and unweighted for simplicity, and further ignore self-closeness for the same.}
		Then we say \(\bc\), \(\mt\), and \(\id\) are all close to each other, while \(\nt\) and \(\et\) are close to none, and add to \ref{eqn:map}.%
		\footnote{So \ref{eqn:map} has \emph{all} pertinent relations among mapped places.}
	%
	%
	% subsec temporal models
	\subsection{Temporal}
	\label{ss:temporal}
		We seem to need a few temporal models, namely, (1) to xx of \ref{eqn:logs}.%
		\footnote{For space restricted to in and about the Bay, per \ref{eqn:map}. For vertices as points in time, and arcs as series of happenings or events through time. \ref{eqn:logs} are a bit trickier than \ref{eqn:map}.} % may note arcs may be null for 'series through 0 time'
		% eqn logs
		\begin{equation}
		\tag{Logs}
		\label{eqn:logs}
		\begin{tikzcd}[row sep=tiny]
		(1) &t_0 \rar &t_1\\
		(2) &t_0 \rar &t_{\og} \rar &t_1
		\end{tikzcd}
		\end{equation}
	
		We start by discarding theories which disregard the common consensus (that piracy somehow lead to logging), and so are left with only ones which invoke at least piracy or logging or both.%
		\footnote{Invoke in some significant way ie deem significantly relevant. This class includes most normative theories, and so includes most Wallacian ones. Temporal models for these would seem to be pretty straightforward in any case, eg just a point in time. Further though, we agree with xx (and most recent literature, it would seem) that normative theories seem to be the least viable ones available. The reason for this stance is sometimes given as historians' having found no record of such an act despite having looked for it since at least the 19th century. But the act may not have been recorded in the first place, or if it was, the native record would have almost \emph{surely} been lost or destroyed, so this does not strictly seem like a particularly good reason for the stance. On the other hand, we note that the act, if it took place, would have \emph{possibly} entered oral tradition. And we \emph{do} have a few 17th and 18th century sources where mention of such an act would have been crucial or at least pertinent, even with reference to tradition, and where we nonetheless find no such notice. \emph{This} absence seems like a stronger reason for the stance, and is the one we give (though strictly this only establishes that the act \emph{possibly} did not take place, unless we say the act \emph{more than} possibly would have entered tradition).}
		So, our models must encompass the upper and lower bounds of both phenomena. So we set their lower bound as that of the earlier phenomenon (piracy), and upper bound as that of the later one (logging), and add to \ref{eqn:logs} as  \(t_0\) and \(t_1\), respectively.%
		\footnote{For appropriate phenomenal bounds, eg 1520s to 1720s for piracy, and 1590s to 1770s for logging.  Though piracy seems to \emph{spatially} subsume logging (cf Section~\ref{ss:spatial}), it does not seem to do so \emph{temporally}, nor would we expect it to, if one lead to the other.}
		
		
		% EDIT BELOW --
		Then, for all existing and at least some possible \(t_{\og}\), we ought to have \(t_0\leq t_{\og}\leq t_1\), and we do.% may say 'plus all existing t_og, plus select possible t_og - and so restrict possible space
		\footnote{Existing \(t_{\og}\) points range from 1550s to 1710s. Possible \(t_{\og}\) points range further to 1780s or 1800s or 1860s (for continuous collective state or large collective state or large corporate state, respectively, ie uninterrupted settlement or multiple settlements or multiple levels of governance).}
		
		And set our lower bound as that of the earlier phenomenon, and upper bound as that of the later one, and add to xx as \(t_{\lb}\) and \(t_{\ub}\), respectively.%
		\footnote{For loose bounds, eg 1520s for piracy and mid 19th century for logging, such that we have \(t_{\lb}\leq t_{\og}\leq t_{\ub}\) for any existing or possible \(t_{\og}\).} % of just set arbitrary bounds ed 1500, 1900 that contain t_og plus start/end of phenomena in W Car/G Mex (appropriately bound)
%
%
%
% sec theories
\section{Theories}
\label{s:theories}
	In this section, we xx
	%
	%
	% subsec existing theories
	\subsection{Existing}
	\label{ss:existing}
		xx
	%
	%
	% subsec possible theories
	\subsection{Possible}
	\label{ss:possible}
		xx
%
%
%
% sec concl
\section{Conclusion}
\label{s:concl}
	xx
%
%
%
% end stuff
%
% s references
\begin{bibdiv}
\label{s:references}
	\begin{biblist}
	\bibselect{pprefs}
	\end{biblist}
\end{bibdiv}
%
%
%
\end{document}