% top stuff
\documentclass{amsart}
%
%
% packages
\usepackage{pgfplots} % for plots - loads tikz, wc itself loads xcolor, graphicx
\usepackage{tikz-cd} % for graphs - loads tikz too
\usepackage[figuresright]{rotating} % to rotate table w LHS at bottom
\usepackage{hyperref} % for links - load last-ish
\usepackage{amsrefs} % for full cites - use author-year option? - load last
%
%
% new commands to avoid text formatting inline
\newcommand{\code}{\texttt} % for code
\newcommand{\lit}{\textit} % for book titles etc
\newcommand{\ship}{\textit} % for ship names etc
\newcommand{\ment}{\textit} % to mention (not use) terms
\newcommand{\forn}{\textit} % for foreign words
%
%
% new commands to avoid maths formatting inline
\DeclareMathOperator{\ab}{Ab} % place variable
\DeclareMathOperator{\cd}{Cd} % ^
\DeclareMathOperator{\id}{Id} % place name
\DeclareMathOperator{\mt}{Mt} % ^
\DeclareMathOperator{\bc}{Bc} % ^
\DeclareMathOperator{\et}{Et} % ^
\DeclareMathOperator{\nt}{Nt} % ^
\DeclareMathOperator{\pre}{pre} % time period name
\DeclareMathOperator{\pst}{pst} % ^
\DeclareMathOperator{\cpd}{cpd} % ^
%
%
% amsthm styles where mainclm > minrclm = gloss > note = bodytext > footnote
% plain style
\newtheorem{mainclm}{Main Claim} % for main claims
\newtheorem{minrclm}{Minor Claim} % for minor claims
% definition style
\theoremstyle{definition}
\newtheorem{gloss}{Gloss} % for glosses
% remark style
\theoremstyle{remark}
\newtheorem*{note}{Note} % for notes
%
%
% amsthm style for proof
\newenvironment{steps}{\begin{proof}[Steps]}{\end{proof}} % for steps
\renewcommand{\qedsymbol}{\textit{End}.} % for all proof env
%
%
% tikz ie pgsplots, tikzcd commands
\pgfplotsset{compat=1.18}
\usepgfplotslibrary{groupplots} % to group plots
\pgfmathdeclarefunction{gauss}{2}{\pgfmathparse{1/(#2*sqrt(2*pi))*exp(-((x-#1)^2)/(2*#2^2))}} % to use in \addplot
%\tikzcdset{} % global graphs style
%
%
% xcolor commands for darkmode - just for draft
%\pagecolor{darkgray}
%\color{white}
%
%
% hyperref commands
\hypersetup{pdfauthor={Angel E Navidad},pdftitle={On the Origins of Belize},pdfsubject={},pdfkeywords={}} % pdf metadata
%
%
\begin{document}
%
%
% starting stuff
\title{On the Origins of Belize}
\author{A Navidad}
\address{Harvard College, Cambridge, MA, US}
\curraddr{Benque Viejo, Cayo, Belize}
\email{navidad@college.harvard.edu}
\date{12 Feb 2025}
\thanks{} % incl non-grant support per ams
\begin{abstract}
\end{abstract}
\keywords{}
\maketitle
%
%
%
%
% sec intro
\section{Introduction}
\label{s:intro}
	The origins of Belize have been an open problem in historical literature since at least the 18th century, and remain so today, with over two dozen theories proposed to date, and little consensus to boot.%
	%
	\footnote{In Table~\ref{tab:literature}. Belize is the only pirate settlement in the Caribbean, and only logwood settlement, to have survived past the 19th century. As such, it is the only country in the Commonwealth for which this problem is open. As a rule though, \emph{all} such settlements seem to have unclear origins, including at least the Bay Islands, Campeachy, Catoche, the Mosquito Shore, and Tortuga. We further note the uncertainty seems baked in, such that even if we had a \emph{complete} historical record of activity, we might nonetheless still have uncertain origins. The preceding uniqueness claim has previously been articulated for smaller comparison classes, but seems to apply for all the Commonwealth. Similarly for the preceding vagueness claim. For Table~\ref{tab:literature}, note this is not a systematic literature review, and further note we claim only that given readings are possible or acceptable. All data for this paper deposited at \url{https://github.com/aenavidad/pp-0/}.}
	%
	% tab literature
\begin{table} % sidewaystable is no go in amsart apparantly
\caption{Stories* in 18th to 21st cent literature.}
\label{tab:literature}
% missg story: s596 / missg cites: s1644,s2320,s2316,rushton2014, prolly vars others
% tab code begins
\begin{sideways}
\begin{tabular}{cp{.35\textheight}lp{.2\textheight}ccc}
No	&Date	&In	&Claim	&Pub&First	&Cites\\
(1)	&mid 16th cent	&Old	&Eng seek haven	&2003	&s2320	&2\\ % campbell - locn:St George's - \href{https://github.com/aenavidad/pp-0/blob/main/literature/s2320.json}{s2320}
(2)	&ca Hond flotilla est	&Old	&Wallace settles	&1925	&s770	&7\\ % asturias - first in live wiki
(3)	&ca 1610	&Old&Wallace settles	&1872	&s2252	&2\\ % egli
(4)	&pre/ca Fuensalida missions	&N	&Brit log	&1957	&s396	&3\\ % roys
(5)	&in 1617	&Old	&Wallace settles	&1925	&s770	&2\\ % asturias
(6)	&pre penult Bacl sacking	&Old	&Eng log	&1732	&s1726	&1\\ % reads 1732
(7)	&ca Prov est	&S &Provers seek refuge or farm	&1946	&s416	&2\\ % winzerling
(8)	&mid 17th cent	&Old	&Wallace settles	&1849	&s1722	&12\\ % fenix 1849
(9)	&pre treaty	&--	&Brit present	&1841	&s2254	&2\\ % mcculloch
(10)	&mid third 17th cent	&Old	&Wallace present	&1878	&s506	&1\\ % ancona
(11)	&in 1638	&Old	&Brit wreck	&1829	&s860	&13\\ % HA 1829
(12)	&in 1638	&Old	&Wallace wrecks	&1827	&s860	&16\\ % HA 1827
(13)	&in 1640	&Old	&Wallace wrecks or settles	&1872	&s2238	&5\\ % ungewitter
(14)	&in Cromwell govt	&--	&Eng seek haven	&1826	&s860	&5\\ % HA 1826
(15)	&post Bacl/Trux sacking or post Camp logging	&Old	&Brit seek haven or log	&2016	&s78	&1\\ % b-thomas
(16)	&last half 17th cent or pre early 1680s	&N	&Eng present	&1994	&s244	&1\\ % finamore
(17)	&post Jam invasion or post Camp logging	&--	&Jamers log	&1777	&s1900	&4\\ % robertson
(18)	&ca 1663	&Old	&Wallace logs	&2010	&s2270	&1\\ % g diaz
(19)	&in 1662	or pre 1670&Old	&Wallace logs	&1944	&s418	&2\\ % c quijano
(20)	&post Catoche/Camp logging	&Old	&Jamers log	&1878	&s506	&2\\ % ancona
(21)	&post treaty	&Old	&Wallace present	&1877	&s510	&1\\ % n ortega
(22)	&last third 17th cent	&Old	&Eng present	&1878	&s506	&1\\ % ancona
(23)	&early 18th cent	&N	&Eng seek haven	&1864	&s1858	&2\\ % brockhaus
(24)	&post Camp logging	&N	&Wallace seeks haven or logs	&1871	&s2256	&4\\ % c ancona - last in live wiki
(25)	&post Camp logging	&Old	&Eng log	&2019	&s1644	&1 % restall
\end{tabular}
\end{sideways}
% tab code ends
\end{table}
%
 % \label{tab:literature}
	
	This paper seeks to either complicate or clarify matters by proposing a way of building such theories in Sections~\ref{s:claim} and \ref{s:support}, then building a number of them in Section~\ref{s:use}, and finally reviewing the exercise in Section~\ref{s:review}.%
	%
	\footnote{We end up \emph{not} recommending the proposed method. It nonetheless looks like a way of building \emph{minimal} theories (ones which presuppose as little as possible and so are maximally open to interpretation) might look something like that proposed in this paper. A minimal theory for this problem seems desireable, given its aforementioned vagueness, and current glut of maximal theories in literature.}
%
%
%
% sec claim
\section{Claim}
\label{s:claim}
	In this section, we present Claim~\ref{clm:maindecency}.
	
	Let \ment{question*} be `What are Belize's origins?' And let \ment{stories*} be answers to question*.\footnote{Existing or possible, subject to some suitability criteria, eg are historical, are not flimsy, so on.}
	
	% main claim
	\begin{clm}
	\label{clm:maindecency}
		\ref{eqn:namedmap} and \ref{eqn:namedlog} are a decent way of building a story*.\footnote{Qualified, eg up to similarity, used jointly, with relevant historical data, so on.}
	\end{clm}
	
	% main eqn map
	\begin{equation}
	\tag{Map}
	\label{eqn:namedmap}
		\begin{tikzcd}[arrows=dash]
		\nt &\bc \rar \ar[rr,bend left] &\mt \rar &\id &\et
		\end{tikzcd}
	\end{equation}
	
	For vertices as certain places in the Bay (of Honduras), and edges as a certain relation between them. The Bay is mapped from northwest to southeast, beginning in northern Yucatan (\(\nt\)), flowing down to northern Belize (\(\bc\)), then the mouth of the Bay (\(\mt\)), next the Bay Islands (\(\id\)), and finally eastern Honduras (\(\et\)). We say a place \(\ab\) is connected to place \(\cd\) (\(\ab-\cd\)) if \(\ab\) is \ment{close} to \(\cd\), and not otherwise.\footnote{Ignoring trivial cases eg where \(\ab=\cd\).} Basically, spaces are close if they are geographically close.
	
	% main eqn log
	\begin{equation}
	\tag{Log}
	\label{eqn:namedlog}
		\begin{tikzcd}[sep=huge]
		t_0 \rar["\pre" description] &t_c \rar["\cpd" description] &t_d \rar[bend left,"\pst_0" description] \rar[bend right,"\pst_1" description] &t_1
		\end{tikzcd}
	\end{equation}
	
	For vertices as certain points in time, and arcs as certain series of events from one point to the other. A story* is logged from start to finish, beginning at the earliest point (\(t_0\)), then a certain \ment{coupling} point (\(t_c\)), next an ensuing \ment{de}coupling point (\(t_d\)), and ending at the latest point in the story* (\(t_1\)). These four points split a story* into three parts, with a coupl\ment{ed} period (\(\cpd\)) in the middle, and a preceding \ment{pre}coupled period (\(\pre\)). The succeeding \ment{post}coupled period (\(\pst\)) is plotted as two series of events (\(\pst_0\) and \(\pst_1\)) rather than one, to account for two possibly distinct sequences of events. Basically, the coupling refers to the coincidence of piracy and logging.	
	% main gloss
	\begin{gloss}
	\label{gloss:decency}
		A way of building a story* is decent if the story* so built does \emph{not} answer question* nor `When was Belize settled?' nor `When was Belize founded?,' and \emph{does} answer at least `When was Belize used for piracy?' or `When was Belize used for logging?'\footnote{Qualified, eg up to similarity, so on. We might want further features here, eg stories* so built are novel, are granular, so on.}
	\end{gloss}
	
	The general idea in Gloss~\ref{gloss:decency} being that, for inherently vague problems, we might want minimal theories which are maximally open to interpretation.
%
%
%
% sec support
\section{Support}
\label{s:support}
	In this section, we first build \ref{eqn:namedmap} and \ref{eqn:namedlog}, then show they are decent.
	%
	% subsec build
	\subsection{Build}
	\label{ss:build}
		We start by noting that almost all existing stories* begin with piracy and end with logging, and noting that there is almost no consensus on the path charted from one to the other, nor on the time it took to traverse said path. Nonetheless, we now have thematic start and end points, call them \ment{first-theme} (piracy) and \ment{last-theme} (logging).
		
		Now we note that, while existing stories* are agreed on first-theme, there is yet no agreement on when nor where to begin, with values for the former ranging over three centuries, and for the latter spanning two landmasses.\footnote{Sixteenth to eighteenth cneturies, and Central America including Yucatan, plus the Greater Antilles.} But despite not having them, call them \ment{first-time} (when story* begins) and \ment{first-place} (where story* begins), and similarly for \ment{last-time} and \ment{last-place}.\footnote{These spatial and temporal points are for stories*, \emph{not} for the start/end of our themes ie phenomena ie piracy and logging.}
		%
		% subsubsec map
		\subsubsection{Map}
		\label{sss:map}
			Now, to get first-places from existing stories*, we either stop and consider each proferred first-place individually, or else sort them into classes and consider each class instead. We do the latter here.
		
			First, sort all first-places located within the Bay into one class, and all others into another.\footnote{Discard null and \ment{Bay} first-places.} We now have the Bay Islands and Belize in the first class, and Campeachy, Catoche, Jamaica, the Mosquito Shore, (Old) Providence, and Tortuga in the other.\footnote{Possibly ignoring minor or overlooked first-places. \ment{Belize} meaning north-central Belize. \ment{Campeachy} including Terminos and Tris. \ment{Catoche} including Cozumel and possibly Ascension. \ment{Tortuga} including north-west Hispaniola. \ment{Mosquito Shore} meaning south-central Shore.}
		
			Before proceeding, we note that these first-places look, of course, like a list of pirate haunts in the greater vicinity of Belize, just as we would expect from first-theme. To have a comprehensive \ref{eqn:namedmap}, then, we would want a \emph{complete} list of pirate haunts in said vicinity.
			
			Now, before checking that we have a complete list of pirate haunts, we here note that maritime entry to the Bay is either through the north or the east, such that for first-places outside the Bay, regardless of their geographic location, the relevant agents would all enter the Bay through one of two points.\footnote{Counting a north-east entrance via the Caymans as an eastern one.} In which case, we might map the north (\(\nt\)) and east (\(\et\)) entrances, and take these as proxies for \emph{all} pirate haunts outside the Bay. So, \(\nt\) and \(\et\) give us a complete list of pirate haunts for the second class.
		
			We have no such shortcut for the first class of first-places (pirate haunts in the Bay) though, so this check is done by hand. This done, we find we might want to add one first-place to this group, namely, the mouth of the Bay (\(\mt\)), giving us three first-places (with the Bay Islands (\(\id\)) and Belize (\(\bc\))). We assume our hand check was thorough, and so conclude that \(\mt\), \(\id\), and \(\bc\) give us a complete list of pirate haunts in this class.%
			%
			\footnote{\(\mt\) is deemed a pirate haven, and thus haunt, due to the greater incidence of piracy there than in \(\bc\). As the latter is deemed a haunt, so ought the former. Strictly, though, we might not need incidence data to identify \(\mt\). For instance, if a \ment{haven} is `a place from which pirates spied on shipping lanes or ports,' and if there was at least one such for any major Spanish port, then \(\mt\) is required, it being the only place from which they might have spied on the major Spanish ports of the Bay. Incidence is plotted in Figure~\ref{fig:piracyinbay}, as gleaned from literature (for all years) and the Guatemala and Mexico series in the Archivo General de Indias (to 1669 inclusive). For Figure~\ref{fig:piracyinbay}, note the review process very likely missed some literature and records, and further note incidents do not seem consistently carved out, and further include some defence incidents, and ones of dubious piracy. The AGI data for Figure~\ref{fig:piracyinbay} further include a tally and partial calendar of defence related records.}
			%
			% fig piracy 
\begin{filecontents}{ppfigpirdata.dat}
% incidents from /incidents in Hond Bay (noall) or Bz only (nobz) - make sure noall value *excludes* nobz value to stack bars in plot
yr	noall	nobz
1601	2	0
1602	0	0
1603	2	0
1604	2	0
1605	0	0
1606	4	0
1607	5	0
1608	0	0
1609	0	0
1610	1	0
1611	0	0
1612	0	0
1613	1	0
1614	0	0
1615	0	0
1616	0	0
1617	0	1
1618	1	0
1619	0	0
1620	0	0
1621	0	0
1622	0	0
1623	0	0
1624	0	0
1625	0	0
1626	0	0
1627	0	0
1628	0	0
1629	0	0
1630	2	0
1631	1	0
1632	1	0
1633	2	0
1634	1	0
1635	0	0
1636	3	0
1637	2	0
1638	5	1
1639	5	0
1640	6	0
1641	2	2
1642	5	2
1643	4	0
1644	4	0
1645	1	0
1646	1	0
1647	1	0
1648	1	1
1649	0	0
1650	3	0
1651	2	0
1652	0	2
1653	0	0
1654	1	1
1655	0	0
1656	1	0
1657	0	0
1658	0	0
1659	1	0
1660	2	0
1661	0	0
1662	0	0
1663	0	0
1664	0	0
1665	3	0
1666	1	0
1667	2	1
1668	0	0
1669	0	0
1670	1	0
1671	0	0
1672	1	0
1673	0	0
1674	0	0
1675	0	0
1676	1	0
1677	0	1
1678	2	0
1679	1	1
1680	0	2
1681	0	0
1682	0	2
1683	5	0
1684	1	0
1685	2	0
1686	3	0
1687	3	0
1688	4	0
1689	0	0
1690	1	0
1691	0	0
1692	0	0
1693	0	0
1694	0	1
1695	0	0
1696	0	0
1697	0	0
1698	0	0
1699	1	0
1700	0	0
\end{filecontents}
\begin{figure}
% pgsplots code begins
\begin{sideways}
\begin{tikzpicture}
	\begin{axis}[
		xmin=1601, xmax=1700,
		xtick={1610,1620,1630,1640,1650,1660,1670,1680,1690}, % to not display all yrs
		xticklabel style={/pgf/number format/1000 sep=}, % to not add comma
		enlarge x limits=0.02, % to not cut off y bars
		ymin=0, ymax=8,
		ybar stacked, % use y bars and stack
		bar width=0.6, % for thinner bars
		width=1\textheight,height=1\textwidth, % expand sideways fig
		]
	\addplot [ % plot for piracy incidents in bz
		black,fill=black,% colour outline and fill
		] table [x=yr,y=nobz] {ppfigpirdata.dat};
	\addplot [ % plot for piracy incidents outside of bz
		gray,fill=gray,% colour outline and fill
		] table [x=yr,y=noall] {ppfigpirdata.dat}; % check noall value *excludes* nobz value to stack
\end{axis}
\end{tikzpicture}
\end{sideways}
% pgsplots code ends
\caption{Piracy incidents in the Bay in the 17th cent (black in Bz).}
\label{fig:piracyinbay}
\end{figure}
%
 % \label{fig:piracyinbay}
			
			So, we now have a list of five first-places which name all pirate haunts in the vicinity of Belize. As \ref{eqn:namedmap} charts only these, it too is complete in this respect.
		
			Lastly, as we would like to chart our first-places in some relevant manner, and not just list them, we look for relevant ways these pirate haunts were related. This is done by hand. This done, the first and most obvious relation is distance, of course. But in addition to this, we have cost, that is, the difficulty of getting from one first-place to the other. And we find no further salient relations.\footnote{Ignoring non-salient or missed relations, but assume no others, or assume these are subsumed by cost. We may want to exclude from or keep constant in cost the weights of nautical knowledge/practice/expereience.}
			
			As cost subsumes distance, we consider only cost. As those were the only salient relations among our first-places, cost gives us \emph{all} the ways our first-places are related.\footnote{All relations we care about, assuming hand check was thorough.} So say some first-place is \ment{close} to another if it's \emph{not} costly to get from the one to the other and vice versa.\footnote{Giving us undirected closeness from directed cost, for simplicity, but we could keep cost directed. By \ment{not costly} we mean for some cutoff of cost which we deem sufficiently cheap.} Then, we have that \(\bc\), \(\mt\), and \(\id\) are all close to each other, while \(\nt\) and \(\et\) are close to none.\footnote{Ignoring self-closeness.}
			
			So, we now have one relation which subsumes all salient or relevant ways in which our first-places are related. As \ref{eqn:namedmap} charts only this relation, it too is complete in this respect.
			
			This fully gives us \ref{eqn:namedmap}, so our work here is done.
		%
		% subsubsec log
		\subsubsection{Log}
		\label{sss:log}
			We start by noting that in existing stories* we have very many first-times and last-times (when stories* start and end). We further have two relevant phenomena, namely, our first-theme and last-theme ie piracy and logging. Call their temporal start and end \ment{first-piracy}, \ment{last-piracy}, \ment{first-logging}, and \ment{last-logging}.
			
			Now, we note that none of these first-times pre-dates first-piracy, and none of these last-times post-dates last-piracy. So, first-theme is present at every point in time in existing stories*. We further note that \emph{only} some of these first-times pre-date first-logging, and none of these last-times post-dates last-logging. So last-theme is either present at every point, or else present only in later points in time in existing stories*.
			
			So, of the temporal points of first-theme and last-theme, \emph{none} show up in \emph{all} existing stories*.
			
			We salvage this sad state by looking at the stories* which \emph{do} include first-logging. In almost all of them, this point is not given prominence but rather minimised in favour of a second point, that when the incidence of last-theme reached some significance threshhold, call it \ment{first-significant-logging}.\footnote{There is a further \ment{first-significant-piracy} in some existing stories*, but it's not needed here.}
			
			If we now look for first-significant-logging in existing stories*, we do find it in all of them.
			
			Instead of plotting first-significant-logging, however, we take stock of \emph{why} first-logging is only in some stories*, and why first-significant-logging is favoured. The immediate answer is that we can't see the incidence of last-theme, of course.\footnote{There being scant records on logging.} But additionally, we further presume that it was \emph{in}significant for some time after first-logging, and that it first reached significance only after first-significant-logging.\footnote{Significance regarding frequency of logging, or amount logged, or number of loggers, and so on.}
			
			So, all existing stories* presume that there was some non-null time between first-logging and first-significant-logging. Again, instead of plotting, we take stock of why this is.
			
			Here, we find that all such stories* presume that first-theme \emph{lead} to last-theme in a \emph{specific} way, namely, by pirates' logging a little bit from first-logging, then more and more, until logging a lot from first-significant-logging.\footnote{For some significance threshhold.} Which is to say, all existing stories* presume the increasing \emph{co}incidence of our first-theme and last-theme.
			
			This, finally, would make for a useful organising principle for \ref{eqn:namedlog}, due not just to existing consensus, but additionally to the greater insight we have into the incidence of first-theme.
			
			Now, we have \emph{strongly}-significant positive coincidence between first-theme and last-theme starting from first-significant-logging. But even \emph{weakly}-significant positive coincidence might be useful when using first-theme incidence as proxy for last-theme incidence.\footnote{For some significance value of \ment{strong} or \ment{weak}.}
			
			So, call the point in time when first-theme and last-theme first significantly positively coincided \(t_c\), and add to \ref{eqn:namedlog}.
			
			Naturally, in the period from start-logging to \(t_c\), first-theme and last-theme did not significantly coincide at all.\footnote{We have no examples of significant negative coincidence in existing stories*.} For stories* with first-times at or after first-logging, this covers the entire period. For stories* with first-times pre-dating first-logging, there would still be no significant coincidence at all, given no incidence of last-theme. So call this period of insignifcant coincidence \(\pre\), and call first-times \(t_0\), and add both to \ref{eqn:namedlog}.
			
			Now, in the period after \(t_c\), first-theme and last-theme did significantly coincide at all points, but in all stories* this is only so up to a point.\footnote{Due to various reasons, including fall of last-theme incidence below significance, fall of first-theme incidence below significance, so on.} This is, of course, what we would expect, given that last-theme outlived first-theme by a lot.
			
			So, at some point after \(t_c\), first-theme and last-theme first \emph{stopped} significantly or positively coinciding. Call it \(t_d\) and add to \ref{eqn:namedlog}. Furthermore, call the period between \(t_c\) and this point \(\cpd\), and likewise add to \ref{eqn:namedlog}.\footnote{Only one path is possible between these points, ie significant coincidence all the time.}
			
			After \(t_d\), all stories* agree that the incidence of first-theme fell below some significance threshhold, while only some say as much of last-theme. Nonetheless, the first consensus is all we need to see all possible paths to all last-times.
			
			So, from \(t_d\) to last-time, the incidence of last-theme either was significant at all points, or it was not, while the coincidence of first-theme and last-theme was either insignificant at all points, or it was not.
			%
			% tab paths
\begin{table}
\caption{Possible paths from \(t_d\) to last-time.}
\label{tab:pathsforlog}
\begin{tabular}{ccccc}
No	&First-theme incid &Last-theme incid &Coincid &Coincid direcn\\
(1)	&insig	&sig	&insig	&irlvt\\
(2)	&insig	&sig	&sig	&+ or --\\
(3)	&insig	&insig	&insig	&irlvt\\
(4)	&insig	&insig	&sig	&+ or --
\end{tabular}
\end{table}
% % \label{tab:pathsforlog}
			
			The paths possible from \(t_d\) to last-time are listed in Table~\ref{tab:pathsforlog}. We note that coincidence is insignificant at \emph{all} points in time in paths (1) and (3), while logging incidence is significant at \emph{all} points in (1) and (2).
			
			For simplicity, we merge paths (1) and (3) and call them \(\pst_0\), and merge paths (2) and (4) and call them \(\pst_1\), and call last-times \(t_1\), and add all to \ref{eqn:namedlog}. 
			
			This fully gives us \ref{eqn:namedlog}, so our work here is done.
	%
	% subsec decency
	\subsection{Decency}
	\label{ss:decency}
		We first note that \ref{eqn:namedlog} requires at least all data on the incidence of piracy in the Bay, so say we have that. We must then (somehow) identify \(t_c\) and \(t_d\) in these data, and get a direct picture of piracy, and proxy one of logging, in the Bay from \(t_0\) to \(t_1\).\footnote{\(t_c\) and \(t_d\) are the only defined temporal points in \ref{eqn:namedlog}.} This straightforwardly falls short of answering any forbidden question, and yet directly answers the first required one, while at least partially answering the second required question.\footnote{Assuming good data.}
		
		So, \ref{eqn:namedmap} and \ref{eqn:namedlog} are decent.
%
%
%
% sec use
\section{Use}
\label{s:use}
	In this section, we build a number of stories* with \ref{eqn:namedmap} and \ref{eqn:namedlog}, first to fit existing theories in literature, and then to see what a free-form one might look like.\footnote{Literature for this exercise haphazardly chosen from Table~\ref{tab:literature}. Qualify all claims regarding litereture here as being \emph{for} a given possible or acceptable reading.}
	%
	% subsec litstories
	\subsection{To fit literature}
	\label{ss:litstories}
		%
		% subsubsec bulmers
		\subsubsection{Bulmer-Thomases}
		\label{sss:bulmers}
		The Bulmer-Thomases offer a haven-camp story* set spatially in \(\bc\), \(\id\), and \(\et\) only.\footnote{Buccaneers possibly use as haven in 1642--1670 and turn to logging in 1670s \cite{bul16}*{pp 137--138, 145, 151--156}.} They explicitly set \(t_0\) in 1642 and \(t_d\) in 1670s and \(\pst\) via first path in \(\pst_0\). \footnote{Path (1) in Table~\ref{tab:pathsforlog}.} They do not explicitly set \(t_1\), but we read \emph{prior} to 1705 for it, and further set \(t_c\) as mid 17th century from explicit values.\footnote{Implicit in \cite{bul16}*{p 149}.} This completes \ref{eqn:namedlog}.
		%
		% subsubsec campbell
		\subsubsection{Campbell}
		\label{sss:campbell}
		Campbell offers a haven-camp-settlement story* set spatially in \(\bc\) only.%
		%
		\footnote{Loose pirate communities from probably mid 16th cent \cite{cam11}*{pp 95--96, 100}, pirate haven on St George's and surrounding cayes from probably mid 1550s \cite{cam03}*{pp 176, 179}, buccaneer haven from 17th cent \cite{cam03}*{p 174}, buccaneer haven definitely from 1670s \citelist{\cite{cam11}*{pp 82--83} \cite{cam03}*{p 176, 178}}, partial Providencer presence from 1641 \cite{cam11}*{p 129}, first important settlement on St George's definitely from 1765 \citelist{\cite{cam11}*{pp 121--122, 129} \cite{cam03}*{pp 175, 179--180}}, and non-buccaneering ie sedentary ie logging settlement from mid 17th cent \cite{cam03}*{p 171}. The Providencer answer possibly a misreading.}
		%
		They explicitly set \(t_0\) in mid 1550s, \(t_d\) in mid 17th century, and \(\pst\) via first path in \(\pst_0\).\footnote{Explicit in \cite{cam03}*{p 171}. Also implicit in \cite{cam11}*{pp 91--92, 104, 108}. But Campbell does not mean \emph{only} \forn{H campech} by \ment{logwood} \cite{cam11}*{pp 104--105}, whereas we do, so a bit tricky to read.} % p91-92 = by 1708, p104 = by 1680s, p108 = by early 1670s
		They do \emph{not} quite so explicitly set \(t_c\) nor \(t_1\), but we read early 17th and early 18th centuries for them.\footnote{That is, \(t_c\) implicitly set to Elizabethan era in \cite{cam11}*{pp 106--107}, and to prior to 1695 and 1697 and 1655 in \cite{cam11}*{pp 84, 91, 106}, respectively for latter. And \(t_1\) implicitly to 1708 in \cite{cam11}*{pp vii, 93}, and possibly more. Again, for \ment{logwood} meaning \ment{dyewood}.} This completes \ref{eqn:namedlog}.
		%
		% subsubsec fenix
		\subsubsection{Fénix}
		\label{sss:fenix}
		Sierra O'Reilly offers a haven-settlement story* set spatially in \(\bc\) only, and temporally in mid 17th century for \(t_1\) only.%
		%
		\footnote{Pirates settle on Old from mid 17th cent \cite{fen49}*{p 3}. This is a Wallace story*. \citelist{\cite{cam09}*{pp 72--110} \cite{bul16}*{pp 137--151} \cite{res19}*{pp 19--24}} have thoroughly shown that \ment{Wallace} in such stories* is most likely apocryphal. This, however, is also a Meridian story*, that is, one from Merida built on or after the 18th cent, when that city was heavily focussed on the Bay. (\citelist{\cite{car71}*{pp 55, 210} \cite{car78}*{pp 260--261} \cite{anc78}*{pp 370--376} \cite{pen69}*{pp 217--219}} are also Meridian stories*.) By the former, these stories* are at least partly apocryphal. By the latter, these stories* are at least possibly better-sourced than some others, given the non-zero chance of unique primary sources in Merida in and after the 18th cent. So, we ignore most Wallace details here, and read \ment{one of the first Baymen} for \ment{Wallace}. Note this means we disregard the move by \cite{bul16}*{pp 137--138, 145, 151--156} from \ment{Wallace}'s being apocryphal to temporal restrictions on acceptable answers to question*, and further disregard charges of dishonesty in Wallace stories* in \citelist{\cite{bul16}*{pp 138, 140--141} \cite{cam09}*{pp 87--88, 105--106} \cite{res19}*{pp 23--24}}, as the former follows and the latter regards only \ment{Wallace} properly (not symbolically) read.}
		%
		We find nothing further, so do not complete \ref{eqn:namedlog}.
		%
		% subsubsec reads
		\subsubsection{Read's}
		\label{sss:reads}
		This Bayman offers a camp-settlement story* set spatially in \(\bc\) only.%
		%
		\footnote{Buccaneer-loggers camp-settle on Old then camp-settle up to Hondo in mid 17th cent \cite{rds32}*{p 3}. This is one of the earliest stories* available in print (at least), and one of the only to ascribe to later Bacalar strikes a reason over piracy (possibly). The unsigned letter, dated Campeachy, 24 Nov 1731, reads in part:\begin{quote}As to the State of the Bay of Honduras, I shall give it you as briefly as possible. The ancient City of Bacalar, situate in that Part of the Province of Yucatan, which lies on the Bay of Honduras, was twice sack'd, and at last totally ruined by the English many Years ago; on which the Logwood-Cutters of that Nation, who had settled on the River of Valis, possessed themselves of the New River and that of the Hondo; which last is distant from the Ruins of Bacalar about five Leagus. Here they built a great many Houses and Hutts, and employ'd Multitudes of Negroes in cutting Logwood, which was transported to Jamaica and Europe by Numbers of Vessels trading from thence to the Bay.\end{quote} \cite{res19}*{pp 13, 16} deems a similar contemporaneous story* `a politically motivated rhetorical exaggeration' (ie the 1734 Pitt story* that logging camps on Old `had been possessed by the English for more than a hundred years,' in AGI Mexico 3099 ff 5--10). We partially disagree on this.} % agi not digitised http://pares.mcu.es/ParesBusquedas20/catalogo/description/374685
		%
		They explicitly set \(\pst\) as the first path in \(\pst_0\). They do not explicitly set \(t_1\), but we read \emph{prior} to mid 17th century for it. Nothing further found, so no complete \ref{eqn:namedlog}.
		%
		% subsubsec restall
		\subsubsection{Restall}
		\label{sss:restall}
		Restall offers a camp-settlement story* set spatially in \(\bc\) and \(\nt\) only, and temporally in 1662 for \(t_0\) and 1717 for \(t_1\) only.\footnote{Haphazard haven-camp possibly from 1662, and settled logging camp definitely from 1716 \cite{res19}*{pp 2--3, 12--13, 17, 20--21, 30}.} Nothing further found, so no complete \ref{eqn:namedlog}.
	%
	% subsec freestories
	\subsection{Free-form}
	\label{ss:freestories}
		%
		% subsubsec storybuild
		\subsubsection{Build}
		\label{sss:storybuild}
		We first set places in \ref{eqn:namedmap}. For entrance, we set \(\et\), given this was the historical entrance in the 16th century, and given the eastern location of resident pirate population in the 17th. Now, of non-entrance places in \ref{eqn:namedmap}, \(\id\) is closest to \(\et\), and is coincidentally is the most prominent haven in the Bay in literature, and in incidence data, so we set \(\id\). Of course, \(\bc\) is close to \(\id\), and the most popular place in reviewed stories*, so we further set \(\bc\). But now we have chosen places in the Bay which are both close to \(\id\) and either good logging or haven locations. If this is our criteria, then we can't really leave out \(\mt\), which is equally as close to \(\id\) as \(\bc\), and would make for good logging, and is the second most prominent haven in the Bay after \(\id\) in incidence data. So we lastly set \(\mt\), which means we have set \emph{all} of the inner Bay in \ref{eqn:namedmap}.\footnote{\ment{Incidence} as in Figure~\ref{fig:piracyinbay}. \ment{Reviewed} as in Section~\ref{ss:litstories} or Table~\ref{tab:literature}.}
		
		We now set points in \ref{eqn:namedlog}, starting with \(t_c\). For this, we might first find incidence peaks and work out \(t_c\) from there.\footnote{We see three major peaks in Figure~\ref{fig:piracyinbay}, ie in the 1600s, 1630s to early 1650s, and late 1670s to 1680s.} Or we might follow reviewed stories* to set \(t_c\). For novelty, we do the former here, and focus on the second incidence peak in data, and set \(t_c\) during its early years, namely, the early 1630s.\footnote{There seem to be more reasons to choose the second peak over the first or third, but the latter are still live options. Reasons likely or possibly include its size and scope, plus its matching a boom in resident pirate population and logwood demand, and matching a bust in Spanish commerce, defence, available ready-for-market logwood, and coastal native population. Similarly, there might be more reasons to choose earlier rather than later years within the peak for \(t_c\), but all are still possible.} Now, we know logging decoupled from piracy by at least 1662 in Campeachy, so say it similarly decoupled in the Bay, and set \(t_d\) in early 1660s.\footnote{Or we might set the first years following the second peak, ie late 1650s. Note, some literature gives 1662 as \emph{the} start date of logging in Campeachy. Strictly, however, logging there started at least \emph{by} said year.} We lastly follow reviewed stories* to set \(t_0\) in early 1590s, and \(t_1\) in early 1700s, and \(\pst\) as the first path in \(\pst_0\), thereby completing \ref{eqn:namedlog}.
		%
		% subsubsec storyinterpret
		\subsubsection{Interpretations}
		\label{sss:storyinterpret}
		There are a number of ways to interpret this story*, but we offer only a few here. First, say we adopt resident pirates' possible sense of \ment{founding} and \ment{settlement}. Then even a seasonal pirate haven or camp might count as polities, making them the earliest predecessor states available, and giving us an answer to question* circa \(t_c\) in the early 1630s. Or say we first scrutinise their initial senses of founding and settlement. Then it is \emph{still} possible to find them reasonable, again, giving us the same answer to question*. Of course, we might just as possibly find them wanting, and adopt stricter senses, giving us an answer to question* in circa \(t_1\) in the 1700s.\footnote{The point here being not that any of these is the uniquely suitable interpretation, but rather that \emph{none} of these is \emph{un}suitable, modulo the story*'s being well built and interpretation fleshed out.}
%
%
%
% sec review
\section{Review}
\label{s:review}
	In this section, we review stories* built in Section~\ref{s:use}, offering a number of observations.\footnote{Mostly only on Section~\ref{ss:litstories} stories* until Section~\ref{ss:onfreestories}. Observations haphazardly made.}
	%
	% subsec onnolog
	\subsection{On \ref{eqn:namedlog} completion}
	\label{ss:onnolog}
		The first thing we note from reviewed stories* above is our having three \ref{eqn:namedlog} not complete.
		
		Now, we read all those stories* as presuming piracy lead to logging.\footnote{And as stories* not missing parts, which might not be the case for \lit{Read's}, but we ignore that here.} This still seems correct on second thought. But per Section~\ref{sss:log}, we expect a complete \ref{eqn:namedlog} from such stories*. So we probably made a mistake somewhere in there. We now review the steps in that section to find the error(s).
		%
		\begin{steps}[In Section~\ref{sss:log}]
		We built \ref{eqn:namedlog} (up to \(t_c\)) like so:
		\begin{enumerate}
		\item First-logging is \emph{not} in all stories*.\footnote{Or is minimised.}
		\item First-significant-logging \emph{is} in all stories*.\footnote{Or is not similarly minimised.}
		\item So, all stories* presume insignificant logging to first-significant-logging.
		\item So, all stories* presume piracy lead to logging in a specific way.\footnote{By pirates logging a little bit from first-logging, then more and more, until logging a lot from first-significant-logging.}
		\item So, all stories* presume increasing coincidence of piracy and logging.
		\item So, for \(t_c\) the point when piracy and logging first significantly positively coincided, \(t_c\) is in all stories*.
		\qedhere % apparantly needed when ending in list
		\end{enumerate}
		\end{steps}
		%
		We now find (1) still holds for all reviewed stories*, and find (2) holds for all but \lit{Fénix}. Now, this is a temporally compressed land-and-settle story*, while the rest are drawn out squat-squat-settle ones.\footnote{Distinction first from \cite{cam11}*{pp 95--96}. The associated distinction (intentional vs accidental) is not imported here.} We ignore stories of the former ilk hereon.
		
		So both (1) and (2) hold for all our stories*. We further find (3) holds too, but the move to (4) is tricky, so we now spell it out more fully.
		
		First, let \ment{piracy-win*} be \ment{successful piracy}, and \ment{piracy-fail*} be \ment{failed piracy}.\footnote{For simple or minimal \ment{success} eg where pirates surprised a ship or sacked a port, even if they got no/nil booty. And similar \ment{failure} eg where they found no ships nor came to port.} And let \ment{piracy*} be either.
		
		That done, we now review step (4). We find that swapping in \ment{piracy-win*} for \ment{piracy} does not work, whereas \ment{piracy-fail*} \emph{does}. That is, for the particular way of \ment{leading} to spelt out in (4), we want to say that all stories* presume piracy-fail* lead to logging, \emph{not} that piracy-win* did so.
		Now let (4a*) be `All stories* presume piracy-fail* lead to logging in \emph{some} way.' Then we find (4a*) holds for all our stories*, independently of (3), and ignore stories* where it doesn't hereon.%
		%
		\footnote{Eg where successful pirates discovered logging then afterwards intentionally/en masse set out only to log, ie sudden conversion. This might be the case eg in stories* where social/legal pressures forced successful pirates en masse to logging, but possibly only if pressures suddenly/rapidly (not gradually) materialised. So, we ignore stories* where pirates suddenly/en masse dropped piracy for logging.}
		%
		
		At this point, it might be helpful to see how (the incidence of) piracy-win* and piracy-fail* might look like plotted together. We first imagine a bell curve for piracy*.\footnote{Note, we presume both piracy-win* and piracy-fail* incidence are multimodal distributions with various peaks and valleys, and so too piracy*. So the following is just a slice of these curves.} Then, under the piracy* curve, piracy-win* and piracy-fail* look like either 1 or 2 in Figure~\ref{fig:yaynayinterference}.%
		%
		\footnote{Figure~\ref{fig:yaynayinterference} presumes pirates were more successful than not (more piracy-win* than piracy-fail* incidents), but even if otherwise, the following still works. Additionally, cases where the piracy-fail* peak precedes the piracy-win* peak are shown later on, but the following still works in these.}
		%
		In the first case, we say they \emph{closely} tracked or \emph{positively} coincided, and say they loosely tracked or negatively coincided in the second case.%
		%
		\footnote{Note, in Section~\ref{sss:log} we said we had a clearer picture of piracy than logging. While this still holds, we now say we have a clearer picture of piracy-win* than logging, and further a clearer picture of piracy-win* than of piracy-fail*. Or possibly no picture of piracy-fail* at all, unless by proxy from piracy-win*. We further note our relative lack of non-Spanish records for piracy* in the Bay, and further, their tendency to label any foreigner \ment{pirate} (inflating piracy-win* incidence), and the (we think likely) chance of their being so overwhelmed or desentised over time to piracy* so as to stop recording/reporting individual piracy-win* incidents (especially minor ones, thereby deflating incidence).}
		%
		% fig interference
\begin{figure}
% pgsplots code begins
\begin{tikzpicture}[baseline] % baseline per pgsplots man
	\pgfmathdeclarefunction{gauss}{2}{\pgfmathparse{1/(#2*sqrt(2*pi))*exp(-((x-#1)^2)/(2*#2^2))}} % to use in \addplot
	\begin{axis}[
		small,
		xlabel=1, ylabel=, % labels
		xtick=\empty, ytick=\empty, % ticks
		]
	\addplot [
      red,
		domain=0:10,
		samples=100,
      ] {gauss(4.5, 1.5)};
	\addplot [
      blue,
      domain=0:10,
      samples=100,
      ] {gauss(3, 1)};
	\end{axis}
\end{tikzpicture}% \% here to avoid whitespace
\hskip 10pt % per pgsplot man
\begin{tikzpicture}[baseline] % baseline per pgsplots man
	\pgfmathdeclarefunction{gauss}{2}{\pgfmathparse{1/(#2*sqrt(2*pi))*exp(-((x-#1)^2)/(2*#2^2))}} % to use in \addplot
	\begin{axis}[
		small,
		yticklabel pos=upper,
		xlabel=2, ylabel=, % labels
		xtick=\empty, ytick=\empty, % ticks
		]
	\addplot [
      red,
		domain=0:10,
		samples=100,
		] {gauss(6, 1.5)};
	\addplot [
      blue,
      domain=0:10,
      samples=100,
      ] {gauss(3, 1)};
	\end{axis}
\end{tikzpicture}
% pgsplots code ends
\caption{Constructive vs destructive interference/close vs loose tracking/positive vs negative coincidence of piracy-yay* (blue) and piracy-nay* (red).}
\label{fig:yaynayinterference}
\end{figure}
%
 % \label{fig:yaynayinterference}
		
		Now, let (4b*) be `All stories* presume piracy-fail* lead to logging in a \emph{particular} way, namely, via increasingly-significant (positive) coincidence.'%
		%
		\footnote{Logging a little bit from first-logging, then more and more, until logging a lot from first-significant-logging, with all of this occurring almost only during piracy-fail* incidents. That is, logging and piracy-fail* coincide only a few times from first-logging, then more and more, until coinciding very many/almost at all times from first-significant-logging.}
		%
		Then, the move from (4a*) to (4b*) is natural or likely, and so made.\footnote{Though (4b*) does not follow from (4a*), but we can't see how else (4a*) might obtain. May be worth exploring this.}
		
		This, of course, immediately gets us to (5*) and (6*) (steps (5) and (6) with \ment{piracy-fail*} swapped in for \ment{piracy}). But we now have a not-so-useful \ref{eqn:namedlog}.
		
		At this point, it might be useful to get a clearer view of 2 in Figure~\ref{fig:yaynayinterference}. The possible orders of peaks are sketched in Figure~\ref{fig:yaynayorder}.\footnote{The shown orders are also possible for 1 in Figure~\ref{fig:yaynayinterference}, with closer peaks, but that's irrelevant here.} That is, if piracy-win* and piracy-fail* only loosely track each other, rather than closely, then periods of heightened piracy-fail* incidence might precede, or succeed, or both precede and succeed periods of heightened piracy-win* incidence.\footnote{Probably 2 in Figure~\ref{fig:yaynayorder} is likelier than the others, eg if pirates' being successful lead to more pirates trying (in a delayed manner).}
		%
		% fig order of peaks
\begin{figure}
% pgsplots code begins
\begin{tikzpicture}
	\begin{groupplot}[
		group style={
			group size=2 by 2,
			every plot/.style={domain=0:10,samples=100,},
			},
		]
		\nextgroupplot [small,xlabel=1, ylabel=,xtick=\empty,ytick=\empty]
			\addplot [red,] {gauss(2, 1.5)};
			\addplot [blue,]{gauss(5, 1)};
		\nextgroupplot [small,xlabel=2, ylabel=,xtick=\empty,ytick=\empty]
			\addplot [red,] {gauss(8, 1.5)};
			\addplot [blue,]{gauss(5, 1)};
		\nextgroupplot [small,xlabel=3, ylabel=,xtick=\empty,ytick=\empty]
			\addplot [red,] {gauss(2, 1.5)};
			\addplot [red,] {gauss(8, 1.5)};
			\addplot [blue,]{gauss(5, 1)};
	\end{groupplot}
\end{tikzpicture}
% pgsplots code ends
\caption{Possible order of peaks of piracy-win* (blue) and piracy-fail* (red) when they negatively coincide with/loosely track each other (as in 2 of Figure~\ref{fig:yaynayinterference}).}
\label{fig:yaynayorder}
\end{figure}
% % \label{fig:yaynayorder}
		
		So, if/when we have 1 in Figure~\ref{fig:yaynayinterference}, piracy-win* data from historical records might serve as a proxy picture for piracy-fail*, regardless of the order of their peaks. However, if/when we have 2 in Figure~\ref{fig:yaynayinterference}, then unless we know which of 1--3 in Figure~\ref{fig:yaynayorder} we have, piracy-win* data would \emph{not} make for good proxy for piracy-fail*.\footnote{With options 1 and 2 in Figure~\ref{fig:yaynayinterference} not exclusive, of course.}
		
		As that's the case, then \ref{eqn:namedlog} might be jointly decent with \ref{eqn:namedmap}, but it's not of much use as regards getting a picture of logging incidence.
		
		But as regards our stories* in Section~\ref{ss:litstories}, if the above is correct, then we ought to have gotten at least \(t_c\) (and therefore \(t_c^*\)) in \lit{Read's} and Restall.\footnote{For \(t_c^*\) as the temporal point when piracy-fail* and logging first significantly positively coincided. Note \lit{Fénix} was discarded above.} However, we get said point in neither. For \lit{Read's}, we say the story* is simply abridged, and so supply a \(t_c^*\) ourselves. For Restall, the story* is \emph{not} abridged, so something else is at play.
		
		We consider three possibilities for Restall. 
		
		First, say some phenomenon \ment{leads} to another \ment{coarsely} if they (as a \emph{whole}) are appropriately related or \ment{tied}, and say it leads to another \ment{finely} if enough of their (\emph{individual}) incidents are appropriately related or tied. Then one might presume that piracy coarsely lead to logging sans presuming one finely lead to the other, for instance, via an en masse or swift pirates-to-loggers conversion. For (4a*) we previously discarded all the swift conversion stories*. We now revise that, and say we discard all stories* that do \emph{not} presume piracy \emph{finely} lead to logging. We read Restall as presuming fine leading to, and so keep.
		
		Next, say piracy finely lead to logging, but with most of the tied incidents \emph{outside} the Bay. Now, this might follow simply from higher piracy incidence outside the Bay, or a stronger piracy-logging tie outside the Bay (or both).
		
		The first case (of higher piracy outside the Bay), all else equal, is granted, and its implications disregarded as causing us no trouble.
		%
		\footnote{In this case, we would have lower logging incidence in Bay than outside, resulting possibly in insignificant logging only, and so in a delayed first-significant-logging. But the move to insignificant logging imports interpretation (how high one's standards for significant logging are, or use of relative vs absolute standards), and so is not relevant here.}
		%
		
		The second case (of a stronger tie outside the Bay), \forn{cet par}, is \emph{not} granted.%
		%
		\footnote{Though this is a live possibility in either direction, and in fact, we probably have either a location- and time-sensitive tie or coincidence between piracy and logging, eg stronger tie in Bay when pirates had less luck there, or stronger tie in Campeachy after Jamaican invasion, so on.}
		%
		Depending on standards used, this might naturally lead to a too-weak-to-count tie in the Bay, and so to only-insignificant piracy-logging coincidence in Bay. For now, all such stories* are discarded, giving us location insensitive tie and coincidence.\footnote{May be worth exploring this.}
		
		The third case (both of the above) is not granted, as the second case is not.
		
		Of the three cases above, we read Restall as of the first variety. That is, we say they have high standards for logging significance, and interpret \(t_c^*=t_1\).
		
		In sum, \ref{eqn:namedlog} isn't of much use, and there were a number of errors in building it which ought to be corrected.%
		%
		\footnote{We further note coincidence defined \emph{only} spatiotemporally, vs coincidence as importing more than proximity of incidents, or else not including spatiotemporal proximity criteria at all, eg requiring additionally/only a tie/appropriate relation between them. It looks like we might only need the minimal one in this paper (or for a corrected \ref{eqn:namedlog}). For minimal coincidence, we further distinguish undirected boolean (values \code{true} or \code{false} for `for any piracy incident, there is at least one logging incident nearby; and for any logging incident, there is at least one piracy incident nearby') from directed categorical (values of \(-1\) or \(0\) or \(1\) with \(1\) for boolean coincidence and \(-1\) for `for any piracy incident, there is no logging incident nearby; and for any logging incident, there is no piracy incident nearby' and \(0\) for `for any piracy incident, there may/may not be at least one logging incident nearby; or for any logging incident, there may/may not be at least one piracy incident nearby'). We may even further make the directed categorical into a graded/quantitative measure with vlaues in interval \([-1,1]\) by mapping coincidence \ment{ratio} (eg `for any piracy incident in some given year, there are on average \(n\) logging incidents nearby'). So, coincidence gets very messy, and we might just be better off discarding it and instead counting logging as a strict subset of piracy-fail* incidence, eg we might look at only piracy-fail-only*, piracy-fail-cum-logging*, and logging-only* incidence curves (appropriately defined).}
		%
	%
	% subsec onmapclusters
	\subsection{On \ref{eqn:namedmap} clustering}
	\label{ss:onmapclusters}
		\(\bc\) seems to be much more popular than \(\id\) and \(\mt\) in the inner Bay, with \(\id\) set (minimally) only by Campbell, and \(\mt\) set by none.
	%
	% subsec ontrendsuse
	\subsection{On missing trends}
	\label{ss:onmissgtrends}
		The use of historical trends or patterns in building stories* seems to be a bit unpopular, with only Restall offering a story* so built (from cartographic patterns, in their case) rather than one from discrete data points.%
		%
		\footnote{Missing patterns, in addition to incidence of piracy, might include dwindling Spanish commerce in the Bay, a growing resident pirate population near Bay, growing social/legal pressures against piracy (though this one \emph{is} generally noted), and growing textile industry (and thus logwood demand) in England (and possibly the Netherlands).}
	%
	% subsec onmissinginterpret
	\subsection{On missing interpretations}
	\label{ss:onmissinginterpret}
		A few possible or even likely interpretations seem to be unpopular. For instance, the \ment{Bay}'s meaning \emph{all} of the inner Bay for early resident pirates is one such.%
		%
		\footnote{\ment{All} meaning \(\id\), \(\mt\), \(\bc\). \ment{Bay}'s meaining \(\id\) and \(\bc\) is also possible. We imagine the inner Bay may have functioned as an early \ment{Bay Triangle}, eg if piracy and logging were coupled before the invasion of Jamaica. We further think this might be important not only to understanding early and later Baymen's sense of place, but to interpreting their answers to question*, \forn{mut mut}. For instance, say they derived their sense of ownership from use. And say they first used \(\id\) as a haven at some time \(t_i\), and only later used \(\mt\) and \(\bc\) as havens or camps at \(t_k\). Then if they already felt like they owned \(\id\) at some intermediate \(t_j\), later use of \(\mt\) and \(\bc\) might count to them as \emph{expansion} of their original territory, such that they might claim first ownership of \emph{all} the inner Bay from \(t_i\) or \(t_j\), despite only gaining two-thirds of it from \(t_k\). (This seems analogous to saying `America was founded in 1776,' despite a vast majority of the country not existing at the time, and despite the War not being won and the Union not being constituted but for several years later.) Further, say any of these places fell into disuse from \(t_m\) for some extended time. If the Baymen came back at some later \(t_n\) to find no one using said place, then they might consider this simply a \emph{continuation} of their original use and ownership from \(t_i\) or \(t_j\), rather than a \emph{renewal} of use or ownership. (This seems analogous to answering question* with any pre-17xxx date, as Baymen's presence here has \emph{only} been uninterrupted since this date.) Finally, say it was the \emph{original} place (\(\id\) here) which fell into disuse from \(t_m\), this time \emph{permanently}. Then the Baymen might nonetheless claim ownership from \(t_i\) or \(t_j\), most especially if they viewed themselves as constituting a polity. (Analogously, if the Thirteen Colonies' successor states seceded from America tomorrow, it would \emph{still} be the case that said country was founded in 1776.) We lastly note that the later political expediency of recognising or rehashing such claims (or not) does \emph{not} seem relevant to their truth or reasonableness, and that official recognition (eg by Providence) might only further encourage or foster self-identification as a polity. The \ment{Bay Triangle} meaning the Bay, Jamaica, and the Mosquito Shore is named in \citelist{\cite{cam11}*{p 130} \cite{cam03}*{p 186}}, but was first identified by \cite{nay89}*{pp 46--53} for 18th century commerce, it seems.}
		%
		Further examples might include the discovery of Santo Tomas or the Tipu Rebellion having important or far-reaching ramifications, or early Spanish maritime control of the Bay \emph{not} being effective.%
		%
		\footnote{The Rebellion for shifting Maya population away from coastal \(\bc\) (at least), and the discovery for shifting Spanish shipping even closer to \(\mt\) from Puerto Caballos. On the first, we note that some literature implies that demographic flight took a few \emph{years}, despite its being a live likelihood that it took only a few \emph{months}, at most, from \forn{k'atun 1 ahaw} (6 Jun 1638). Similarly, some literature does \emph{not} properly name it (with a capital \emph{R}), but not doing so seems like an oversight, or just stylistic. This seems to be in line with the general reluctance to properly name significant events in literature. For instance, the near-annual Anglo-Spanish hostilities during the 1690s to 1790s are never properly named, eg as the Bay's \ment{Hundred Years' War}, despite the boon this would be to discussion, and the merit of the case. Or this might have to do with the generally loose naming in literature. For instance, quite a few works name \ment{St George's Cay} rather than \ment{Caye}, ostensibly due to the English variant being used (American or British, in this case), despite the fact that proper names are \emph{not} translated across variants (eg the Florida Keys are \emph{not} named \ment{Cays} in British nor \ment{Cayes} in Belizean English).}
		%
	%
	% subsec onfreestory
	\subsection{On free-form stories*}
	\label{ss:onfreestories}
		Our story*'s \ref{eqn:namedmap} seems solidly built, but its \ref{eqn:namedlog} does \emph{not}, this review having shown we have more choices to make, and thus likely need more data, than \ref{eqn:namedlog} strictly requires.%
		%
		\footnote{Making our story* less minimal than it ought to be. For instance, our choice of \(t_c\) presumes piracy-fail* only closely tracks piracy-win*, despite loose tracking still be a live possibility. For data, we'd probably need at least piracy incidence  for Campeachy/Yucatan. Relevant commercial and demographic trends, and piracy in the Greater Antilles, might also prove useful. A modified version of \ref{eqn:namedlog} might still work here, though.}
		%
		In addition, the given interpretations are not fleshed out, and so are poor, but as \emph{any} of them may be so supported, they \emph{all} still seem suitable.%
		%
		\footnote{Subject to our preferred sense of \ment{founding} or \ment{settlement}. There does not seem to be a principled way of choosing one among the many meanings available for these, except possibly extreme laxity or strictness, or else deference to our agents' understanding (despite their political motives). Note the latter is the case when we say `America was founded in 1776' (the Declaration being blatantly political) or `Belize gained self-governance in 19xx' or `Belize gained independence in 1981' (the choices of these points rather than other available ones being arguably political). For these last, we note the country strictly \emph{re}gained self-governance and \emph{last} gained independence in the given years, the former being being the case since founding or settlement, and possibly since unofficial majority in 18xx, and the latter gained from Jamaica in 18xx, and arguably Spain 17xx, and possibly even the Mosquito Shore in 17xx.}
		%
%
%
%
% sec concl
\section{Conclusion}
\label{s:concl}
	Minimal theories seem to be a bit more difficult to build than maximal ones, and yet might be more desireable for our inherently vague problem than the latter. The way of building theories proposed in this paper presumes, upon review, more than we ought to allow for minimal theories. A \emph{similar} method, nonetheless, seems like it might work for them.
%
%
%
% end stuff
%
% s references
\begin{bibdiv}
\label{s:references}
	\begin{biblist}
	\bibselect{pprefs}
	\end{biblist}
\end{bibdiv}
%
%
%
\end{document}