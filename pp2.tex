% top stuff
\documentclass{amsart}
%
%
% packages
\usepackage{pgfplots} % for plots - loads tikz, wc itself loads xcolor, graphicx
\usepackage{tikz-cd} % for graphs - loads tikz too
\usepackage[figuresright]{rotating} % to rotate table w LHS at bottom
\usepackage{hyperref} % for links - load last-ish
\usepackage{amsrefs} % for full cites - use author-year option? - load last
%
%
% new commands to avoid text formatting inline
\newcommand{\code}{\texttt} % for code
\newcommand{\lit}{\textit} % for book titles etc
\newcommand{\ship}{\textit} % for ship names etc
\newcommand{\ment}{\textit} % to mention (not use) terms
\newcommand{\forn}{\textit} % for foreign words
%
%
% new commands to avoid maths formatting inline
\DeclareMathOperator{\ab}{Ab} % place variable
\DeclareMathOperator{\cd}{Cd} % ^
\DeclareMathOperator{\id}{Id} % place name
\DeclareMathOperator{\mt}{Mt} % ^
\DeclareMathOperator{\bc}{Bc} % ^
\DeclareMathOperator{\et}{Et} % ^
\DeclareMathOperator{\nt}{Nt} % ^
\DeclareMathOperator{\pre}{pre} % time period name
\DeclareMathOperator{\pst}{pst} % ^
\DeclareMathOperator{\cpd}{cpd} % ^
%
%
% amsthm styles where mainclm > minrclm = gloss > note = bodytext > footnote
% plain style
\newtheorem{mainclm}{Main Claim} % for main claims
\newtheorem{minrclm}{Minor Claim} % for minor claims
% definition style
\theoremstyle{definition}
\newtheorem{gloss}{Gloss} % for glosses
% remark style
\theoremstyle{remark}
\newtheorem*{note}{Note} % for notes
%
%
% amsthm style for proof
\newenvironment{steps}{\begin{proof}[Steps]}{\end{proof}} % for steps
\renewcommand{\qedsymbol}{\textit{End}.} % for all proof env
%
%
% tikz ie pgsplots, tikzcd commands
\pgfplotsset{compat=1.18}
\usepgfplotslibrary{groupplots} % to group plots
\pgfmathdeclarefunction{gauss}{2}{\pgfmathparse{1/(#2*sqrt(2*pi))*exp(-((x-#1)^2)/(2*#2^2))}} % to use in \addplot
%\tikzcdset{} % global graphs style
%
%
% xcolor commands for darkmode - just for draft
%\pagecolor{darkgray}
%\color{white}
%
%
% hyperref commands
\hypersetup{pdfauthor={Angel E Navidad},pdftitle={On the Origins of Belize},pdfsubject={},pdfkeywords={}} % pdf metadata
%
%
\begin{document}
%
%
% starting stuff
\title{On the Origins of Belize}
\author{A Navidad}
\address{Harvard College, Cambridge, MA, US}
\curraddr{Benque Viejo, Cayo, Belize}
\email{navidad@college.harvard.edu}
\date{12 Feb 2025}
\thanks{} % incl non-grant support per ams
\begin{abstract}
\end{abstract}
\keywords{}
\maketitle
%
%
%
%
% sec intro
\section{Introduction}
\label{s:intro}
	The origins of Belize have been an open problem in historical literature since at least the 18th century, and remain so today, with over two dozen theories proposed so far, and little consensus to boot.%
	%
	\footnote{In Table~\ref{tab:literature}. Belize is the only pirate settlement in the Caribbean, and only logwood settlement, to have survived past the 19th century. As such, it is the only country in the Commonwealth for which this problem is open. As a rule though, \emph{all} such settlements seem to have unclear origins, including at least the Bay Islands, Campeachy, Catoche, the Mosquito Shore, and Tortuga. We further note the uncertainty seems baked in, such that even if we had a \emph{complete} historical record of activity, we might nonetheless still have uncertain origins. The preceding uniqueness claim has previously been articulated for smaller comparison classes, but seems to hold for all the Commonwealth. Similarly for the preceding vagueness claim. For Table~\ref{tab:literature}, note this is not a systematic literature review, and further note we claim only that given readings are possible or acceptable. All data for this paper deposited at \url{https://github.com/aenavidad/pp-0/}.}
	%
	% tab literature
\begin{table} % sidewaystable is no go in amsart apparantly
\caption{Stories* in 18th to 21st cent literature.}
\label{tab:literature}
% missg story: s596 / missg cites: s1644,s2320,s2316,rushton2014, prolly vars others
% tab code begins
\begin{sideways}
\begin{tabular}{cp{.35\textheight}lp{.2\textheight}ccc}
No	&Date	&In	&Claim	&Pub&First	&Cites\\
(1)	&mid 16th cent	&Old	&Eng seek haven	&2003	&s2320	&2\\ % campbell - locn:St George's - \href{https://github.com/aenavidad/pp-0/blob/main/literature/s2320.json}{s2320}
(2)	&ca Hond flotilla est	&Old	&Wallace settles	&1925	&s770	&7\\ % asturias - first in live wiki
(3)	&ca 1610	&Old&Wallace settles	&1872	&s2252	&2\\ % egli
(4)	&pre/ca Fuensalida missions	&N	&Brit log	&1957	&s396	&3\\ % roys
(5)	&in 1617	&Old	&Wallace settles	&1925	&s770	&2\\ % asturias
(6)	&pre penult Bacl sacking	&Old	&Eng log	&1732	&s1726	&1\\ % reads 1732
(7)	&ca Prov est	&S &Provers seek refuge or farm	&1946	&s416	&2\\ % winzerling
(8)	&mid 17th cent	&Old	&Wallace settles	&1849	&s1722	&12\\ % fenix 1849
(9)	&pre treaty	&--	&Brit present	&1841	&s2254	&2\\ % mcculloch
(10)	&mid third 17th cent	&Old	&Wallace present	&1878	&s506	&1\\ % ancona
(11)	&in 1638	&Old	&Brit wreck	&1829	&s860	&13\\ % HA 1829
(12)	&in 1638	&Old	&Wallace wrecks	&1827	&s860	&16\\ % HA 1827
(13)	&in 1640	&Old	&Wallace wrecks or settles	&1872	&s2238	&5\\ % ungewitter
(14)	&in Cromwell govt	&--	&Eng seek haven	&1826	&s860	&5\\ % HA 1826
(15)	&post Bacl/Trux sacking or post Camp logging	&Old	&Brit seek haven or log	&2016	&s78	&1\\ % b-thomas
(16)	&last half 17th cent or pre early 1680s	&N	&Eng present	&1994	&s244	&1\\ % finamore
(17)	&post Jam invasion or post Camp logging	&--	&Jamers log	&1777	&s1900	&4\\ % robertson
(18)	&ca 1663	&Old	&Wallace logs	&2010	&s2270	&1\\ % g diaz
(19)	&in 1662	or pre 1670&Old	&Wallace logs	&1944	&s418	&2\\ % c quijano
(20)	&post Catoche/Camp logging	&Old	&Jamers log	&1878	&s506	&2\\ % ancona
(21)	&post treaty	&Old	&Wallace present	&1877	&s510	&1\\ % n ortega
(22)	&last third 17th cent	&Old	&Eng present	&1878	&s506	&1\\ % ancona
(23)	&early 18th cent	&N	&Eng seek haven	&1864	&s1858	&2\\ % brockhaus
(24)	&post Camp logging	&N	&Wallace seeks haven or logs	&1871	&s2256	&4\\ % c ancona - last in live wiki
(25)	&post Camp logging	&Old	&Eng log	&2019	&s1644	&1 % restall
\end{tabular}
\end{sideways}
% tab code ends
\end{table}
%
 % \label{tab:literature}
	
	This paper seeks to xx
%
%
%
% sec problem
\section{Problem}
\label{s:problem}
	Say Belize's earliest predecessor state, the \ment{Bay}, first emerged or came to be at some time \(t_i\).\footnote{Alternatively, \ment{Honduras}.} Then Belize did \emph{not} exist at any time before \(t_i\), and \emph{did} exist at least at some times after \(t_i\).\footnote{Allowing for state discontinuity.} Then the problem of Belize's origins is a class of pertinent questions, namely, `What is the Bay?,' `When is \(t_i\)?,' `How did the Bay come to be at \(t_i\)?,' so on.\footnote{Other carvings of the problem possible, but this one seems straightforward and usful enough. We do not claim that this is how historians have previously carved this problem.} And theories for this problem answer these questions.\footnote{Appropriately qualified, eg answer \emph{all} these questions, answer them immediately or directly, so on.}
	
	We note that this carving presumes concepts of state and their emergence and succession and continuity, but does not specify them. It further allows for accidental emergence of a state, the collective sense of state, and a discontinuous state.\footnote{Spatial sense of state answered by `Where is' question, and temporal sense by `When is.' Voluntary and involuntary discontinuity allowed.} We briefly note there is no mention of the founding or establishment of a state, nor the settlement of a territory. In fact, there is no interpretation or presupposed representational model in them. So, any theory must first interpret these questions, then represent the historical record, and xx.
	
	We say a theory is minimal if it does not presuppose particular senses of the relevant concepts, and we say a problem is vague if, given a complete historical record, each particular sense of the relevant concepts would give us significantly different theories each time. Then, the problem at hand is vague, and we currently have very few minimal theories for it.\footnote{Of course, the historian's job is to fix on a particular sense and say `these are the ones we ought to use.' Nonetheless, when there is wide discordance in literature regarding the senses we ought to us, it seems to us that either a minimal theory might be useful to subsume all such senses, or else that maximal theories (which fix on a sense) ought to be transparently maximal, eg giving us reasons to reject competing senses, or at least noting that competing senses exist. There do not seem to be very many transparently maximal theories in literature.}
%
%
%
% sec theories
\section{Theories}
\label{s:theories}
%
%
%
% sec concl
\section{Conclusion}
\label{s:concl}
	xx
%
%
%
% end stuff
%
% s references
\begin{bibdiv}
\label{s:references}
	\begin{biblist}
	\bibselect{pprefs}
	\end{biblist}
\end{bibdiv}
%
%
%
\end{document}