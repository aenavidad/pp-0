% top stuff
\documentclass{amsart}
%
%
% packages
\usepackage{pgfplots} % for plots - loads tikz, wc itself loads xcolor, graphicx
\usepackage{tikz-cd} % for graphs - loads tikz too
\usepackage[figuresright]{rotating} % to rotate table w LHS at bottom
\usepackage{hyperref} % for links - load last-ish
\usepackage{amsrefs} % for full cites - use author-year option? - load last
%
%
% new commands to avoid text formatting inline
\newcommand{\code}{\texttt} % for code
\newcommand{\lit}{\textit} % for book titles etc
\newcommand{\ship}{\textit} % for ship names etc
\newcommand{\ment}{\textit} % to mention (not use) terms
\newcommand{\forn}{\textit} % for foreign words
%
%
% new commands to avoid maths formatting inline
\DeclareMathOperator{\ab}{Ab} % place variable
\DeclareMathOperator{\cd}{Cd} % ^
\DeclareMathOperator{\id}{Id} % place name
\DeclareMathOperator{\mt}{Mt} % ^
\DeclareMathOperator{\bc}{Bc} % ^
\DeclareMathOperator{\et}{Et} % ^
\DeclareMathOperator{\nt}{Nt} % ^
\DeclareMathOperator{\pre}{pre} % time period name
\DeclareMathOperator{\pst}{pst} % ^
\DeclareMathOperator{\cpd}{cpd} % ^
%
%
% amsthm styles where (i) mainclm > minrclm = gloss > note = bodytext > footnote, or (ii) clm >= gloss > etc
% plain style
\newtheorem{mainclm}{Main Claim} % for main claims
\newtheorem{minrclm}{Minor Claim} % for minor claims
\newtheorem{clm}{Claim} % for claims
% definition style
\theoremstyle{definition}
\newtheorem{gloss}{Gloss} % for glosses
% remark style
\theoremstyle{remark}
\newtheorem*{note}{Note} % for notes
%
%
% amsthm style for proof
\newenvironment{steps}{\begin{proof}[Steps]}{\end{proof}} % for steps
\renewcommand{\qedsymbol}{\textit{End}.} % for all proof env
%
%
% tikz ie pgsplots, tikzcd commands
\pgfplotsset{compat=1.18}
\usepgfplotslibrary{groupplots} % to group plots
\pgfmathdeclarefunction{gauss}{2}{\pgfmathparse{1/(#2*sqrt(2*pi))*exp(-((x-#1)^2)/(2*#2^2))}} % to use in \addplot
%\tikzcdset{} % global graphs style
%
%
% xcolor commands for darkmode - just for draft
\pagecolor{darkgray}
\color{white}
%
%
% hyperref commands
\hypersetup{pdfauthor={Angel E Navidad},pdftitle={On the Origins of Belize},pdfsubject={},pdfkeywords={}} % pdf metadata
%
%
\begin{document}
%
%
% starting stuff
\title{On the Origins of Belize}
\author{A Navidad}
\address{Harvard College, Cambridge, MA, US}
\curraddr{Benque Viejo, Cayo, Belize}
\email{navidad@college.harvard.edu}
\date{12 Feb 2025}
\thanks{} % incl non-grant support per ams
\begin{abstract}
\end{abstract}
\keywords{}
\maketitle
%
%
%
%
% sec intro
\section{Introduction}
\label{s:intro}
	The origins of Belize have been an open problem in historical literature since at least the 18th century, and remain so today, with over two dozen theories proposed so far, and little consensus to boot.%
	%
	\footnote{Belize is the only pirate settlement in the Caribbean, and only logwood settlement, to have survived past the 19th century. As such, it is the only country in the Commonwealth for which this problem is open. \emph{All} pirate or logwood settlements seem to have unclear origins though, including at least the Bay Islands, Campeachy, Catoche, the Mosquito Shore, and Tortuga. The preceding uniqueness claim has previously been made in literature for smaller comparison classes, but seems to hold for all the Commonwealth. Similarly, the generality claim has been prior made for at least some pirate or logwood settlements, but seems to hold for all of them. Theories for Belize listed in Table~\ref{tab:literature}. Note this table is not a systematic literature review, and further note we claim only that given readings are possible or acceptable. All data for this paper deposited at \url{https://github.com/aenavidad/pp-0/}. In particular, Table~\ref{tab:literature} data in \href{https://github.com/aenavidad/pp-0/tree/main/literature}{\nolinkurl{/literature}}.}
	%
	% tab literature
\begin{table} % sidewaystable is no go in amsart apparantly
\caption{Stories* in 18th to 21st cent literature.}
\label{tab:literature}
% missg long (s596) + s1644 rejection
% tab code begins
\begin{sideways}
\begin{tabular}{cp{.4\textheight}lp{.3\textheight}cc}
No	&Date	&In	&Claim	&Yes	&No\\
(1)	&mid 16th cent	&Old	&Eng seek haven	&2	&0\\ % campbell - locn:St George's - yes (s2316, s2320) - no ()
(2)	&ca Sp Hon flotilla est	&Old	&Wallace settles	&	&\\ % asturias - first in live wiki
(3)	&ca 1610	&Old&Wallace settles	&&\\ % egli
(4)	&pre/ca Fuensalida missions	&N	&Brit log	&&\\ % roys
(5)	&in 1617	&--	&Wallace present&&\\ % asturias
(6)	&pre penult Bacl sacking	&Old	&Eng log	&&\\ % reads 1732
(7)	&ca Prov est	&S	&Provers seek refuge or farm	&&\\ % winzerling
(8)	&mid 17th cent	&--	&Wallace settles	&--	&1\\ % fenix 1849 - yes () - no (s1644)
(9)	&pre treaty	&?	&Brit present	&&\\ % mcculloch
(10)	&mid third 17th cent	&--	&Wallace present	&&\\ % ancona
(11)	&in 1638	&--	&Brit wreck	&&\\ % HA 1829
(12)	&in 1638	&--	&Wallace wrecks	&--	&1\\ % HA 1827 - yes () - no (s1644)
(13)	&in 1640	&--	&Wallace wrecks	&&\\ % ungewitter
(14)	&in Cromwell govt	&--	&Eng seek haven	&--	&1\\ % HA 1826 - yes () - no (s1644)
(15)	&post Bacl/Trux sacking or post Camp logging	&--	&Brit seek haven or log	&&\\ % b-thomas
(16)	&last half 17th cent or pre early 1680s	&--	&Eng present	&&\\ % finamore
(17)	&post Jam invasion or post Camp logging	&--	&Jamers log	&&\\ % robertson
(18)	&ca 1663	&--	&Wallace logs	&&\\ % g diaz
(19)	&in 1662	 or pre 1670&--	&Wallace logs	&--	&1\\ % c quijano - yes () - no (s2316)
(20)	&post Catoche/Camp logging	&--	&Jamers log	&&\\ % ancona
(21)	&post treaty	&--	&Wallace present	&&\\ % n ortega
(22)	&last third 17th cent	&--	&Eng present	&&\\ % ancona
(23)	&early 18th cent	&--	&Eng seek haven	&&\\ % brockhaus
(24)	&post Camp logging	&--	&Wallace seeks haven or logs	&&\\ % ancona - last in live wiki
(25)	&post Camp logging	&Old	&Eng log	&1	&0 % restall - yes (s1644) - no ()
\end{tabular}
\end{sideways}
% tab code ends
\end{table}
%
 % \label{tab:literature}
	
	This paper seeks to either complicate or clarify matters by presenting the problem in Section~\ref{s:problem}, existing and possible representations of the historical record in Section~\ref{s:models}, and existing and possible theories in Section~\ref{s:theories}.%
	%
	\footnote{In this paper, we naively imagine or sketch historical reasoning as first taking a problem, then carving or refining it into well-specified questions, then representing or modelling the historical record, and finally using said model to answer the posed questions, where such answers hold \emph{for} the questions as so specified and \emph{for} the model as so built. We do not claim this is \emph{actually} how historians have previously tackled the problem at hand. Rather, we say only that the naive sketch might help to at least get a good grip of the problem, make sense of current discordance in literature, and see and fill in gaps therein. This paper uses the name \ment{Belize} anachronistically, and does \emph{not} regard the related but distinct problem of the \emph{name's} origins. The problem at hand seems to have been first noted by \cite{alm26}*{p~5} (implicitly), first discussed by \cite{nun77}*{pp~6--9, 12--14} (obliquely), and most recently discussed by \citelist{\cite{bul16} \cite{cam09} \cite{res19}}.}
	%
%
%
%
% sec problem
\section{Problem}
\label{s:problem}
	In this section, we characterise the problem in Section~\ref{ss:carving}, and note some salient features in Section~\ref{ss:features}.\footnote{We claim only that the given carving here is possible or acceptable, not that it is uniquely so, nor that it is how historians have previously carved the problem. This carving nonetheless seems straightforward and useful enough.}
	%
	% subsec carving
	\subsection{Carving}
	\label{ss:carving}
		Let \ment{Bay} be the earliest predecessor state of modern Belize, and say it first emerged or came to be at some time \(t_{\og}\).\footnote{\ment{Honduras} also appropriate here.} Then Belize did \emph{not} exist at any time before \(t_{\og}\), and \emph{did} exist at least at some times after \(t_{\og}\).\footnote{Allowing for state discontinuity.} In which case, we say the problem of the origins of Belize is a class of questions pertinent to this happening or process.%
		%
		\footnote{Appropriately qualified, eg class is non-empty and finite, questions are direct or non-embedded and immediately or straightforwardly pertinent, so on. In this carving, we imagine the pertinent questions are at least `What is the Bay?,' `Where is the Bay?,' `When is \(t_{\og}\)?,' and `How did the Bay come to be at \(t_{\og}\)?' Call these the \ment{What}, \ment{Where}, \ment{When}, \ment{How} questions of the problem. Other carvings possible, but this one seems straightforward and useful enough. We do not claim this is how historians have previously carved the problem.}
		%
		And a theory for this problem is a class of answers to those questions.\footnote{Appropriately qualified, eg class has an answer for each required question, answers are immediate or direct, so on.}
	%
	% subsec features
	\subsection{Features}
	\label{ss:features}
		We note this carving requires at least concepts of \ment{state} and their emergence and continuity and succession, but does not specify them. In not doing so, the carving is open to a number of distinct senses or specifications of these concepts.%
		%
		\footnote{We mainly have in mind here existing or possibly useful senses. These are at least the spatiotemporal vs collective vs corporate state, small vs large state, intentional vs accidental emergence of state, continuous vs discontinuous state, and voluntary vs involuntary discontinuity of state.}
		%
	
		Further, let a \ment{vague} problem be one where, given a \emph{complete} historical record, at least some of the particular senses of the required concepts give us discordant theories.%
		%
		\footnote{Appropriately qualified, eg record is relevant, discordance is significant, so on. The idea being that the historical record itself might not straightforwardly close the problem for us.}
		%
		Then we say the problem at hand is vague.%
		%
		\footnote{Further, say the problem's vagueness is \ment{artificial} if the non-extant historical record would dissolve the vagueness, and say it is \ment{inherent} if it would not. Then we say the problem might even be \emph{inherently} vague. We note the problem seems to be vague and possibly inherently so for all pirate or logwood settlements. The preceding vagueness claim has previously been made in literature for at least some of these polities.}
		%
		Further, let a \ment{minimal} theory be one which does \emph{not} presuppose or fix particular senses of the required concepts.\footnote{Appropriately qualified, eg has no trivial answers, so on.} Then we say we currently have very few minimal theories for the problem at hand.%
		%
		\footnote{These might be desireable, given the current glut of maximal theories in literature, ie of theories which \emph{do} presuppose or fix particular senses of the required concepts. Arguably though, the historian's job \emph{is} to provide such maximal theories and say `these are the senses we ought to use here.' In which case, we might at least want \emph{transparent} theories, eg ones which flag the senses in use, give us reasons to accept their proffered senses, to reject competing ones, or at least which note competing senses exist. Here we note we do not seem to have very many transparent theories in literature either.}
		%
%
%
%
% sec theories
\section{Models}
\label{s:models}
	In this section, we xx
	%
	% subsec existing
	\subsection{Spatial}
	\label{ss:spatial}
	%
	% subsec existing
	\subsection{Temporal}
	\label{ss:temporal}
%
%
%
% sec theories
\section{Theories}
\label{s:theories}
	In this section, we xx
	%
	% subsec existing
	\subsection{Existing}
	\label{ss:existing}
	%
	% subsec existing
	\subsection{Possible}
	\label{ss:possible}
%
%
%
% sec concl
\section{Conclusion}
\label{s:concl}
	xx
%
%
%
% end stuff
%
% s references
\begin{bibdiv}
\label{s:references}
	\begin{biblist}
	\bibselect{pprefs}
	\end{biblist}
\end{bibdiv}
%
%
%
\end{document}