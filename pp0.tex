\documentclass{amsart}
%
%
% packages
\usepackage{pgfplots} % for plots - loads tikz, wc itself loads xcolor, graphicx
\usepackage{tikz-cd} % for graphs - loads tikz too
\usepackage[figuresright]{rotating} % to rotate table w LHS at bottom
\usepackage{hyperref} % for links - load last-ish
\usepackage{amsrefs} % for full cites - load last
%
%
% new commands to avoid text formatting inline
\newcommand{\code}[1]{\texttt{#1}} % for code
\newcommand{\literature}[1]{\textit{#1}} % for book titles etc
\newcommand{\ship}[1]{\textit{#1}} % for book titles etc
\newcommand{\mention}[1]{\textit{#1}} % to mention (not use) terms
\newcommand{\foreign}[1]{\textit{#1}} % for foreign words
%
%
% new commands to avoid maths formatting inline
\DeclareMathOperator{\ab}{Ab} % place variable
\DeclareMathOperator{\cd}{Cd} % ^
\DeclareMathOperator{\id}{Id} % place name
\DeclareMathOperator{\mt}{Mt} % ^
\DeclareMathOperator{\bc}{Bc} % ^
\DeclareMathOperator{\et}{Et} % ^
\DeclareMathOperator{\nt}{Nt} % ^
\DeclareMathOperator{\pre}{pre} % time period name
\DeclareMathOperator{\pst}{pst} % ^
\DeclareMathOperator{\cpd}{cpd} % ^
%
%
% amsthm styles where mainclm > minrclm = gloss > note = bodytext > footnote
% plain style
\newtheorem{mainclm}{Main Claim} % for main claims
\newtheorem{minrclm}{Minor Claim} % for minor claims
% definition style
\theoremstyle{definition}
\newtheorem{gloss}{Gloss} % for glosses
% remark style
\theoremstyle{remark}
\newtheorem*{note}{Note} % for notes
%
%
% amsthm style for proof
\newenvironment{steps}{\begin{proof}[Steps]}{\end{proof}} % for steps
\renewcommand{\qedsymbol}{\textit{End}.} % for all proof env
%
%
% tikz ie pgsplots, tikzcd commands
\pgfplotsset{compat=1.18}
\usepgfplotslibrary{groupplots} % to group plots
\pgfmathdeclarefunction{gauss}{2}{\pgfmathparse{1/(#2*sqrt(2*pi))*exp(-((x-#1)^2)/(2*#2^2))}} % to use in \addplot
%\tikzcdset{} % global graphs style
%
%
% xcolor commands for darkmode - just for draft
%\pagecolor{darkgray}
%\color{white}
%
%
% hyperref commands
\hypersetup{pdfauthor={A Navidad},pdftitle={pp0},pdfsubject={},pdfkeywords={}} % pdf metadata
%
%
\begin{document}
%
%
% starting stuff
\title{pp0}
\author{A Navidad}
\address{Harvard College, Cambridge MA}
\curraddr{Benque Viejo CY}
\email{navidad@college.harvard.edu}
\date{12 Feb 2025}
\thanks{} % incl non-grant support per ams
\begin{abstract}
\end{abstract}
\keywords{}
\maketitle
%
%
%
% s claim
\section{Claim}
\label{s:claim}
	We present this paper's main claim here, and map out the support it will be given later on. Let \mention{question*} be `What are Belize's origins?' And let \mention{stories*} be answers to question*.\footnote{Existing or possible, subject to some suitability criteria, eg are historical, are not flimsy, so on.} Then,
	%
	% main claim
	\begin{mainclm}
	\label{clm:maindecency}
	\ref{eqn:namedmap} and \ref{eqn:namedlog} are a decent way of building a story*.\footnote{Qualified, eg up to similarity. Tools used jointly only, plus historical data.}
	\end{mainclm}
	%
	\begin{equation}
	\tag{Map}
	\label{eqn:namedmap}
		\begin{tikzcd}[arrows=dash]
		\nt &\bc \rar \ar[rr,bend left] &\mt \rar &\id &\et
		\end{tikzcd}
	\end{equation}
	%
	For vertices as certain places in the Bay, and edges as a certain relation between them. The Bay is mapped from northwest to southeast, beginning in northern Yucatan (\(\nt\)), flowing down to northern Belize (\(\bc\)), then the mouth of the Bay (\(\mt\)), next the Bay Islands (\(\id\)), and finally eastern Honduras (\(\et\)). We say a place \(\ab\) is connected to place \(\cd\) (\(\ab-\cd\)) if \(\ab\) is \mention{close} to \(\cd\), and not otherwise.\footnote{Ignoring trivial cases eg where \(\ab=\cd\).} Basically, spaces are close if they are geographically close.
	%
	\begin{equation}
	\tag{Log}
	\label{eqn:namedlog}
		\begin{tikzcd}[sep=huge]
		t_0 \rar["\pre" description] &t_c \rar["\cpd" description] &t_d \rar[bend left,"\pst_0" description] \rar[bend right,"\pst_1" description] &t_1
		\end{tikzcd}
	\end{equation}
	%
	For vertices as certain points in time, and arcs as certain series of events from one point to the other. A story* is logged from start to finish, beginning at the earliest point (\(t_0\)), then a certain \mention{coupling} point (\(t_c\)), next an ensuing \mention{de}coupling point (\(t_d\)), and ending at the latest point in the story* (\(t_1\)). These four points split a story* into three parts, with a coupl\mention{ed} period (\(\cpd\)) in the middle, and a preceding \mention{pre}coupled period (\(\pre\)). The succeeding \mention{post}coupled period (\(\pst\)) is plotted as two series of events (\(\pst_0\) and \(\pst_1\)) rather than one, to account for two possibly distinct sequences of events. Basically, the coupling refers to the coincidence of piracy and logging in the Bay.
	
	%
	% paper flow
	We bolster Main Claim~\ref{clm:maindecency} by spelling out what makes for a decent way of building a story*, and then demonstrating that \ref{eqn:namedmap} and \ref{eqn:namedlog} are decent. So we first gloss \mention{decency} here.
	%
	\begin{gloss}
	\label{gloss:decency}
		A way of building a story* is decent if the story* so built does \emph{not} answer any relevant \mention{thick} question and \emph{does} at least one relevant \mention{thin} one, where relevancy is to the question*, and thick questions are interpretation-heavy and thin ones less so.\footnote{The latter possibly in such a way as to only partially help answer at least one thick question. Might want further features too, eg is new, offers granular stories*, so on.}
	\end{gloss}
	%
	The general idea being that a maximally useful way of building a story* ought to be open to iterpretation.
	%
	\begin{minrclm}
	\label{clm:minorquestions}
	Relevant thick questions are only `When was Belize settled?' and `When was Belize founded?' Relevant thin ones are at least `When was Belize used for piracy?' and `When was Belize used for logging?'\footnote{Again for relevance to the question*. Up to similarity. Taken as granted, though support for Main Claim~\ref{clm:maindecency} hinges on the first part here.}
	\end{minrclm}
	%
	In Section~\ref{s:support}, we first build \ref{eqn:namedmap} and \ref{eqn:namedlog}, then show they jointly answer neither thick question, and lastly show they do answer at least one thin one.
%
%
%
% s support
\section{Support}
\label{s:support}
	%
	% ss:build
	\subsection{Build}
	\label{ss:build}
		We build \ref{eqn:namedmap} and \ref{eqn:namedlog} from the ground up here.
		
		We start by noting that almost all existing stories* begin with piracy and end with logging, and noting that there is almost no consensus on the path charted from one to the other, nor on the time it took to traverse said path. Nonetheless, we now have thematic start and end points, call them \mention{first-theme} (piracy) and \mention{last-theme} (logging).
		
		Now we note that, while existing stories* are agreed on first-theme, there is yet no agreement on when nor where to begin, with values for the former ranging over three centuries, and for the latter spanning two landmasses.\footnote{Sixteenth to eighteenth cneturies, and Central America including Yucatan, plus the Greater Antilles.} But despite not having them, call them \mention{first-time} (when story* begins) and \mention{first-place} (where story* begins), and similarly for \mention{last-time} and \mention{last-place}.\footnote{These spatial and temporal points are for stories*, \emph{not} for the start/end of our themes ie phenomena ie piracy and logging.}
		%
		% sss:map
		\subsubsection{Map}
		\label{sss:map}
			Now, to get first-places from existing stories*, we either stop and consider each proferred first-place individually, or else sort them into classes and consider each class instead. We do the latter here.
		
			First, group all first-places located within the Bay, and group all others.\footnote{Discard null and `Bay' first-places.} We now have the Bay Islands and Belize on the one hand, and Campeachy, Catoche, Jamaica, Mosquito Shore, Providence and Tortuga on the other.\footnote{Possibly ignoring minor/unreviewed first-places. For Belize, mostly north-central. For Campeachy, including Terminos and Tris. For Catoche, including Cozumel and Ascension, possibly. For Tortuga, including north-west Hispaniola. For Mosquito Shore, assuming/mostly south-central. Providence is \emph{Old} Providence.}
		
			Before proceeding, we note that these first-places look, of course, like a list of pirate haunts in the greater vicinity of Belize, just as we would expect from first-theme. To have a comprehensive \ref{eqn:namedmap}, then, we would want a \emph{complete} list of pirate haunts in said vicinity.
		
			Now, before checking that we have a complete list of pirate haunts, we here note that nautical entry to the Bay is either through the north or the east, such that for first-places outside the Bay, regardless of their geographic location, the relevant agents would all enter the Bay through one of two points.\footnote{Counting north-east entrance via Caymans as east.} In which case, we might map the north (\(\nt\)) and east (\(\et\)) entrances, and take these as proxies for \emph{all} pirate haunts outside the Bay. So, \(\nt\) and \(\et\) give us a complete list of pirate haunts for the second group.
		
			We have no such shortcut for the first group of first-places (pirate haunts in the Bay) though, so this check is done by hand. This done, we find we might want to add one first-place to this group, namely, the mouth of the Bay (\(\mt\)), giving us three first-places (with the Bay Islands (\(\id\)) and Belize (\(\bc\))). We assume our hand check was thorough, and so conclude that \(\mt\), \(\id\), and \(\bc\) give us a complete list of pirate haunts for the first group.\footnote{This step to be supported later on ie that hand check was thorough, and that \(\mt\) was a pirate haunt (at least up to criteria used to say \(\bc\) was one.}
		
			So, we now have a list of five first-places which name all pirate haunts in the vicinity of Belize. As \ref{eqn:namedmap} charts only these, it too is complete in this respect.
		
			Lastly, as we would like to chart our first-places in some relevant manner, and not just list them, we look for relevant ways these pirate haunts were related. This is done by hand. This done, the first and most obvious relation is distance, of course. But in addition to this, we have cost, ie the difficulty of getting from one first-place to the other. And we find no further salient relations.\footnote{Ignoring non-salient/missed relations, but assume no others, or assume these are subsumed by cost. May want to exclude from or keep constant in cost the nautical knowledge/practice/expereience weight. Thoroughness of hand check supported later.}
			
			As cost subsumes distance, we consider only cost. As those were the only salient relations among our first-places, cost gives us \emph{all} the ways our first-places are related.\footnote{All ways we care about, assuming hand check was thorough.} So say some first-place is \mention{close} to another if it's \emph{not} costly to get from the one to the other and vice versa.\footnote{Giving us undirected closeness from directed cost, for simplicity, but could keep directed. By \mention{not costly} we mean for some cutoff of cost which we deem sufficiently cheap.} Then, we have that \(\bc\), \(\mt\), and \(\id\) are close to each other, while \(\nt\) and \(\et\) are close to none.\footnote{Ignoring self-closeness.}
			
			So, we now have one relation which subsumes all salient or relevant ways in which our first-places are related. As \ref{eqn:namedmap} charts only this relation, it too is complete in this respect.
			
			This fully gives us \ref{eqn:namedmap}, so our work here is done.
		%
		% sss:log
		\subsubsection{Log}
		\label{sss:log}
			We start by noting that in existing stories* we have very many first-times and last-times (when stories* start and end). We further have two relevant phenomena, ie our first-theme and last-theme ie piracy and logging. Call their temporal start and end \mention{first-piracy}, \mention{last-piracy}, \mention{first-logging}, and \mention{last-logging}.
			
			Now, we note that none of these first-times pre-dates first-piracy, and none of these last-times post-dates last-piracy. So, first-theme is present at every point in time in existing stories*. We further note that \emph{only} some of these first-times pre-date first-logging, and none of these last-times post-dates last-logging. So last-theme is either present at every point, or else present only in later points in time in existing stories*.
			
			So, of the temporal points of first-theme and last-theme, \emph{none} show up in \emph{all} existing stories*.
			
			We salvage this sad state by looking at the stories* which \emph{do} include first-logging. In almost all of them, this point is not given prominence but rather minimised in favour of a second point, that when the incidence of last-theme reached some significant threshhold, call it \mention{first-significant-logging}.\footnote{There is a further \mention{first-significant-piracy} in some existing stories*, but not needed here.}
			
			If we now look for first-significant-logging in existing stories*, we do find it in all of them.
			
			Instead of plotting first-significant-logging, however, we take stock of \emph{why} first-logging is only in some stories*, and why first-significant-logging is favoured. The immediate answer is that we can't see the incidence of last-theme, of course.\footnote{No records on logging.} But additionally, we further presume that it was \emph{in}significant for some time after first-logging, and that it first reached significance only after first-significant-logging.\footnote{Significance regarding frequency of logging, or amount logged, or number of loggers, and so on.}
			
			So, all existing stories* presume that there was some non-null time between first-logging and first-significant-logging. Again, instead of plotting, we take stock of why this is.
			
			Here, we find that all such stories* presume that first-theme \emph{lead to} last-theme in a \emph{specific} way, namely, by pirates' logging a little bit from first-logging, then more and more, until logging a lot from first-significant-logging.\footnote{For some significance threshhold.} Which is to say, all existing stories* presume the increasing \emph{co}incidence of our first-theme and last-theme.
			
			This, finally, would make for a useful organising principle for \ref{eqn:namedlog}, due not just to existing consensus, but additionally to the greater insight we have into the incidence of first-theme.
			
			Now, we have \emph{strong} significant positive coincidence between first-theme and last-theme starting from first-significant-logging. But even \emph{weak} significant positive coincidence might be useful when using first-theme incidence as proxy for last-theme incidence.\footnote{For some value of storng or weak.}
			
			So, call the point in time when first-theme and last-theme first positively, significantly coincided \(t_c\), and add to \ref{eqn:namedlog}.
			
			Naturally, in the period from start-logging to \(t_c\), first-theme and last-theme did not significantly coincide at all.\footnote{No examples of significant negative coincidence in stories*.} For stories* with first-times at or after first-logging, this covers the entire period. For stories* with first-times pre-dating first-logging, there would still be no significant coincidence at all, given no incidence of last-theme. So call this period of insignifcant coincidence \(\pre\), and call first-times \(t_0\), and add both to \ref{eqn:namedlog}.
			
			Now, in the period after \(t_c\), first-theme and last-theme did significantly coincide at all points, but in all stories* this is only so up to a point.\footnote{Due to various reasons including fall of last-theme incidence below significance, fall of first-theme incidence below significance, so on.} This is, of course, what we would expect, given that last-theme outlived first-theme by a lot.
			
			So, at some point after \(t_c\), first-theme and last-theme first \emph{stopped} positively or significantly coinciding. Call it \(t_d\) and add to \ref{eqn:namedlog}. Furthermore, call the period between \(t_c\) and this point \(\cpd\), and likewise add to \ref{eqn:namedlog}.\footnote{Only one possible path between these points, ie significant coincidence all the time.}
			
			After \(t_d\), all stories* agree that the incidence of first-theme fell below some significance threshhold, while only some say as much of last-theme. Nonetheless, the first consensus is all we need to see all possible paths to all last-times.
			
			So, from \(t_d\) to last-time, the incidence of last-theme either was significant at all points, or it was not, while the coincidence of first-theme and last-theme was either insignificant at all points, or it was not.
			%
			% table
			\begin{table}
			\caption{Paths from \(t_d\) to last-time.}
			\label{tab:pathsforlog}
			\begin{tabular}{cccc}
			first-theme &last-theme &coinc &coinc direcn\\
			insig	&sig	&insig	&irlvt\\
			insig	&sig	&sig		&+ or --\\
			insig	&insig	&insig	&irlvt\\
			insig	&insig	&sig		&+ or --
			\end{tabular}
			\end{table}
			%
			
			The paths possible from \(t_d\) to last-time are listed in Table~\ref{tab:pathsforlog}. We note that coincidence is insignificant at \emph{all} points in time in the first and third, while logging is significant at \emph{all} points in first and second.
			
			For simplicity, we merge the first and third paths and call them \(\pst_0\), and merge the second and fourth call them \(\pst_1\), and call last-times \(t_1\), and add all to Table~\ref{tab:pathsforlog}. 
			
			This fully gives us \ref{eqn:namedlog}, so our work here is done.
	%
	% ss:decency
	\subsection{Decency}
	\label{ss:decency}
		We show \ref{eqn:namedmap} and \ref{eqn:namedlog} as built are decent here. We first note that \ref{eqn:namedlog} requires at least all data on the incidence of piracy in the Bay, so say we have that. We must then (somehow) identify \(t_c\) and \(t_d\) in these data, and get a picture direct picture of piracy, and proxy picture of logging, in the Bay from \(t_0\) to \(t_1\).\footnote{\(t_c\) and \(t_d\) are the only defined temporal points in \ref{eqn:namedlog}.} This obviously falls way short of answering either thick question, and yet directly answers the first listed thin one, while at least partially answering the second thin question.\footnote{Assuming good data for the thin ones.}
		
		So, \ref{eqn:namedmap} and \ref{eqn:namedlog} are decent.
	%
	% ss:ends
	\subsection{Ends}
	\label{ss:ends}
		We finish work left for later in Section~\ref{ss:build} here.
		%
		% sss:onmt
		\subsubsection{On \(\mt\)}
		\label{sss:onmt}
		For thoroughness of first hand check in Section~\ref{sss:map}, we note the literature and sources reviewed, and those missed.\footnote{List here.} For the inclusion of \(\mt\) as a pirate haunt, we note only the greater incidence of piracy near it than near \(\bc\). As the latter is deemed a haunt, so ought the former.
		%
		% sss:onmt
		\subsubsection{On closeness}
		\label{sss:oncloseness}
		For the second hand check in Section~\ref{sss:map}, we again note literature reviewed and missed.\footnote{List here.}
		%
		% sss:onstories
		\subsubsection{On stories*}
		\label{sss:onstories}
		For Section~\ref{sss:log}, read \mention{existing} stories* as those reviewed in literature noted here.\footnote{List here.}
%
%
%
%
% s use
\section{Use}
\label{s:use}
	We use \ref{eqn:namedmap} and \ref{eqn:namedlog} here, first with help from only particular existing stories*, then in a generalised manner. For the particular stories*, we use those with the earliest and latest answers to question*, ie Campbell and Restall, plus a few in between, ie the Fénix column and Read's letter and Bulmer-Thomases paper. \footnote{\citelist{\cite{cam11} \cite{cam03} \cite{res19} \cite{rds32} \cite{fen49} \cite{bul16}}. A non-exhaustive list of stories* to consider.}
	%
	% ss:particular
	\subsection{Particular}
	\label{ss:particular}
		Qualify all claims regarding stories* here as being \emph{for} a given reading. We contend only that the given readings are \emph{possible}.
		%
		% sss:bulmers
		\subsubsection{Bulmer-Thomases}
		\label{sss:bulmers}
		The Bulmer-Thomases offer a haven-camp story* set spatially in \(\bc\), \(\id\), and \(\et\) only.\footnote{Buccaneers possibly use as haven in 1642--1670 and turn to logging in 1670s \cite{bul16}*{pp 137--138, 145, 151--156}.} They explicitly set \(t_0\) in 1642 and \(t_d\) in 1670s and \(\pst\) via first path in \(\pst_0\). They do not explicitly set \(t_1\), but we read \emph{prior} to 1705 for it, and further set \(t_c\) as mid 17th century from explicit values.\footnote{Implicit in \cite{bul16}*{p 149}.} This completes \ref{eqn:namedlog}.
		%
		% sss:campbell
		\subsubsection{Campbell}
		\label{sss:campbell}
		Campbell offers a haven-camp-settlement story* set spatially in \(\bc\) only.\footnote{Loose pirate communities from probably mid 16th cent \cite{cam11}*{pp 95--96, 100 \foreign{pass}}, pirate haven on St George's and surrounding cayes from probably mid 1550s \cite{cam03}*{pp 176, 179}, buccaneer haven from 17th cent \cite{cam03}*{p 174}, buccaneer haven definitely from 1670s \citelist{\cite{cam11}*{pp 82--83} \cite{cam03}*{p 176, 178}}, partial Providencer presence from 1641 \cite{cam11}*{p 129}, first important settlement on St George's definitely from 1765 \citelist{\cite{cam11}*{pp 121--122, 129} \cite{cam03}*{pp 175, 179--180}}, and non-buccaneering ie sedentary ie logging settlement from mid 17th cent \cite{cam03}*{p 171}. The Providencer answer possibly a misreading.} They explicitly set \(t_0\) in mid 1550s, \(t_d\) in mid 17th century, and \(\pst\) via first path in \(\pst_0\) ie first in Table~\ref{tab:pathsforlog}.\footnote{Explicit in \cite{cam03}*{p 171}. Also implicit in \cite{cam11}*{pp 91--92, 104, 108}. But Campbell does not mean \emph{only} \foreign{H campech} by \mention{logwood} \cite{cam11}*{pp 104--105}, whereas we do, so a bit tricky to read.} % p91-92 = by 1708, p104 = by 1680s, p108 = by early 1670s
		They do \emph{not} quite so explicitly set \(t_c\) nor \(t_1\), but we read early 17th and early 18th centuries for them.\footnote{That is, \(t_c\) implicitly set to Elizabethan era in \cite{cam11}*{pp 106--107}, and to prior to 1695 and 1697 and 1655 in \cite{cam11}*{pp 84, 91, 106}, respectively for latter. And \(t_1\) implicitly to 1708 in \cite{cam11}*{pp vii, 93}, and possibly more. Again, for \mention{logwood} meaning \mention{dyewood}.} This completes \ref{eqn:namedlog}.
		% For Campbell, if our reading is right, then we get coincident piracy and logging from early to mid 17th century ie during \(\cpd\), with significant logging starting at some point therein.\footnote{And attendant significant piracy, given coincidence. Read \emph{significant} \emph{positive} coincidence for coincidence \foreign{simpliciter} hereon.} And as after \(t_d\) we take the first \(\pst_0\), we further get significant logging at all points to early 18th century.\footnote{Modulo interruptions, which Campbell tolerates.}
		%
		% sss:fenix
		\subsubsection{Fénix}
		\label{sss:fenix} Sierra O'Reilly offers a haven-settlement story* set spatially in \(\bc\) only, and temporally in mid 17th century for \(t_1\) only.\footnote{Pirates settle on Old from mid 17th cent \cite{fen49}*{p 3}. This is a Wallace story*. \citelist{\cite{cam09}*{pp 72--110} \cite{bul16}*{pp 137--151} \cite{res19}*{pp 19--24}} have thoroughly shown that \mention{Wallace} in such stories* is most likely apocryphal. This, however, is also a Meridian story*, that is, one from Merida built on or after the 18th cent, when that city was heavily focussed on the Bay. (\citelist{\cite{car71}*{pp 55, 210} \cite{car78}*{pp 260--261} \cite{anc78}*{pp 370--376} \cite{pen69}*{pp 217--219}} are also Meridian stories*.) By the former, these stories* are at least partly apocryphal. By the latter, these stories* are at least possibly better-sourced than some others, given the non-zero chance of unique primary sources in Merida in and after the 18th cent. So, we ignore most Wallace details here, and read \mention{one of the first Baymen} for \mention{Wallace}. Note this means we disregard the move by \cite{bul16}*{pp 137--138, 145, 151--156} from \mention{Wallace}'s being apocryphal to temporal restrictions on acceptable answers to question*, and further disregard charges of dishonesty in Wallace stories* in \citelist{\cite{bul16}*{pp 138, 140--141} \cite{cam09}*{pp 87--88, 105--106} \cite{res19}*{pp 23--24}}, as the former follows and the latter regards only \mention{Wallace} properly (not symbolically) read.}
		We find nothing further, so do not complete \ref{eqn:namedlog}.
		%
		% sss:reads
		\subsubsection{Read's}
		\label{sss:reads} This Bayman offers a camp-settlement story* set spatially in \(\bc\) only.\footnote{Buccaneer-loggers camp-settle on Old then camp-settle up to Hondo in mid 17th cent \cite{rds32}*{p 3}. This is one of the earliest stories* available in print (at least), and one of the only to ascribe to later Bacalar strikes a reason over piracy (possibly). The unsigned letter, dated Campeachy, 24 Nov 1731, reads in part:\begin{quote}As to the State of the Bay of Honduras, I shall give it you as briefly as possible. The ancient City of Bacalar, situate in that Part of the Province of Yucatan, which lies on the Bay of Honduras, was twice sack'd, and at last totally ruined by the English many Years ago; on which the Logwood-Cutters of that Nation, who had settled on the River of Valis, possessed themselves of the New River and that of the Hondo; which last is distant from the Ruins of Bacalar about five Leagus. Here they built a great many Houses and Hutts, and employ'd Multitudes of Negroes in cutting Logwood, which was transported to Jamaica and Europe by Numbers of Vessels trading from thence to the Bay.\end{quote} \cite{res19}*{pp 13, 16} deems a similar contemporaneous story* `a politically motivated rhetorical exaggeration' (ie the 1734 Pitt story* that logging camps on Old `had been possessed by the English for more than a hundred years,' in AGI Mexico 3099 ff 5--10). We partially disagree on this.} % agi not digitised http://pares.mcu.es/ParesBusquedas20/catalogo/description/374685
		They explicitly set \(\pst\) as the first path in \(\pst_0\). They do not explicitly set \(t_1\), but we read \emph{prior} to mid 17th century for it. Nothing further found, so no complete \ref{eqn:namedlog}.
		%
		% sss:restall
		\subsubsection{Restall}
		\label{sss:restall}
		Restall offers a camp-settlement story* set spatially in \(\bc\) and \(\nt\) only, and temporally in 1662 for \(t_0\) and 1717 for \(t_1\) only.\footnote{Haphazard haven-camp possibly from 1662, and settled logging camp definitely from 1716 \cite{res19}*{pp 2--3, 12--13, 17, 20--21, 30}.} Nothing further found, so no complete \ref{eqn:namedlog}.
	%
	% ss:noteson
	\subsection{Notes}
	\label{ss:noteson}
		We make a number of observations here regarding the preceding stories*.
		%
		% sss:onnolog
		\subsubsection{On \ref{eqn:namedlog} completion}
		\label{sss:onnolog}
		The first thing we note from reviewed stories* above is our having three \ref{eqn:namedlog} not complete.
		
		Now, we read all those stories* as presuming piracy lead to logging.\footnote{And as stories* not missing parts, which might not be the case for Read's in Section~\ref{ss:particular}, but we ignore that here.} This still seems correct on second thought. But per Section~\ref{sss:log}, we expect complete \ref{eqn:namedlog} from such stories*. So we probably erred somewhere in Section~\ref{sss:log}. We now review the steops in that section to find the error(s).
		%
		\begin{steps}[In Section~\ref{sss:log}]
		We built \ref{eqn:namedlog} (up to \(t_c\)) like so:
		\begin{enumerate}
		\item First-logging is \emph{not} in all stories*.\footnote{Or is minimised.}
		\item First-significant-logging \emph{is} in all stories*.\footnote{Or is not similarly minimised.}
		\item So, all stories* presume insignificant logging to first-significant-logging.
		\item So, all stories* presume piracy lead to logging in a specific way.\footnote{By pirates logging a little bit from first-logging, then more and more, until logging a lot from first-significant-logging.}
		\item So, all stories* presume increasing coincidence of piracy and logging.
		\item So, for \(t_c\) the point when piracy and logging first positively significantly coincided, \(t_c\) is in all stories.
		\qedhere % apparantly needed when ending in list
		\end{enumerate}
		\end{steps}
		%
		We now find (1) still holds for all stories* in Section~\ref{ss:particular}, and find (2) holds for all but Fénix. Now, this is a temporally compressed land-and-settle story*, while the rest are drawn out squat-squat-settle ones.\footnote{Distinction first from \cite{cam11}*{pp 95--96}. The associated distinction (intentional vs accidental) is not imported here.} We ignore stories of the former ilk hereon.
		
		So both (1) and (2) hold for all our stories*. We further find (3) holds too, but the move to (4) is tricky, so we now spell it out more fully.
		
		First, let \mention{piracy-win*} be \mention{successful piracy}, and \mention{piracy-fail*} be \mention{failed piracy}.\footnote{For simple or minimal \mention{success} eg where pirates surprised a ship or sacked a port, even if they got no/nil booty. And similar \mention{failure} eg where they found no ships nor came to port.} And let \mention{piracy*} be either.
		
		That done, we now review step (4). We find that swapping in \mention{piracy-win*} for \mention{piracy} does not work, whereas \mention{piracy-fail*} \emph{does}. That is, for the particular way of \mention{leading to} spelt out in (4), we want to say that all stories* presume piracy-fail* lead to logging, \emph{not} that piracy-win* did so.
		
		Now let (4a*) be `All stories* presume piracy-fail* lead to logging in \emph{some} way.' Then we find (4a*) holds for all our stories*, independently of (3), and ignore stories* where it doesn't hereon.\footnote{Eg where successful pirates discovered logging then afterwards intentionally/en-masse set out only to log ie sudden conversion. This might be the case eg in stories* where social/legal pressure forced successful pirates en masse to logging, but maybe only if pressures suddenly/rapidly not gradually materialise. So, we ignore stories* where pirates suddenly/en masse dropped piracy for logging.}
		
		At this point, it might be helpful to see how (the incidence of) piracy-win* and piracy-fail* might look like plotted together. We first imagine a bell curve for piracy*.\footnote{Even if incidence of piracy* is distributed multimodally, the following still works.} Then, under the piracy* curve, piracy-win* and piracy-fail* look like either 1 or 2 in Figure~\ref{fig:yaynayinterference}.\footnote{Figure~\ref{fig:yaynayinterference} presumes pirates were more successful than not (more piracy-win* than piracy-fail* incidents), but even if otherwise, the following still works. Additionally, cases where piracy-fail* peak precedes piracy-win* peak are not shown, but their curves are just 1 and 2 in Figure~\ref{fig:yaynayinterference} flipped, so following still works.} In the first case, we say they \emph{closely} tracked or \emph{positively} coincided, and say they loosely tracked or negatively coincided in the second case.
		% small fig 
		% fig interference
\begin{figure}
% pgsplots code begins
\begin{tikzpicture}[baseline] % baseline per pgsplots man
	\pgfmathdeclarefunction{gauss}{2}{\pgfmathparse{1/(#2*sqrt(2*pi))*exp(-((x-#1)^2)/(2*#2^2))}} % to use in \addplot
	\begin{axis}[
		small,
		xlabel=1, ylabel=, % labels
		xtick=\empty, ytick=\empty, % ticks
		]
	\addplot [
      red,
		domain=0:10,
		samples=100,
      ] {gauss(4.5, 1.5)};
	\addplot [
      blue,
      domain=0:10,
      samples=100,
      ] {gauss(3, 1)};
	\end{axis}
\end{tikzpicture}% \% here to avoid whitespace
\hskip 10pt % per pgsplot man
\begin{tikzpicture}[baseline] % baseline per pgsplots man
	\pgfmathdeclarefunction{gauss}{2}{\pgfmathparse{1/(#2*sqrt(2*pi))*exp(-((x-#1)^2)/(2*#2^2))}} % to use in \addplot
	\begin{axis}[
		small,
		yticklabel pos=upper,
		xlabel=2, ylabel=, % labels
		xtick=\empty, ytick=\empty, % ticks
		]
	\addplot [
      red,
		domain=0:10,
		samples=100,
		] {gauss(6, 1.5)};
	\addplot [
      blue,
      domain=0:10,
      samples=100,
      ] {gauss(3, 1)};
	\end{axis}
\end{tikzpicture}
% pgsplots code ends
\caption{Constructive vs destructive interference/close vs loose tracking/positive vs negative coincidence of piracy-yay* (blue) and piracy-nay* (red).}
\label{fig:yaynayinterference}
\end{figure}
%
 % \label{fig:yaynayinterference}
		
		Note, in Section~\ref{sss:log} we said we had a clearer picture of piracy than logging. While this still holds, we now say we have a clearer picture of piracy-win* than logging, and further a clearer picture of piracy-win* than of piracy-fail*.\footnote{Or possibly no picture of piracy-fail* at all, unless by proxy from piracy-win*. We further note our relative lack of non-Spanish records for piracy* in the Bay, and further, their tendency to label any foreigner \mention{pirate} (inflating piracy-win* incidence), and the (we think likely) chance of their being so overwhelmed or desentised over time to piracy* so as to stop recording/reporting individual piracy-win* incidents (especially minor ones, thereby deflating incidence).}
		
		Moving on though, let (4b*) be `All stories* presume piracy-fail* lead to logging in a \emph{particular} way, namely, via increasingly-significant (positive) coincidence.'\footnote{Logging a little bit from first-logging, then more and more, until logging a lot from first-significant-logging, with all of this occurring almost only during piracy-fail* incidents. That is, logging and piracy-fail* coincide only a few times from first-logging, then more and more, until coinciding very many/almost at all times from first-significant-logging.} Then, the move from (4a*) to (4b*) is natural or likely, and so made.\footnote{Though (4b*) does not follow from (4a*), but cannot see how else (4a*) might obtain. May explore later.}
		
		This, of course, immediately gets us to (5*) and (6*) (steps (5) and (6) with \mention{piracy-fail*} swapped in for \mention{piracy}). But we now have a not-so-useful \ref{eqn:namedlog}.
		
		At this point, it might be useful to get a clearer view of 2 in Figure~\ref{fig:yaynayinterference}. The possible orders of peaks are sketched in Figure~\ref{fig:yaynayorder}.\footnote{Also possible for 1 in Figure~\ref{fig:yaynayinterference}, with closer peaks, but irrelevant here.} That is, if piracy-win* and piracy-fail* only loosely track each other, rather than closely, then periods of heightened piracy-fail* incidence might precede, or succeed, or both precede and succeed periods of heightened piracy-win* incidence.\footnote{Probably 2 in Figure~\ref{fig:yaynayorder} is likelier than the others eg pirates being successful lead to more pirates trying (in a delayed manner).}
		% small fig 
		% fig order of peaks
\begin{figure}
% pgsplots code begins
\begin{tikzpicture}
	\begin{groupplot}[
		group style={
			group size=2 by 2,
			every plot/.style={domain=0:10,samples=100,},
			},
		]
		\nextgroupplot [small,xlabel=1, ylabel=,xtick=\empty,ytick=\empty]
			\addplot [red,] {gauss(2, 1.5)};
			\addplot [blue,]{gauss(5, 1)};
		\nextgroupplot [small,xlabel=2, ylabel=,xtick=\empty,ytick=\empty]
			\addplot [red,] {gauss(8, 1.5)};
			\addplot [blue,]{gauss(5, 1)};
		\nextgroupplot [small,xlabel=3, ylabel=,xtick=\empty,ytick=\empty]
			\addplot [red,] {gauss(2, 1.5)};
			\addplot [red,] {gauss(8, 1.5)};
			\addplot [blue,]{gauss(5, 1)};
	\end{groupplot}
\end{tikzpicture}
% pgsplots code ends
\caption{Possible order of peaks of piracy-win* (blue) and piracy-fail* (red) when they negatively coincide with/loosely track each other (as in 2 of Figure~\ref{fig:yaynayinterference}).}
\label{fig:yaynayorder}
\end{figure}
% % \label{fig:yaynayorder}
		
		So, if/when we have 1 in Figure~\ref{fig:yaynayinterference}, piracy-win* data from historical records might serve as a proxy picture for piracy-fail*, regardless of the order of their peaks. However, if/when we have 2 in Figure~\ref{fig:yaynayinterference}, then unless we know which of 1--3 in Figure~\ref{fig:yaynayorder} we have, piracy-win* data would \emph{not} make for good proxy for piracy-fail*.\footnote{With options 1 and 2 in Figure~\ref{fig:yaynayinterference} not exclusive, of course.}
		
		As that's the case, then \ref{eqn:namedlog} might be jointly decent with \ref{eqn:namedmap}, but it's not of much use as regards getting a picture of logging incidence.
		
		But as regards our stories* in Section~\ref{ss:particular}, if the above is correct, then we ought to have gotten at least \(t_c\) (and therefore \(t_c^*\)) in Read's and Restall.\footnote{For \(t_c^*\) as the temporal point when piracy-fail* and logging first positively significantly coincided. Note Fénix was discarded above.} However, we get said point in neither. For Read's, we say the story* is abridged, and so supply a \(t_c^*\) ourselves. For Restall, the story* is \emph{not} abridged, so something else it at play.
		
		We consider three possibilities for Restall. 
		
		First, say some phenomenon \mention{leads} to another \mention{coarsely} if they (as a \emph{whole}) are appropriately related or \mention{tied}, and say it leads to another \mention{finely} if enough of their (\emph{individual}) incidents are appropriately related ie tied. Then one might presume that piracy coarsely lead to logging sans presuming one finely lead to the other, eg via an en masse/swift pirates-to-loggers conversion. For (4a*) we previously discarded all the swift conversion stories*. We now revise that, and say we discard all stories* that do \emph{not} presume piracy \emph{finely} lead to logging. We read Restall as presuming fine leading to, and so keep.
		
		Next, say piracy finely lead to logging, but with most of the tied incidents \emph{outside} the Bay. Now, this might follow simply from higher piracy incidence outside the Bay, or a stronger pirac-logging tie outside the Bay (or both). The second here might make coincidence location-sensitive (eg higher coincidence in Campeachy vs Bay).
		
		The first case (of higher piracy outside the Bay), all else equal, is granted, and its implications disregarded as causing us no toruble.\footnote{In this case, we would have lower logging incidence in Bay than outside, resulting possibly in insignificant logging only, and so in a delayed first-significant-logging. But the move to insignificant logging imports interpretation (how high one's standards for significant logging are, or use of relative vs absolute standards), and so not relevant here.}
		
		The second case (of a stronger tie outside the Bay), all else equal, is \emph{not} granted. Depending on standards used, this might naturally lead to a too-weak-to-count tie in the Bay, and so to only-insignificant piracy-logging coincidence in Bay.
		
		The third case (both of the above) is xx granted.
	%
	% ss:general
	\subsection{General}
	\label{ss:general}
		%
		%
		%
		%
		%
		%
		%
		%
		%
% delete later
\section{Scrubber}
			ABOVE WE assume/conclude-
			- piracy-fail* lead to logging (somehow) in all
			- piracy-fail* lead to logging (via increasing coinc) in all
			- 1/ first-logging not in all
			- 2/ first-significant-logging yes in all but insta-settle stories* eg Fenix
			- discard insta-settle stories
			- 3/ insig logging from first-log to first-sig-log in all stories*
			- clearer view of pir-win* than log and than pir-fail* in hist records
			
			NOT NOTED YET
			- sig pos coinc = 1:1 pir:log incidents 'tied' (perfect coinc), or 2:1 or 1:2 (half coinc)
			- sig neg coinc = 0:1 pir:log OR 0:1 pir:log incidents 'tied' (perfect coinc)
			ie perfect coinc (pos) = every pir incid tied to a log incid AND viceversa
			ie percet coinc (neg) = no pir incid tied to a log incid AND viceversa
			NEED PLOT FOR THIS (eg showing rise of logging vs pir-fail*)
			What of 'whereever we don't see log we see piracy and viceversa'? - this seems pretty impossible
			%% fig interference
\begin{filecontents}{rand1.txt}
x	y	z	w
40	51	79	53
64	96	45	4
6	42	63	92
87	34	25	97
76	45	46	4
76	44	44	75
18	30	43	49
58	77	56	10
10	88	63	100
20	92	91	77
49	64	60	81
31	82	57	49
98	95	34	60
40	21	20	56
73	37	51	88
29	55	44	44
98	50	37	63
31	46	87	61
42	36	81	76
90	69	19	30
90	22	3	22
26	93	60	18
43	32	98	44
22	93	2	26
7	50	44	71
44	46	65	3
26	57	44	27
57	16	75	25
17	67	90	19
32	52	8	5
68	91	50	64
66	88	39	7
12	34	54	88
97	50	39	75
47	52	100	57
23	30	73	100
94	7	26	56
20	98	3	62
98	82	8	16
19	63	91	96
91	38	9	75
84	62	90	9
57	45	73	81
79	50	9	71
62	50	14	95
35	23	32	71
66	33	78	71
3	57	44	5
48	29	56	69
31	60	89	61
3	66	24	18
30	19	61	63
46	46	19	2
27	85	100	96
62	68	45	17
28	39	84	80
41	38	5	51
54	84	51	34
72	48	49	48
48	20	31	59
62	65	99	52
37	98	49	3
78	38	79	14
76	80	32	29
52	64	5	94
69	74	57	46
46	42	26	77
6	7	55	14
29	93	56	100
33	6	97	76
58	48	42	86
39	59	80	22
76	88	13	17
66	57	52	85
83	12	90	8
92	93	34	38
79	20	97	76
65	13	11	95
50	36	92	94
18	14	63	72
84	16	62	36
66	32	28	70
80	64	41	51
39	10	28	22
39	63	73	62
84	4	69	99
60	18	36	82
10	24	1	4
33	76	87	17
62	47	12	100
7	78	10	91
82	54	52	20
54	3	94	58
64	98	61	3
100	100	70	1
33	77	51	46
16	78	58	40
42	85	81	3
97	67	38	8
40	28	69	67
94	13	14	49
83	68	82	96
46	94	4	23
11	23	34	14
56	44	88	66
39	15	61	51
89	42	32	48
63	27	4	28
75	69	30	22
4	95	54	69
65	18	56	37
28	14	93	9
62	7	32	21
94	43	18	11
8	37	50	10
100	69	57	76
61	88	96	65
19	42	72	45
96	73	47	13
93	86	23	56
29	59	46	44
49	3	10	12
37	59	74	45
3	27	64	65
87	1	39	58
50	93	60	80
79	18	11	1
10	61	29	11
99	61	73	7
86	72	92	40
51	11	4	2
61	75	31	32
75	27	67	38
62	11	44	22
47	33	13	42
4	82	21	19
93	37	50	45
81	86	43	4
96	74	7	60
8	23	44	94
82	65	50	91
96	60	91	60
16	11	3	28
73	21	6	23
84	86	82	19
32	7	76	18
78	54	99	63
23	27	26	28
62	80	61	48
28	17	42	99
\end{filecontents}
\begin{figure}
% pgsplots code begins
\tikzset{every mark/.append style={scale=2}}
\begin{tikzpicture}
	\begin{groupplot}[
		group style={
			group size=2 by 2,
			every plot/.style={only marks,},
			},
		]
		\nextgroupplot [small,xlabel=1,ylabel=,xtick=\empty,ytick=\empty,]
			\addplot+ [green,mark=+,] table [x=x,y=y] {rand1.txt};
			\addplot+ [red,mark=x,] table [x=x,y=y] {rand1.txt};
		\nextgroupplot [small,xlabel=2,ylabel=,xtick=\empty,ytick=\empty,]
			\addplot+ [green,mark=+,] table [x=x,y=y] {rand1.txt};
			\addplot+ [red,mark=x,] table [x=y,y=x] {rand1.txt};
		\nextgroupplot [small,xlabel=3,ylabel=,xtick=\empty,ytick=\empty,]
			\addplot+ [green,mark=+,] table [x=x,y=z] {rand1.txt};
			\addplot+ [red,mark=x,] table [x=y,y=w] {rand1.txt};
	\end{groupplot}
\end{tikzpicture}
% pgsplots code ends
\caption{Coincidence scenarios of piracy-fail* (red) and logging (green) across space.}
\label{fig:naylogcoincidence}
\end{figure}
% % \label{fig:naylogcoincidence} - shows only perfect coinc rn - interprets 'tie' as geographic only, which is wrong!
			
			Say an incident is \mention{tied} to another if they are appropriately related. For instance, we might say a particular pitacy incident is tied to a logging one if a pirate crew decides to log (and does or tries to do so) during their off time while on a pirating expedition. Or we might say they are tied if they are merely spatiotemporally close. The a \emph{granular} reading of lead to implies coincidence of individual incidents of two phenomena, while a \emph{coarse} reading only does for the phenomena as a whole eg where we piracy 'lead to' loggign sans coincidence of incidents eg via mass conversion.

			PREV OBSERV NOT NOTED YET
			- need assumptions to hold for Bay not just WI, whereas vars stories* speak of WI
			- pir waxed-and-waned so peaks-and-valleys for pir-win*, and pir-fail*
			- inner Bay triangle barely noted in stories* reviewed
			- have not discussed/plotted: growing resident-pirates population in WI, decreasing Sp commerce in Bay, growing social/legal pressures against piracy in WI
			- underuse of incidence data in stories, or coincidence not presumed eg 'lead to' as in 'discovered Bay, used as pir haven, left for greener pastures, then came back now as loggers'?
%
%
%
% s data
\section{Data}
\label{s:data}
	We present the data this paper uses.
%
%
%
% large fig and tab
% fig piracy 
\begin{filecontents}{ppfigpirdata.dat}
% incidents from /incidents in Hond Bay (noall) or Bz only (nobz) - make sure noall value *excludes* nobz value to stack bars in plot
yr	noall	nobz
1601	2	0
1602	0	0
1603	2	0
1604	2	0
1605	0	0
1606	4	0
1607	5	0
1608	0	0
1609	0	0
1610	1	0
1611	0	0
1612	0	0
1613	1	0
1614	0	0
1615	0	0
1616	0	0
1617	0	1
1618	1	0
1619	0	0
1620	0	0
1621	0	0
1622	0	0
1623	0	0
1624	0	0
1625	0	0
1626	0	0
1627	0	0
1628	0	0
1629	0	0
1630	2	0
1631	1	0
1632	1	0
1633	2	0
1634	1	0
1635	0	0
1636	3	0
1637	2	0
1638	5	1
1639	5	0
1640	6	0
1641	2	2
1642	5	2
1643	4	0
1644	4	0
1645	1	0
1646	1	0
1647	1	0
1648	1	1
1649	0	0
1650	3	0
1651	2	0
1652	0	2
1653	0	0
1654	1	1
1655	0	0
1656	1	0
1657	0	0
1658	0	0
1659	1	0
1660	2	0
1661	0	0
1662	0	0
1663	0	0
1664	0	0
1665	3	0
1666	1	0
1667	2	1
1668	0	0
1669	0	0
1670	1	0
1671	0	0
1672	1	0
1673	0	0
1674	0	0
1675	0	0
1676	1	0
1677	0	1
1678	2	0
1679	1	1
1680	0	2
1681	0	0
1682	0	2
1683	5	0
1684	1	0
1685	2	0
1686	3	0
1687	3	0
1688	4	0
1689	0	0
1690	1	0
1691	0	0
1692	0	0
1693	0	0
1694	0	1
1695	0	0
1696	0	0
1697	0	0
1698	0	0
1699	1	0
1700	0	0
\end{filecontents}
\begin{figure}
% pgsplots code begins
\begin{sideways}
\begin{tikzpicture}
	\begin{axis}[
		xmin=1601, xmax=1700,
		xtick={1610,1620,1630,1640,1650,1660,1670,1680,1690}, % to not display all yrs
		xticklabel style={/pgf/number format/1000 sep=}, % to not add comma
		enlarge x limits=0.02, % to not cut off y bars
		ymin=0, ymax=8,
		ybar stacked, % use y bars and stack
		bar width=0.6, % for thinner bars
		width=1\textheight,height=1\textwidth, % expand sideways fig
		]
	\addplot [ % plot for piracy incidents in bz
		black,fill=black,% colour outline and fill
		] table [x=yr,y=nobz] {ppfigpirdata.dat};
	\addplot [ % plot for piracy incidents outside of bz
		gray,fill=gray,% colour outline and fill
		] table [x=yr,y=noall] {ppfigpirdata.dat}; % check noall value *excludes* nobz value to stack
\end{axis}
\end{tikzpicture}
\end{sideways}
% pgsplots code ends
\caption{Piracy incidents in the Bay in the 17th cent (black in Bz).}
\label{fig:piracyinbay}
\end{figure}
%
 % \label{tab:literature}
% tab literature
\begin{table} % sidewaystable is no go in amsart apparantly
\caption{Stories* in 18th to 21st cent literature.}
\label{tab:literature}
% missg story: s596 / missg cites: s1644,s2320,s2316,rushton2014, prolly vars others
% tab code begins
\begin{sideways}
\begin{tabular}{cp{.35\textheight}lp{.2\textheight}ccc}
No	&Date	&In	&Claim	&Pub&First	&Cites\\
(1)	&mid 16th cent	&Old	&Eng seek haven	&2003	&s2320	&2\\ % campbell - locn:St George's - \href{https://github.com/aenavidad/pp-0/blob/main/literature/s2320.json}{s2320}
(2)	&ca Hond flotilla est	&Old	&Wallace settles	&1925	&s770	&7\\ % asturias - first in live wiki
(3)	&ca 1610	&Old&Wallace settles	&1872	&s2252	&2\\ % egli
(4)	&pre/ca Fuensalida missions	&N	&Brit log	&1957	&s396	&3\\ % roys
(5)	&in 1617	&Old	&Wallace settles	&1925	&s770	&2\\ % asturias
(6)	&pre penult Bacl sacking	&Old	&Eng log	&1732	&s1726	&1\\ % reads 1732
(7)	&ca Prov est	&S &Provers seek refuge or farm	&1946	&s416	&2\\ % winzerling
(8)	&mid 17th cent	&Old	&Wallace settles	&1849	&s1722	&12\\ % fenix 1849
(9)	&pre treaty	&--	&Brit present	&1841	&s2254	&2\\ % mcculloch
(10)	&mid third 17th cent	&Old	&Wallace present	&1878	&s506	&1\\ % ancona
(11)	&in 1638	&Old	&Brit wreck	&1829	&s860	&13\\ % HA 1829
(12)	&in 1638	&Old	&Wallace wrecks	&1827	&s860	&16\\ % HA 1827
(13)	&in 1640	&Old	&Wallace wrecks or settles	&1872	&s2238	&5\\ % ungewitter
(14)	&in Cromwell govt	&--	&Eng seek haven	&1826	&s860	&5\\ % HA 1826
(15)	&post Bacl/Trux sacking or post Camp logging	&Old	&Brit seek haven or log	&2016	&s78	&1\\ % b-thomas
(16)	&last half 17th cent or pre early 1680s	&N	&Eng present	&1994	&s244	&1\\ % finamore
(17)	&post Jam invasion or post Camp logging	&--	&Jamers log	&1777	&s1900	&4\\ % robertson
(18)	&ca 1663	&Old	&Wallace logs	&2010	&s2270	&1\\ % g diaz
(19)	&in 1662	or pre 1670&Old	&Wallace logs	&1944	&s418	&2\\ % c quijano
(20)	&post Catoche/Camp logging	&Old	&Jamers log	&1878	&s506	&2\\ % ancona
(21)	&post treaty	&Old	&Wallace present	&1877	&s510	&1\\ % n ortega
(22)	&last third 17th cent	&Old	&Eng present	&1878	&s506	&1\\ % ancona
(23)	&early 18th cent	&N	&Eng seek haven	&1864	&s1858	&2\\ % brockhaus
(24)	&post Camp logging	&N	&Wallace seeks haven or logs	&1871	&s2256	&4\\ % c ancona - last in live wiki
(25)	&post Camp logging	&Old	&Eng log	&2019	&s1644	&1 % restall
\end{tabular}
\end{sideways}
% tab code ends
\end{table}
%
 % \label{fig:piracyinbay}
%
%
%
% s references
\begin{bibdiv}
\label{s:references}
	\begin{biblist}
	\bibselect{pprefs}
	\end{biblist}
\end{bibdiv}
%
%
%
\end{document}