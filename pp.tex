\documentclass{amsart}
%
%
% packages
%\usepackage[shortlabels]{enumitem}% for lists - prolly no need
\usepackage{lipsum}% for dummy text - just for draft
%\usepackage{xcolor}% for darkmode pdf - just for draft
%\usepackage{graphicx}% for graphics - may not need
\usepackage{pgfplots}% for plots (loads tikz, wc itself loads xcolor, graphicx)
\usepackage{tikz-cd}% tikz derivative - for graph - testing
\usepackage[figuresright]{rotating}% to rotate table w LHS at bottom
%\usepackage{tabularx}% to set table width - prolly can avoid
\usepackage{hyperref}% for links - load last-ish
\usepackage{amsrefs}% for full citations - load last
%
%
% new commands to avoid formatting inline
\newcommand{\code}[1]{\texttt{#1}}% for code
\newcommand{\titleit}[1]{\textit{#1}}% for book titles etc
\newcommand{\mention}[1]{\textit{#1}}% to mention (not use) terms
\newcommand{\foreign}[1]{\textit{#1}}% for foreign words
\DeclareMathOperator{\pir}{pir}% for model sec
\DeclareMathOperator{\own}{own}% for model sec
\DeclareMathOperator{\id}{Id}% for pir(), log(), graphs
\DeclareMathOperator{\mt}{Mt}% for ditto
\DeclareMathOperator{\bc}{Bc}% for dtto
\DeclareMathOperator{\et}{Et}% for ditto
\DeclareMathOperator{\nt}{Nt}% for ditto
\DeclareMathOperator{\ab}{Ab}% for ditto - variable
\DeclareMathOperator{\cd}{Cd}% for ditto - variable
\DeclareMathOperator{\pre}{pre}% for log
\DeclareMathOperator{\pst}{pst}% for log
\DeclareMathOperator{\cpd}{cpd}% for log
%
%
% general theorem styles
\theoremstyle{definition}% for model
\newtheorem{model}{Model}
\theoremstyle{definition}% for claim
\newtheorem{claim}{Claim}
\theoremstyle{remark}% for appendix - testing
\newtheorem*{note}{Note}
\theoremstyle{definition}% for claim - testing
\newtheorem*{claim*}{Claim}
\theoremstyle{definition}% for defin - testing
\newtheorem{gloss}{Gloss}
%
%
% xcolor commands for darkmode - just for draft
\pagecolor{darkgray}
\color{white}
%
%
% offload tikz/pgfplots - got error when compiling
% \usepgfplotslibrary{external}
% \tikzexternalize
%
%
% tikz ie pgsplots, tikzcd commands
\pgfplotsset{compat=1.18}% apparently needed
%\tikzcdset{}% global style
%
%
\begin{document}
%
%
\title{pp}
\author{A Navidad}
\address{Harvard College, Cambridge MA}
\curraddr{Benque Viejo CY}
\email{\href{mailto:navidad@college.harvard.edu}{navidad@college.harvard.edu}}% href just for draft
\date{12 Feb 2025}
\thanks{\lipsum[1][1-3]}% incl non-grant support per ams
%
%
\begin{abstract}
\lipsum[1][1-6]
\end{abstract}
\keywords{\lipsum[1][1]}
%
%
\maketitle
%
%
%
%
% s intro
\section{Introduction}
\label{s:intro}
\lipsum[1-2]
%
%
%
%
% s done
\section{Dummy}
\label{s:dummy}
We present this paper's main claim here, and map out the support it will be given later on.
%
%
% main claim
\begin{claim*}[Main]
\label{clm:main}
\ref{map} and \ref{log} are a decent approach to work on the origins of Belize.% qualify 'up to isomorphism' ?
%
\begin{equation}
\tag{Map}
\label{map}
\begin{tikzcd}[arrows=dash]
\nt &\bc \rar \ar[rr,bend left] &\mt \rar &\id &\et
\end{tikzcd}
\end{equation}
%
For vertices as certain places in the Bay, and edges as a certain relation between them. The Bay is mapped from northwest to southeast, beginning in northern Yucatan (\(\nt\)), flowing down to northern Belize (\(\bc\)), then the mouth of the Bay (\(\mt\)), next the Bay Islands (\(\id\)), and finally eastern Honduras (\(\et\)). We say a place \(ab\) is connected to place \(\cd\) (\(\ab-\cd\)) if \(\ab\) is \mention{close} to \(\cd\), and not otherwise, ignoring trivial cases.\footnote{Eg where \(\ab=\cd\).} Primitively, spaces are close if they are geographically close.
%
\begin{equation}
\tag{Log}
\label{log}
\begin{tikzcd}[sep=huge]
t_{0} \rar["\pre" description] &t_{c} \rar["\cpd" description] &t_{d} \rar[bend left,"\pst_{0}" description] \rar["\pst_{1}" description] \rar[bend right,"\pst_{2}" description] &t_{1}
\end{tikzcd}
\end{equation}
%
For vertices as certain points in time, and arcs as certain series of events from one point to the other. The story relevant to Belize's origin is logged from start to finish, beginning at the earliest point (\(t_{0}\)), then a certain \mention{coupling} point (\(t_{c}\)), next an ensuing \mention{de}coupling point (\(t_{d}\)), and ending at the latest point in the story (\(t_{1}\)). These four points split our story into three parts, with a coupl\mention{ed} period (\(\cpd\)) in the middle, and a preceding \mention{pre}coupled period (\(\pre\)). The succeeding \mention{post}coupled period (\(\pst\)) is plotted as three series of events (\(\pst_{0}\), \(\pst_{1}\), \(\pst_{2}\)) rather than one, to account for three possibly distinct sequences of events. Basically, the coupling refers to the coincidence of piracy and logging in the Bay.
\end{claim*}
%
%
% map of support
We bolster \nameref{clm:main} by spelling out what makes for a decent approach to the relevant work, and then demonstrating that \ref{map} and \ref{log} are jointly decent. So we first clarify \mention{decency} here.\footnote{These are a working gloss and claims, so taken for granted.}
%
\begin{gloss}[Decency]
An approach to the relevant work is decent if its does not answer any \mention{thick} question and does at least one \mention{thin} one.% may add further eg 'in such a way as to only partially help answer at least one thick question', etc
\end{gloss}
%
Where a narrow answer is granular, a thick question is interpretation heavy, and a thin question is less so. For work on the origins of Belize, we claim:
%
\begin{claim}[Thickness]% incl w/in gloss environ ?
Thick questions are at least `When was Belize settled?' and `When was Belize founded?'
\end{claim}
%
\begin{claim}[Thinness]
Thin questions are at least `When was Belize used for piracy?' and `When was Belize used for logging?'
\end{claim}
%
The general idea here is that a thick question is interpretation heavy, while a thin one is less so, so a maximally useful quide ought to be open to interpretation, it seems.

We demonstrate both parts of decency jointly xx. % or take for granted first part, and only show 'answers a thin question' ?
Towards this end, we first build both \ref{map} and \ref{log}, and then employ them jointly as required.\footnote{We assume either singly would not be decent.}

by domonstrating the suitability of \ref{map} and \ref{log} as suitable guides to relevant work. Our workflow looks like so.
1- spell out what makes a guide suitable ie split origin story into -founding vs settlement questions and subquestions,
2. build map ground up to show internal qualities/suitability,
3. apply map alone to show unsuitabilty as itself
4. build log ground up as well,
5. apply log alone to show unsuitable.
6. apply together to show suitability.
%
%
%
%
% s claims
\section{Claim}% make claim
\label{s:claim}
We build some preliminaries here so as to make this paper's claim.
%
\subsection{Preliminaries}
\label{ss:prelims}
\subsubsection{Briefly}
\label{sss:briefly}
For any places \(\ab\) and \(\cd\), say \(\ab-\cd\) if \(\ab\) is close to \(\cd\) and vice versa. Then, for certain places in the Bay we have:
%
\begin{equation}
\tag{Test}
\label{}
\begin{tikzcd}[arrows=dash,sep=large]
\nt \rar[out=120,in=60,loop] &\bc \rar \rar[out=120,in=60,loop] \ar[rr,bend right] &\mt \rar \rar[out=120,in=60,loop] &\id \rar[out=120,in=60,loop] &\et \rar[out=120,in=60,loop]
\end{tikzcd}
\end{equation}
%
Now for any uses \(f\) and \(g\), say \(f(\ab)\to g(\cd)\) if the use of \(\ab\) for \(f\) lead to the use of \(\cd\) for \(g\). Then, for piracy \(p\) and logging \(l\) in particular, we have \(p(\ab)\to g(\cd)\) only if \(\ab-\bc\) and each edge in the path from \(\ab\) to \(\bc\) was the most profitable or least costly one available. In which case, for certain places in the Bay we have:
%
\begin{equation}
\label{}
p(\id)\to l(\bc)
\end{equation}
%
Next for uses \(f\), \(g_{0},\ldots,g_{n}\) and attitiude \(\alpha\), say \(f(\ab)\) if any \(g_{i}(\ab)\) for \(i\leq n\) and say \(\alpha(\ab)\) if the attitude towards \(\ab\) is one of attachment.% only for certain agents, certain places ?
%
\begin{gather}
f(\ab)\to\alpha(\ab)\\
\alpha(\ab)\to f(\ab)
\end{gather}
%
\subsubsection{Illustration}% insert actual historical example here
\label{sss:illustration}
Say the \titleit{Swallow} crew in Rattan wake up one sunny morning to discover they've missed the Honduras Ships at Truxillo. At this point, they've got a number of options. They might head east to Gracias a Dios or north to Catoche, but both these courses seem either wasteful or taxing (it's a longer trip to either than it is to stay put). But while some R\&R might be due, they all know what a waste that would be, the Ships being but a few leagues away in Amatique. So the crew agree, `Hey, Amatique's not so far away, and the Ships might still be there!' and set off due west of Rattan. On arriving, they are griefstricken to find the Ships has outfoxed them again. Now they are way in the Bay's mouth, right in the thick of Spanish defences, and the only way out is a taxing voyage due east or another due north. But `Oh!,' says Wallace, `Old River's just a hop away, and flush with logwood!' And so thinking it better to get at least something more than good memories, they sail up Belize's coast, cutting logwood, before making their merry way out.
\subsubsection{At length} We are almost all agreed that a faithful discussion of Belize's origins begins with piracy and ends with logging. There is, however, no consensus on the path from one to the other, nor on the time it took to traverse it. For now, however, all we need is agreement on the start and endpoints. So, assume those.

Now, while we have agreed on a \emph{thematic} starting point, we have yet no agreement on time nor place, with values for the former ranging over three centuries (16th to 18th), and for the latter extending so far out so as to subsume at least two landmasses (Central America incl Yucatan, and the Greater Antilles). At this point we might either stop and cosider each proposal indivually (a tall order), or else sort them into bins and consider each bin collectively (still taxing, but doable). For now, however, we only need the \emph{geographic} starting point, leaving the \emph{temporal} one for later. So, we first sort by geographic starting points. Take all proposals with none such, or with the Bay proffered as the point, as discard. We are left with either proposals with starting points \emph{in} the Bay (the Bay Islands and north-central Belize, mainly) or outside of it (Providence, southern Mosquito Shore, Jamaica, Campeachy, Tortuga, mainly). These, of course, look like a list of pirate haunts in the west West Indies, and as our thematic starting point would seem to require, it's jsut where the pirates frequented that we'd want to set out geographic starting point. Now, if we count an entrance into the Bay from the Caymans as eastern, then any pirates outside the Bay would have to enter throught the north or east. Count these as two geographic starting points, named \(\nt\) and \(\et\) respectively, and further call the Bay Islands \(\id\) and north-central Belize \(\bc\).

Now, before calling only these four our geographic starting points, assume here one more -- the mouth of the Bay, and call it \(\mt\). The reasons for this are like so. First, we agree or concede that \emph{all} pirates in haunts outside the Bay would enter via \(\nt\) or \(\et\), such that those two points act as proxies for all of those. Since we have all haunts outside of the Bay, then, we would naturally want all haunts within it as well. It will be part of the work of this paper to establish \(\mt\) as a pirate haunt (at least up to the criteria used for \(\bc\)). For now, we take it for granted, leaving us with three inner-Bay and two outer-Bay geographical starting points.

Here we stop to notice a salient feature of our chosen geo starting points. The inner points are each close to themselves (of course) and the two others (some 30 leagues), while the outer ones are close only to themselves (some 60 leagues from their nearest non-identical point). We might represent this (and other factors eg energy expended in getting from one to another point, cost, etc, but not importing knowledge/practice/experience into weights) as a \mention{closeness} relation, and graph those sufficiently close as being connected, and those not as being disconnected, eg in graph above.

We further take stock of a feature in our agents' move from start (piracy) to end (logging), namely, the \emph{reason} for which said move was taken. A healthy number of reasons have been offered in literature, but it will not be the work of this paper to support one or the other. Rather, here we note that the reasons offered are broadly economic (looking for profit, declining Spanish commerce, etc) vs non-economic (criminalisation, conscience, etc) and short- vs long-term (reasons for action in \emph{this} expedition/voyage vs life choices generally). In this paper, we import only short-term economic reasons (incl the usual rational choices eg staying safe, minimising energy expenditure, etc), and disregard any long-term or non-economic ones.% rest of paper might work with long or non-econ ones though - not sure

If the above two features are granted, then we immediately have a first pass of the path we were looking for from piracy to logging. Namely, if closeness relations between our geo points stand as above graphed (with closeness taking cost into account), and if pirates made short term econ decisions (seeking immediate profit), then a rough, minimal model of such short-term decision-making might look like the arrow function above. This again will not be established in this paper, but rather taken as following from the two features granted above.% of maybe show another function wouldn't work?

Now, if we grant that the said function models \emph{short} term decision for pirates of interest to us, then we are probably left with at least hundreds of such decisions taken and not taken by crews, or thousands taken and not taken by individual pirates, ie a mess of which we don't have a faint nor clear picture (due to lack of records, bias in records to covering piracy and not logging, naturally, etc).

However, say the short-term piracy to logging decision were taken significantly more often than not after a certain point/by certain agents/under certain conditions. Call these \emph{coupling} conditions %logging couples w pirating at signigicant levels here
and assume them for now (will return later to them). Then we might not really need to see both input and output of the funciton, and might instead take pirating decisions as proxies for logging decisions under these coupling conditions. While this wouldn't give us a granular view of short-term instances, we'd now have a rough picture of long term trends. That is, at least under coupling conditions, we might a sketch of the piracy to logging move. It will be the work of this paper to offer possible coupling conditions, but not to establish any with any certainty at all.

If the coupling conditions are granted, then decoupling ones ought to be too, given that we are widely agreed that logging outlived piracy in the West Indies. Taken as granted/following from above, so not established here. However, it will be the work of this paper to suggest (but not establish) decoupling conditions as well.

What this means for us is that our picture of piracy as proxy for logging has an expiration date, and so can only give us a glimpse at the piracy to logging path, and so is of limited use.

At this low point we take a break to again take stock of our position. We still have our thematic starting and ending points, our geographic starting points, the closeness relation between these, the (econ short) reasons for our agents' move from thematic start to end, the resulting minimal model of their short econ decisions, the coupling and decoupling conditions, and the proxy picture of logging decisions between said conditions, and nothing else. Actually, in addition to this, we now note that we have had from the beginning our geographic end point, naturally, ie Belize ie \(\bc\) or \(\mt\) or a sum of these. We have nothing else though (importantly missing temporal points).

At this post-decoupling point, it seems that the rough minimal model of short econ pir to log decisions would no longer work, given the now significant positive coincidence. So the function too only models short term decisions up to decoupling. Ignore this for now.% prolly coupling = rise of buccaneers, and decoupling = fall of Sp commerce/easy booty

Now, we were prior agreed that logging outlived piracy. Here, further assume that piracy outlived decoupling. (This point to be established in paper.) 

Then we would have one of between the temporal points of the decoupling and the fall of piracy (for convenience here assume there was significant logging at least by fall of piracy). Either there was no or insignificant logging during the said points, or there was at least some significant logging during the same. % remember coupling meant near 1:1 pir:log significant positive coincidence eg >= +0.7, so after decoupling we have either insignificant coincidence eg -0.7 to +0.7, or significant negative coincidence eg <= -0.7

And further, either there was no significant piracy:logging coincidence, or there was at least some significant coincidence (negative as decoupling occurred, by definition, when significant positive coincidence was no more). So in total we have four cases for this time period. % ie no signif coinc + sig logging, no signif coinc + no sig logging, signif neg coinc + sig logging, signif neg coinc + no sig logging
However, the case where we have no significant logging and significant negative piracy:logging coincidence is not possible, given we've presumed the fall of piracy during this period. So really we have three cases here.

It is the work of this paper to discuss possible paths within each of these cases.

This done, we will temporally be at the fall of piracy, at which point, per assumption, logging is occurring at significant levels, and we have reached the end of the piracy to logging path at some undefined temporal endpoint.

It will be the work of this paper to support now some (not all) of the above asked for grants/assumptions, and to offer possible temporal start and end points.
% for (decoupling point)
% nil log + neg coinc NOT POSSY DURING PIR FALL
% nil log + nil coinc (possy ie independent)
% sig log + neg coinc (possy during pir fall)
% sig log + nil coinc (possy ie independent) 
%
\subsection{Map}
\label{ss:map}
So for temporal start point \(t_{0}\) and end point \(t_{1}\), and coupling and decoupling points \(t_{c}\) and \(t_{d}\), we have \(0\leq c<d< 1\) (as per above). % decoupling after coupling, and decoupling before out end point
Then we have \mention{precoupled} (\(t_{0}\) to \(t_{c}\), possibly null), \mention{coupled} (\(t_{c}\) to \(t_{d}\)), and \mention{decoupled} (\(t_{d}\) to \(t_{1}\)) periods. And for each, we have 0 or 1, 1, and 3 cases to review (as per above). And at \(t_{1}\) we have \(l(\bc)\).
%
\begin{equation}
\label{}
\begin{tikzcd}[sep=huge]
t_{0} \rar &t_{c} \rar \rar[bend left] \rar[bend right] &t_{d} \rar &t_{1}
\end{tikzcd}
\end{equation}
%
And in outline:
\begin{enumerate}
\item Get thematic start and end points (piracy, logging) (min grant).
\item Add spatial points (\(\nt\), \(\bc\), \(\mt\), \(\id\), \(\et\)):
	\begin{itemize}
	\item Get those in Bay from literature (min grant),
	\item Change those out of Bay from literature to \(\nt\) and \(\et\) (min/med support),% min support prolly
	\item Add \(\mt\) (med support).
	\end{itemize}
\item Add closeness of spatial points (min support).
\item Add short-term, economic agential reasons for move from thematic start to end point (min support).
\item Add arrow function (min support).% to show adequacy
\item Add coupling conditions (max support).% rise of resident pirates (incl foreign colonies, entrepots), fall of home country attachment, increase in smuggling, 
\item Add decoupling conditions (min support).
\item Add sketch of coupled period (min/med support).
\item Add sketch of precoupled period (min support).
\item Add sketch of decoupled period (med support).
\end{enumerate}
%
%
\subsection{Claim}
\label{ss:claim}
%
%
%
%
% s intro
\section{Graph}
\label{s:graph}
The piracy-to-logging pipeline is generally accepted in literature. In service of this gloss, when discssing Belize's origins, we usually start by naming pirate havens in the Bay, usually Belize and the Bay Islands, in addition to relevant non-Bay havens to the east and south, eg Mosquito Shore, Tortuga, so on, and to the north and west, eg Catoche, Campeachy, etc. Generally, then, we are basically left with four havens to discuss Belize's origins. If we were to plot these havens physically, their graph might look like Graph~\ref{graph:lit}, where north-central Belize (Bc) is pretty close to the Bay Islands (Id) ie \(\bc\to\id\), but neither of them is pretty close to non-Bay havens (Mosq Shore ie Bluefields or Prov, Campeachy, etc) eg \(\id\dashrightarrow\bc\).
%
\begin{equation}
\label{graph:lit}% contiguity graph of havens in literature - or [r,dashed] or dotted] for less conqiguous
\begin{tikzcd}[arrows=leftrightarrow]
\nt \rar[dashed] &\bc \rar &\id \rar[dashed] &\et
\end{tikzcd}
\end{equation}
%
The reasons for this selection, rather than another, seem to be the havens' prominence and utility. For instance, the Bay Islands are well attested in primary sources, and are of course physically prominent in life and maps, and are strategically placed and bountiful. If we were to focus on a location's \emph{strategic} value, however, a new haven is possible -- the Bay's mouth.
%
\begin{equation}
\label{graph:prop}% proposed contiguity graph of havens - add dashed arrow for increased cost eg makes less sense/costs more - prolly no need of/not enough evidence for Vd (Valladolid/N Yuc) haven area
\begin{tikzcd}[arrows=leftrightarrow]
\nt \rar[dashed] &\bc \rar \ar[rr,bend left] &\mt \rar &\id \rar[dashed] &\et
\end{tikzcd}
\end{equation}
%
The identification of havens is important here in that piracy-to-logging pipeline is generally assumed to have been haphazard ie by convenience. That is, while the selection of havens was purposeful (following the Spanish ships), the selection of a logging camp from here (once pirates were in their haven/when their booty didn't materialise) is thought to have been a matter of `Where's the nearest logwood stand?' % so want all paths to logging in Bc as justified from prim sources/evidence
Which is to say, piracy in Bay Islands leading to logging in them (ie \(\id\leadsto\id\)) is much more likely than \(\id\leadsto\mt\) which is similarly more likely than 
\(\id\leadsto\et\). Furthermore, if we were to have weighted directed edges in the graph above (eg by availabilty/ease of logwood, distance/difficulty to travel, presence of Spanish), then (ignoring self edges), for instance, \(\id\leadsto\bc\) would be likelier (eg more logwood in Bc) than the inverse.

In addition, a lot of literature gives a \(\bc\leadsto\bc\) or \(\id\leadsto\bc\) account, given the paucity of evidence we have. (With minorities giving an \(\nt\leadsto\bc\) or \(\et\leadsto\bc\) account.) In this paper we support the \(\id\leadsto\bc\) pipeline, and offer an easier \(\id\to\mt\to\bc\) path for it.

Also, a lot of literature dates the given pipelines to oddly early and oddly specific dates (eg 1638). The \(\id\to\mt\to\bc\) path might better rationalise or explain these datings than existing ones.

In literature we identify, in addition to the accepted piracy-to-logging pipeline, the logging-to-logging and other-to-logging pipelines, eg Restall for former, Winzerling for latter. In addition, variants exists on whether the path was haphazard (logging camp was just close by/convenient) or intentional (specifically chosen), in addition to other variations (sort of pirates -buccaneers, etc, their provenance -Jam, etc). The log-to-log is just a nested pir-to-log pipeline over widers stretch of time, so really the types are pir-to-log or not-pir-to-log (and the timeline for each). So for any given start Ab and end Cd, we have short and lengthy pir-to-log (ie \(\ab\leadsto\cd\) and \(\ab\leadsto^{*}\cd\) xx.
%
%
%
%
% s claims
\section{Claims}
\label{s:claimsold}
Let \(s\) be S~Bay, \(w\) W~Bay, \(n\) N~Bay, \(\to\) the historical \mention{lead to} relation, \(t_{i}\) a point in time, \(\pir(i)\) the use of place \(i\) as a pirate haven, \(\log(i)\) of \(i\) as a logging camp, \(\own(i)\) the thinking of place \(i\) as one's own. Say \(\pir(s)\) at \(t_{0}\) and \(\own(s)\) at \(t_{1}\) and \(\log(w)\) at \(t_{i}\) and \(\own(w)\) at \(t_{j}\).\footnote{Where \mention{S Bay} is the southern Bay of Honduras and Spanish ports there (Truxillo, Pto Caballos, Sto Tomas, etc) and waters off this coast and these ports to distance from which pirates could spy on ports and shipping lanes (ie incl Bay Isl and cayes off Toledo), \mention{W Bay} is similarly defined but for western Hond Bay and ports and lanes there (Tamalcab for Bacalar ie incl cayes off Belize, Corozal), \mention{N Bay} is similarly defined but for northern Hond Bay and ports and lanes there (Valladolid's port and Yucn Channel ie incl Cozumel probably).}
%
%
% model 1
\begin{model}
\label{m1}
Say \(\pir(s)\to\log(w)\) at \(t_{k}\) for \(1\leq k\). % ie log(w) at t_{k}
Then buccaneers thought of \(\pir(s)\to\log(w)\) as an expansion of their territory. Call their resulting territory \mention{Bay} and let \(b\) be it. % ie Bay came about at t_{k}
Then at and after \(t_{k}\), they \(\own(b)\) since \(t_{1}\).
\end{model}
%
%
% model 2
\begin{model}
\label{m2}
Say \(\pir(s)\to\log(w)\) at \(t_{k}\) for \(1\not\leq k\). % ie log(w) at t_{k} before own(s)
Then buccaneers did \emph{not} think of \(\pir(s)\to\log(w)\) as an expansion of their territory. So \(\pir(s)\to\log(w)\) did not result in an expansion of territory.
\end{model}
%
%
% model 3
\begin{model}
\label{m3}
Say \(\pir(s)\not\to\log(w)\) at any \(t_{k}\). % eg via shipwreck, settled intentionally, 'lead to' indirectly rather than directly
\end{model}
%
%
Then we probably want to claim and show that Model~\ref{m1} is more fitting than \ref{m2} or \ref{m3}. % prolly is just a refinement on existing published lit
Further, we may want to sketch out responses for `When was Belize settled or founded?'% ie from permanent settlement, from whenever buccaneers/Baymen thought they owned it, from first stay at in Bz during wet season (w/o logging), from ca Bacalar withdrawal, etc
%
%
% claim 1
\begin{claim}
\label{c1}
Model~\ref{m1} is more fitting than: 
\begin{enumerate}
\item Model~\ref{m2}, and 
\item Model~\ref{m3}.
\end{enumerate}
\end{claim}
%
%
% claim 2
\begin{claim}
\label{c2}
Fitting narrow answers to `When was Belize settled?' are exactly: On the earliest date from which there has been% excl broad ansers like 17th cent etc
\begin{enumerate}
\item \emph{no} gap in presence ie continuous presence, or% on last withdrawal to Mosq Shr
\item \emph{only} acceptable gaps in presence.% few number of gaps, gaps of only x yrs, forced gaps, gaps to Mosq Shore, gaps after z yrs of gapless presence
\end{enumerate}
\end{claim}
%
%
% claim 3
\begin{claim}
\label{c3}
Fitting narrow answers to `When was Belize founded?' are exactly: On the date of
\begin{enumerate}
\item the creation or constitution of its earliest predecessor state, or% not reconstitution, and assuming constitution requires settlement
\item null.
\end{enumerate}
\end{claim}
%
%
%
%
% s sketch
\section{Sketches}
\label{s:sketches}
%
%
%
%
% ss c1
\subsection{For Claim~\ref{c1}}
For these, we'd want to see or assume that buccaneers came to think of their havens as de facto theirs. The pirate's haven to logging camp sequence is assumed already in literature. Model~\ref{m3} is already prevalent in literature so is fitting to extent that it's been considered. Most lit mentions of Mdl~\ref{m3} I think carve Hond Bay differently or not at all, but stick to have-to-camp sequence. I don't think I've seen Mdl~\ref{m1} in lit, and prolly hardest part would be to show buccaneers would in fact see pir-to-log as territory expansion, though in theory sounds about right. Plus giving probable dates. Mdl~\ref{m2} I don't think has come up, and could prolly ignore.
\subsubsection{First item} Ignore this one.
\subsubsection{Second item} Assume Mdl~\ref{m3} is fitting (as has been published and is prevalent in lit). Show Mdl~\ref{m1} is at least as fitting (ie not explicitly contradicted by existing evidence). Finally, show Mdl~\ref{m1} is more fitting in at least some aspects (Hond Bay carving into S, W, N makes more sense than other carvings given pirate aims/practices and given Bacl/Valladolid barely targetted in first half of 17th cent, etc).
%
%
%
%
% ss c2
\subsection{For Claim~\ref{c2}}
\subsubsection{First item} Straightforward, but not claimed in lit. Give acceptable date (1760s iirc)
\subsubsection{Second item} Most commonly claimed in lit but sans discussion of acceptability criteria. List acceptable dates per each criterion (vars).
\subsubsection{Other cases} Other cases mostly disproved by prior lit, or else proposed tentatively, or else using messy/tricky criteria for which gaps in presence to accept/reject.
%
%
%
%
% ss c3
\subsection{For Claim~\ref{c3}}
\subsubsection{First item} Straightforward criteria, but tricky application re whether to accept predecessor state with impermanent settlement.
\subsubsection{Second item} Null.
\subsubsection{Other cases} Usually re recognition by other states (eg Spain recognising via treaty, UK recognising by claim/sending Supt/making colony/self-gov/etc, or via independence from UK/Jam/Spain). Not sure how to show these are ill-fitting other than by reference to first item. Could resort to intra-criteria inconsistencies (counting UK-independence as founding date vs Spain-independence, etc) or inter-criteria ones (maybe criteria applied to other Commonwealth unitary states follows \(x\) trend, so we ought to follow it too in Bz's case).
%
%
%
%
% s support
\section{Support}
\label{s:support}
% ss c1
\subsection{For Claim~\ref{c1}} We first present the models, then build support for the claim itself.
\subsubsection{On Mdl~\ref{m1}}
\subsubsection{On Mdl~\ref{m3}}
\subsubsection{On Mdl~\ref{m1} vs Mdl~\ref{m3}}
%
%
%
%
% s concl
\section{Conclusion}
\label{s:concl}
\lipsum[1][1-6]
%
%
%
% 
% s refs
\begin{bibdiv}
\label{s:refs}
\begin{biblist}
\bibselect{pprefs}
\end{biblist}
\end{bibdiv}
%
%
%
%
\newpage% break - just for draft
\appendix% no \cite below here
%
%
% s terms
\section{Terms}% incl gazetteer, glossary
\label{s:terms}
Senses not used in this paper tagged with asterisk.
%
\subsection{Gazetteer} Read personal identifications here as further importing the named entity's coincident physical features, eg read \ref{term:bz}.i as having `or their territory' appended.\footnote{Not the inverse though.}
%
% sort names alphabetically, sort meanings logically?
\subsubsection{Bacalar} One of (i) modern city or predecessors, (ii) Pachecos distrito or successors, (iii) site on southwest lakeshore.
%
\subsubsection{Bay}\label{term:bay} One of (i) \ref{term:hondbay}.i or \ref{term:hondbay}.ii here, (ii) predecessors in \ref{term:bz}.i here, (iii) xx.
%
\subsubsection{Bay Islands}
%
\subsubsection{Belize}\label{term:bz} One of (i) modern state or predecessors, (ii) river or its watershed or mouth or tributaries, (iii) modern city or predecessors, (iv) modern district or predecessors.
%
\subsubsection{Campeachy} Campeche.
%
\subsubsection{Guanaxa} Guanaja.% hist also Bonacca
%
\subsubsection{Honduras} One of (i) modern state or predecessors or their subdivisions, (ii) \ref{term:bay} here, (iii) \ref{term:bz} here.
%
\subsubsection{Honduras Bay}\label{term:hondbay} One of (i) gulf bound by line from Catoche to Gracias a Dios, (ii) by line from Herrero to Camaron, (iii) by line shorter than \ref{term:hondbay}.ii here.
%
\subsubsection{Old} \ref{term:bz}.ii here.
%
\subsubsection{Providence} Old Providence.
%
\subsubsection{Rattan} Ruatan.
%
\subsubsection{Shore} Mosquito Shore.
%
\subsubsection{Tris}
%
\subsubsection{Truxillo} Trujillo.
%
\subsubsection{Yucatan} One of (i) \ref{term:yucpen} here, (ii) Velazquez de Cuellar adelantado or successors.
%
\subsubsection{Yucatan Peninsula}\label{term:yucpen} One of (i) headland bound by line from Terminos to Amatique, (ii) by line north of \ref{term:yucpen}.i here.
%
\subsection{Glossary}
%
% sort terms alphabetically - yes, sort meanings logically - ?
\subsubsection{Bayman} One of (i) resident of \ref{term:bz}.i, (ii) member of certain camps or havens in Bay or Campeachy.
%
\subsubsection{buccaneer}\label{term:bucc} One of (i) \ref{term:pir}.i or \ref{term:pir}.ii or \ref{term:pir}.iii here, (ii) \ref{term:bucc}.i here resident in the West Indies incl Bay, (iii) member of a certain group of foreigners from northwest Hispaniola.
%
	\begin{note}[residence]
	For \ref{term:bucc}.ii, \emph{resident} as in spending some sufficiently significant share of total time here, or of time off here, or holding some sufficiently \emph{in}significant sense of attachment to their home country.
	\end{note}
%
	\begin{note}[presence]
	For \ref{term:bucc}.ii, assume present in sufficiently significant numbers by at least the 1620s.% add 'near Bay'?
	\footnote{Origins of ii in lit xx. Origins of iii in lit xx.}
	\end{note}
%
\subsubsection{camp} Logging or pirate camp.% not for pir-offensive?
%
\subsubsection{foreigner} Not Spanish.% incl black, native as 'foreign'
%
\subsubsection{haven} Pirate haven.% for def and offence and off time
%
\subsubsection{logwood} Dyewood or \foreign{H campechianum}.
%
\subsubsection{Mosquito} Miskitu.
%
\subsubsection{pirate}\label{term:pir} One of (i) anyone so called by the Spanish, (ii) foreigner not in an official or sanctioned fleet or ship or colony or settlement, (iii) anyone per any sensu lato meaning in literature, (iv) per any stricter senses.
%
\subsubsection{Shoreman}
%
\subsubsection{Spanish} Hispanic or Hispanicised.
%
%
%
%
% s data
\section{Data}
\label{s:data}
\lipsum[2][1-4]
Data available in \code{json} at \href{http://github.com/aenavidad/pp-0}{github.com}.
\subsection{Scope} Covers all available 17th cent activity in Hond Bay.
\subsection{Work} As collected from all accessible published literature, plus select AGI records in the Guatemala and Mexico series.
%
%
%
%
% fig 1
\begin{figure}
% pgsplots code begins
\begin{sideways}
\begin{tikzpicture}
\begin{axis}[
	xmin=1601, xmax=1700,
	xtick={1610,1620,1630,1640,1650,1660,1670,1680,1690},% to not display all yrs
	xticklabel style={/pgf/number format/1000 sep=},% to not add comma
	enlarge x limits=0.02,% to not cut off y bars
	ymin=0, ymax=7,
	ybar stacked,% use y bars and stack
	bar width=0.6,% for thinner bars
	width=1\textheight,height=1\textwidth,% expand sideways fig
]
\addplot [% plot for piracy incidents in bz
	black,fill=black,% colour outline and fill
	] table [x=yr,y=nobz] {ppfigpirdata.dat};
\addplot [% plot for piracy incidents outside of bz
	gray,fill=gray,% colour outline and fill
	] table [x=yr,y=noall] {ppfigpirdata.dat};% check noall value *excludes* nobz value to stack
\end{axis}
\end{tikzpicture}
\end{sideways}
% pgsplots code ends
\caption{Piracy incidents in the Bay in the 17th cent (black in Bz).}
\label{fig1}
\end{figure}
%
%
%
%
% tab 1
\begin{table}% sidewaystable is no go in amsart apparantly
\caption{Claims on Bz origins in 18th to 21st cent literature.}
\label{tab1}
% tab code begins
\begin{sideways}
\begin{tabular}{cp{.4\textheight}lp{.3\textheight}cc}
No	&Date	&In	&Claim	&Yes	&No\\
(1)	&mid 16th cent	&--	&Eng pirates form community	&1	&0\\% campbell
(2)	&ca Sp Hon flotilla est	&Old	&Wallace settles	&?	&?\\% asturias - first in live wiki
(3)	&ca 1610	&Old&Wallace settles	&?	&?\\% egli
(4)	&pre/ca Fuensalida missions	&N	&Brit log	&?	&?\\% roys
(5)	&in 1617	&--	&Wallace present&&\\% asturias
(6)	&pre penult Bacl sacking	&Old	&Eng log	&&\\% reads 1732
(7)	&ca Prov est	&S	&Provers seek refuge or farm	&&\\% winzerling
(8)	&mid 17th cent	&--	&Wallace settles	&&\\% fenix 1849
(9)	&pre treaty	&?	&Brit present	&&\\% mcculloch
(10)	&mid third 17th cent	&--	&Wallace present	&&\\% ancona
(11)	&in 1638	&--	&Brit wreck	&&\\% HA 1829
(12)	&in 1638	&--	&Wallace wrecks	&&\\% HA 1827
(13)	&in 1640	&--	&Wallace wrecks	&&\\% ungewitter
(14)	&in Cromwell govt	&--	&Eng seek haven	&&\\% HA 1826
(15)	&post Bacl/Trux sacking or post Tris logging	&--	&Brit seek haven or log	&&\\% b-thomas
(16)	&last half 17th cent or pre early 1680s	&--	&Eng present	&&\\% finamore
(17)	&post Jam invasion or post Tris logging	&--	&Jamers log	&&\\% robertson
(18)	&ca 1663	&--	&Wallace logs	&&\\% g diaz
(19)	&in 1662	 or pre 1670&--	&Wallace logs	&&\\% c quijano
(20)	&post Catoche/Tris logging	&--	&Jamers log	&&\\% ancona
(21)	&post treaty	&--	&Wallace present	&&\\% n ortega
(22)	&last third 17th cent	&--	&Eng present	&&\\% ancona
(23)	&early 18th cent	&--	&Eng seek haven	&&\\% brockhaus
(24)	&post Tris logging	&--	&Wallace seeks haven or logs	&&\\% ancona - last in live wiki
(25)	&post Tris logging	&--	&Eng log	&&% restall
\end{tabular}
\end{sideways}
% tab code ends
\end{table}
%
%
\end{document}